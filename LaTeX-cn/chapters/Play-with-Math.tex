%!TEX root = ../LaTeX-cn.tex
\chapter{數學排版}
\section{行間與行內公式}
\co{行內公式}指將公式嵌入到文段的排版方式,主要要求公式垂直距離不能過高,否則影響排版效果。行內公式的書寫方式:
\begin{latex}
$...$ 或者 \(...\) 或者 \begin{math}...\end{math}
\end{latex}

一般推薦第一或第二種方式。例:\verb|$\sum_{i=1}^{n}a_i$|,即:$\sum_{i=1}^{n}a_i$.

另外一種公式排版方式是\co{行間公式},也稱行外公式,使用:
\begin{latex}
\[...\] 或者 \begin{displaymath}...\end{displaymath}
或者 amsmath 提供的 \begin{equation*}...\end{equation*}
\end{latex}

一般也推薦第一種命令\footnote{還有一種\texttt{\$\$...\$\$}的寫法,源自底層\TeX,不建議使用。},例如:\verb|\[\sum_{i=1}^n{a_i}\]|,得到:
\[\sum_{i=1}^{n}a_i\]

從上面的兩個例子可以看出,即使輸出相同的內容,行內和行間的排版也是有區別的,比如累加符號上標是寫在正上方還是寫在右上角。

如果行間公式需要編號,使用\envi{equation}環境\footnote{需要注意有一個\RED{已被放棄}的多行公式編號環境叫\texttt{eqnarray},請不要再使用。},還可以插入標籤:

\begin{codeshow}
\begin{equation}
\label{eq:NoExample}
  |\epsilon|>M
\end{equation}
\end{codeshow}

\section{數學字體、字號與空格}
\label{sec:mathfont}
\subsection{空格}
在數學環境中,行文空格是被忽略的。比如\verb|$x,y$|和\verb|$x, y$|並沒有區別。數學環境有獨有的空格命令,最後一個是$-3/18$的空格:

\begin{codeshow}
  $沒有空格,3/18空\,格$ \\
  $4/18空\:格,5/18空\;格$ \\
  $9/18空\ 格,一個空\quad 格$ \\
  $兩個空\qquad 格,負3/18空\!格$
\end{codeshow}

事實上,以上命令也可以在數學模式外使用,其中使用最廣泛的是\latexline{,},比如上文提到過的千位分隔符。在數學環境中它也應用廣泛(含有隱式的\latexline{,}):

\begin{codeshow}
\[ \int_0^1 x \ud{}x
= \frac{1}{2} \]
\end{codeshow}

其中\verb|\ud|命令是自定義的,這也是微分算子的正常定義\label{cmd:ud}:
\begin{latex}
\newcommand{\ud}{\mathop{}\negthinspace\mathrm{d}}
\end{latex}

\subsection{間距}
命令\latexline{abovedisplayskip}和\latexline{belowdisplayskip}控制了行間公式與上下文之間的間距,並且該值不會隨字號調整而調整。有時你需要自行指定。默認值\texttt{12pt plus 3pt minus 9pt}。多行公式之間的間距用\latexline{jot}來控制,默認\texttt{3pt}。命令\latexline{mathsurround}給出了行內公式與文字間,除了預留空格之外的間距,默認值為\texttt{0pt}。另外一個有趣的命令\latexline{smash},可以忽略參數的全高:
\begin{codeshow}
\[\underline{\smash{\int f(x)\ud x}}=1\]
\end{codeshow}

也能夠通過參數,選擇只忽略高度(t)或只忽略深度(b):
\begin{codeshow}
$\sqrt{A_{n_k}} \qquad
\sqrt{\smash[b]{A_{n_k}}}$
\end{codeshow}

\subsection{字號}
\LaTeX\ 提供四種字號尺寸命令:
\begin{para}
\item[\latexline{displaystyle}] 行間公式尺寸。如$\displaystyle \sum_{i=1}^n a$
\item[\latexline{textstyle}] 行內公式尺寸。如$\textstyle \sum_{i=1}^n a$
\item[\latexline{scriptstyle}] 上下標尺寸。如$\scriptstyle \sum_{i=1}^n a$
\item[\latexline{scriptscriptstyle}] 次上下標尺寸。如$\scriptscriptstyle \sum_{i=1}^n a$
\end{para}

\subsection{數學字體}
將字體轉為正體使用\latexline{mathrm}命令。如需保留空格,使用\latexline{textrm}命令——這與正文一致。但是,\latexline{textrm}命令內的字號可能不會自適應,\latexline{mathrm}則表現起來穩定得多。

例如自然對數的底數$\ue$,在本文中就是這樣定義的:
\begin{latex}
\newcommand{\ue}{\mathrm{e}}
\end{latex}

以下簡單介紹幾種數學字體。數學字體的總表參見\tref{tab:mathfont}。

\subsubsection{數學粗體}
數學粗體使用\pkg{amsmath}宏包支持的\latexline{boldsymbol}命令。命令\latexline{boldmath}的問題在於它只能加粗一個數學環境,其中很可能包括了標點符號,而這是不嚴謹的。命令\latexline{mathbf}就差的更遠,它只能把字體轉為\textbf{正}粗體,而數學字體都是斜體的。

\begin{codeshow}
$\mu,M$\\ $\boldsymbol{\mu},
\boldsymbol{M}$
\end{codeshow}

\subsubsection{空心粗體}
空心粗體使用\pkg{amsfonts}或\pkg{amssymb}宏包的\latexline{mathbb}命令。這裏用\latexline{textrm}而不是\latexline{mathrm},是為了保留空格。

\begin{codeshow}
$x^2 \geq 0 \qquad
\textrm{for all }x\in\mathbb{R}$
\end{codeshow}

\section{基本命令}
基本函數默認用正體書寫,包括:
\begin{verbatim}
\sin \cos \tan \cot \arcsin \arccos \arctan \cot \sec \csc
\sinh \cosh \tanh \coth \log \lg \ln \ker \exp \dim \arg \deg 
\lim \limsup \liminf \sup \inf \min \max \det \Pr \gcd
\end{verbatim}

以上函數,最後一行的10個是可以帶上下限參數的,即在行間公式模式下,上標和下標將在函數正上方和正下方書寫內容。

\pkg{amsmath}宏包允許\latexline{DeclareMathOperator}命令自定義基本函數,用法類似於\latexline{newcommand}命令。如果命令帶星號\latexline{DeclareMathOperator*},則可以帶上下限參數。

此外有一個叫\latexline{mathop}的命令,可以把參數轉換為數學對象,使其能夠堆疊上下標;\latexline{mathbin}與\latexline{mathrel}則分別能把參數轉換為二元運算符、二元關係符,並正確設置兩側的空距。

\subsection{上下標與虛位}
用低劃線和尖角符表示上標和下標,請仔細體會下述例子:

\begin{codeshow}
$a^3_{ij}$ \\
${a_{ij}}^3\text{或}a_{ij}{}^3$\\
$\mathrm{e}^{x^2}\geq 1$
\end{codeshow}

上面的指數3的位置讀者可以多多體會一下。此外,\latexline{phantom}被稱為虛位命令,從下例你也能夠體會到他的作用:

\begin{codeshow}
${}^{12}_{6}\mathrm{C}$ \\
${}^{12}_{\phantom{1}6}
\mathrm{C}$ \\
$a^3_{ij}$ \\
$a^{\phantom{ij}3}_{ij}$
\end{codeshow}

宏包\pkg{mathtools}提供了\latexline{prescript}來避免手工調整:
\begin{codeshow}
$\prescript{12}{6}{\mathrm{C}}$
\end{codeshow}

\subsection{微分與積分}
導數直接使用單引號\verb|'|,積分使用\latexline{int}符號:

\begin{codeshow}
$y'=x \qquad \dot{y}(t)=t$ \\
$\ddot{y}(t)=t+1$
$\dddot{y}+\ddddot{y}=0$ \\
$\iint_{D}f(x)=0$
$\int_{0}^{1}f(x)=1$
\end{codeshow}

有時候需要更高級的微分或積分號,其中\latexline{ud}命令在\hyperref[cmd:ud]{上文這裏}定義過:
\begin{codeshow}
\[\left.\frac{\ud y}{\ud x}\right|_{x=0}\quad
\frac{\partial f}{\partial x}
\quad\oint\;\varoiint_S \]
\end{codeshow}

其中的\latexline{dot}系的導數形式\LaTeX\ 只原生支持到二階導數。後面的三階、四階需要\pkg{amsmath}宏包。\latexline{int}系的積分命令類似。而環形雙重積分命令\latexline{varoiint}需要\pkg{esint}宏包\footnote{該宏包可能與\pkg{amsmath}衝突,即便使用也請其放在\pkg{amsmath}之後加載。}。

\latexline{left.}或\latexline{right.}命令\footnote{參考\hyperref[subsec:delimiter]{定界符}部分的內容。}只用於匹配,本身不輸出任何內容。

\subsection{分式、根式與堆疊}
分式使用\latexline{frac}命令。或者\pkg{amsmath}宏包支持的\latexline{dfrac}、\latexline{tfrac}命令來強制獲得行間公式、行內公式大小的分數。如果想自定義分式樣式,參考\secref{subsec:binom}一節的\latexline{genfrac}命令。
\begin{codeshow}
\[\frac{x}{y}+\dfrac{x}{y}
+\tfrac{a}{b}\]
\end{codeshow}

該宏包還支持另一個命令\latexline{cfrac},用於輸入連分式。
\begin{codeshow}
\[\cfrac{1}{1+\cfrac{2}{1+x}}\]
\end{codeshow}

空根式用\latexline{surd}輸出,更常用的是\latexline{sqrt}:
\begin{codeshow}
$\sqrt{2} \qquad \surd$\\
$\sqrt[\beta]{k}$
\end{codeshow}

開方次數的位置可以用這兩個命令微調,參數是整數:
\begin{codeshow}
$\sqrt[\leftroot{-2}\uproot{2} \beta]{k}$
\end{codeshow}

劃線命令使用\latexline{underline}和\latexline{overline},水平括號使用brace或者bracket代替line,例如\latexline{underbrace}:

\begin{codeshow}
$\overline{m+n}$ \\
$\underbrace{a_1+\cdots+a_n}_{n}$
$\overbrace{a_1+\cdots+a_n}^{n}$
% 可選參數:線寬;豎直空距
$\underbracket[0.4pt][1ex]
  {a_1+\cdots+a_n}_n$
\end{codeshow}

兩個互有重疊的括號需要一個箱子命令\latexline{rlap},會在後面提到。不過在$j$之前的空距有些異常,可能需要\latexline{,}進行修正。
\begin{codeshow}
\[b+\rlap{$\overbrace{\phantom{
  c+d+e+f+g}}^x$}c+d+\underbrace{
  e+f+g+h+i}_y+\,j \]
\end{codeshow}

事實上\latexline{overline}命令也存在問題,請比較:

\begin{codeshow}
$\overline{A}\overline{B}$ \\
$\closure{A}\closure{B}$
$\closure{AB}$
\end{codeshow}

其中\latexline{closure}是在導言區定義的:
\begin{latex}
\newcommand{\closure}[2][3]{{}\mkern#1mu
    \overline{\mkern-#1mu#2}}
\end{latex}

還可以輸出能堆疊到其他對象上的箭頭符,比如向量符號:

\begin{codeshow}
  $\vec a\quad \overrightarrow{PQ}$
  $\overleftarrow{EF}$
\end{codeshow}

你也許還需要能夠添加上下堆疊的箭頭符:

\begin{codeshow}
\[ a\xleftarrow{x+y+z} b \]
\[ c\xrightarrow[x<y]{a*b*c}d \]
\end{codeshow}

尖帽符號、波浪符號,還有\pkg{yhmath}宏包支持的圓弧符號:

\begin{codeshow}
$\hat{A}\quad\widehat{AB}$\\
$\tilde{C}\quad\widetilde{CD}
\qquad\wideparen{APB}$
\end{codeshow}

強制堆疊命\latexline{stackrel},位於上方的符號與上標同等大小。如果有\pkg{amsmath}宏包,可以使用\latexline{overset}或者\latexline{underset} 命令,前者與\latexline{stackrel}命令完全等同:

\begin{codeshow}
$\int f(x) \stackrel{?}{=} 1$\\
$A\overset{abc}{=}B$ \quad $C\underset{def}{=}D$
\end{codeshow}

一個很強大的堆疊放置命令\latexline{sideset},只用於巨算符:

\begin{codeshow}
\[\sideset{_a^b}{_c^d}\sum\]
\[\sideset{}{'}\sum_{n=1}\text{或}
\,{\sum\limits_{n=1}}'\]
\end{codeshow}

去心鄰域\latexline{mathring}大概也屬於堆疊符的一種?這樣輸出:

\begin{codeshow}
$\mathring{U}$
\end{codeshow}

在下一次節:累加與累積中,還介紹了更多的堆疊命令。

\subsection{累加與累積}
使用\latexline{sum}和\latexline{prod}命令,效果如下:

\begin{codeshow}
\[\sum_{i=1}^{n}a_i=1 \qquad
\prod_{j=1}^{n}b_j=1\]
\end{codeshow}

有時需要複雜的堆疊方式,效果如下:

\begin{codeshow}
\[\sum_{\substack{0<i<n \\
  0<j<m}} p_{ij}=
  \prod_{\begin{subarray}{l}
  i\in I \\  1<j<m
  \end{subarray}}q_{ij}\]
\end{codeshow}

有時候需要強制實現堆疊的效果,可以使用\latexline{limits}命令。如果堆積目標不是數學對象,還需要使用\latexline{mathop}命令:

\begin{codeshow}
\[\max\limits_{i>1}^{x}\quad
\mathop{xyz}\limits_{x>0}\quad
\lim\nolimits_{x\to \infty}\]
\end{codeshow}

\subsection{矩陣與省略號}
最樸素的矩陣排版可以通過\envi{array}環境和自適應定界符完成:

\begin{codeshow}
\[\mathbf{A}=
\left(\begin{array}{ccc}
x_{11} & x_{12} & \ldots \\
x_{21} & x_{22} & \ldots \\
\vdots & \vdots & \ddots
\end{array}\right)\]
\end{codeshow}

還有就是\latexline{cdots}命令。\pkg{mathdots}宏包支持省略號縮放,並提供了\latexline{iddots}:$\iddots$. 或許什麼時候需要使用呢?

矩陣排版更多地使用\envi{matrix}環境,以圓括號矩陣\envi{pmatrix}最為常見:
\begin{codeshow}
\centering $\begin{matrix}
0 & 1 \\ 1 & 0 \end{matrix}\qquad
\begin{pmatrix} 0 & 2 \\
2 & 0 \end{pmatrix}$
\end{codeshow}

方括號和花括號使用\envi{[Bb]matrix}環境:
\begin{codeshow}
\centering $\begin{bmatrix}
0 & 3 \\ 3 & 0 \end{bmatrix}\qquad
\begin{Bmatrix} 0 & 4 \\
4 & 0 \end{Bmatrix}$
\end{codeshow}

行列式使用\envi{[Vv]matrix}環境:
\begin{codeshow}
\centering $\begin{vmatrix}
0 & 5 \\ 5 & 0 \end{vmatrix}\qquad
\begin{Vmatrix} 0 & 6 \\
6 & 0 \end{Vmatrix}$
\end{codeshow}

宏包\pkg{mathtools}的帶星\texttt{matrix}命令,可更改列對齊:
\begin{codeshow}
$\begin{pmatrix*}[r]
100 & -200 \\ 20 & 10
\end{pmatrix*}$
\end{codeshow}

在矩陣中排版\latexline{dfrac}分式時,處理行距如下例的 \texttt{\char92\char92 [8pt]}:
\begin{codeshow}
\[\mathbf{H}=\begin{bmatrix}
\dfrac{\partial^2 f}{\partial x^2} &
\dfrac{\partial^2 f}
{\partial x \partial y} \\[8pt]
\dfrac{\partial^2 f}
{\partial x \partial y} &
\dfrac{\partial^2 f}{\partial y^2}
\end{bmatrix}\]
\end{codeshow}

宏包\pkg{amsmath}還支持行內小矩陣\latexline{smallmatrix},需手動加括號。
\begin{codeshow}
矩陣 $\left(\begin{smallmatrix}
x & -y\\ y & x\end{smallmatrix}
\right)$ 可以顯示在行內。
\end{codeshow}

一種帶邊注的矩陣\latexline{bordermatrix},用法有些奇怪:
\begin{codeshow}
\[\bordermatrix{& 1 & 2\cr
1 & A & B \cr
2 & C & D \cr} \]
\end{codeshow}

最後,如果想排出更好看優雅的矩陣,可以參考宏包\pkg{nicematrix}。或者想用簡單的方法輸入一些特殊矩陣, 可以參考宏包\pkg{physics},這裏不過多介紹。

\subsection{分段函數與聯立方程}
用\envi{cases}環境書寫分段函數,它自動生成一個比\latexline{left\{}更緊湊的花括號:

\begin{codeshow}
\[y=\begin{cases}
\int x, & x>0 \\
0,   & x=0 \\
x-1, & x<0
\end{cases},\,x\in\mathbb{R}\]
\end{codeshow}

如果想要生成display樣式的內容(比如上面的積分號只是text樣式的),使用\pkg{mathtools}宏包的\envi{dcases}環境代替\envi{cases}環境。如果\envi{cases}環境的第二列條件不是數學語言而是一般文字,可以考慮使用\envi{dcases*}環境,列中用\&{}隔開。

\begin{codeshow}
\[y=\begin{dcases}
  \int x, & x>0 \\
  x^2, & x\leqslant 0
  \end{dcases}\]
\[z=\begin{dcases*}
  y, & when $y$ is prime\\
  y^2, & otherwise
  \end{dcases*}\]
\end{codeshow}

\subsection{多行公式及其編號}
\label{subsec:multieqnum}
多行公式可以使用\pkg{amsmath}下的\envi{align}環境——因為原生的\envi{eqnarray}環境真的很差!而且\envi{align}環境不需要像\envi{array}環境那樣給出列的數目和參數,能夠根據
\texttt{\&}符號的數量來自調整。\qd{這個環境會自動對齊等號或者不等號,所以必要時請用\&指定對齊位置}。下面是一個例子:

\begin{codeshow}
\begin{align}
  a^2  &= a\cdot a \\
       &= a*a      \\
       &= a^2
\end{align}
\end{codeshow}

\LaTeX\ 中長公式不能自動換行\footnote{不過\pkg{breqn}宏包的\envi{dmath}環境可以實現自動換行,讀者可以自行嘗試效果。},請如上自行指定斷行位置和縮進距離。

至於多行公式換頁,可以在導言區加上\latexline{allowdisplaybreaks}實現(可選參數:1為儘量避免換頁,2至4為傾向於換頁),或在特定位置加上\latexline{displaybreak}(可選參數:0為允許在下個換行符後換頁,但不傾向換頁;2-3介中;4為強制換頁)。兩種的默認可選參數都是4。

上例給出三個編號,如果你只需要一個,可以:

\begin{codeshow}
\begin{align}
  a^2&= a\cdot a& b&=c\nonumber\\
  g  &= a*a & d&>e>f  \nonumber\\
  step&= a^2 & &Z^3
\end{align}
\end{codeshow}

如果你想讓編號顯示在這三行的中間而不是最下面一行,可以嘗試把公式寫在\envi{aligned}或者\envi{gathered}環境中,然後再嵌套到\envi{equation}環境內:

\begin{codeshow}
\begin{equation}
  \begin{aligned}
    a^2  &= a\cdot a \\
         &= a*a      \\
         &= a^2
  \end{aligned}
\end{equation}
\end{codeshow}

如果你根本不想給多行公式編號,嘗試\envi{align*}環境。

另外,\pkg{amsmath}宏包的\envi{multline}環境將自動把編號放在末行。首行左對齊,末行右對齊,中間的行居中。
\begin{codeshow}
\begin{multline}
a>b \\
b>c \\
\therefore a>c
\end{multline}
\end{codeshow}

如果想在環境中插入小段行間文字,使用\latexline{intertext}命令,或者\pkg{mathtools}宏包的\latexline{shortintertext}命令。區別是後者的垂直間距更小一些。

\begin{codeshow}
\begin{align*}
\shortintertext{If}
 y &= 0 \\
 x &< 0\\
\shortintertext{then}
 z &= x+y
\end{align*}
\end{codeshow}

當然,\envi{align}環境用於分列對齊的。如果僅想所有行居中,使用\pkg{amsmath}宏包的\envi{gather}環境即可。這是一個非常實用的環境,你也可以用\envi{gather*}環境排版居中的、非編號的多行公式。

\begin{codeshow}
\begin{gather}
  X=1+2+\cdots+n \\
  Y=1
\end{gather}
\end{codeshow}

\subsection{二項式}
\label{subsec:binom}
二項式可能需要藉助\pkg{amsmath}宏包的\latexline{binom}命令。它也有像分式一樣的行間和行內兩個命令\latexline{tbinom}與\latexline{dbinom}:

\begin{codeshow}
$\mathrm{C}_n^k=\binom{n}{k}
\qquad a_n=\dbinom{n}{k}$
\end{codeshow}

你也可以通過該宏包支持的\latexline{genfrac}自定義類似二項式命令:
\begin{latex}
\genfrac{left-delim}{right-delim}{thickness}{mathstyle}
{numerator}{denominator}
% thickness為分式線線寬,留空空表示默認
% mathstyle從0-3由\displaystyle減至\scriptscriptstyle
\newcommand{\Bfrac}[2]{\genfrac{[}{]}{0pt}{}{#1}{#2}}
\end{latex}

你可以藉此得到新的命令\latexline{Bfrac}:
\begin{codeshow}
\[\text{We define } \Bfrac{n}{k} = \binom{k}{n}\]
\end{codeshow}

\subsection{定理}
在使用下述定理內容時,請加載\pkg{amsthm}宏包。

首先是定理環境格式的自定義。如同定義命令一樣,在導言區加上:
\begin{latex}
\newtheorem{envname}[counter]{text}[section]
\end{latex}

其中\textit{name}表示定理的引用名稱,即下文將其作為一個環境名來識別;\textit{text}表示定理的顯示名稱,即下文中定理將以其作為打印內容。而\textit{counter}參數表示你是否與先前聲明的某定理共同編號。\textit{section}參數表示定理的計數層級,如果是section,表示每節分別計數;chapter表示每章分別計數。

來看一個例子。首先在導言區定義如下三個樣式:
\begin{latex}
\theoremstyle{definition}\newtheorem{laws}{Law}[section]
\theoremstyle{plain}\newtheorem{ju}[laws]{Jury}
\theoremstyle{remark}\newtheorem*{marg}{Margaret}
\end{latex}

以上三個\latexline{theoremstyle}即是它預定義的所有樣式類型。definition標題粗體,內容羅馬體;plain標題粗體,內容斜體;remark標題斜體,內容羅馬體。帶星號表示不進行計數。在環境的使用中可以添加可選參數,用於以括號的形式註釋定理。然後這是示例:

\begin{codeshow}
\begin{laws}
Never believe easily.
\end{laws}
\begin{ju}[The 2nd]
Never suspect too much.
\end{ju}
\begin{marg}
Nothing else.
\end{marg}
\end{codeshow}

\pkg{amsthm}宏包還提供了\envi{proof}環境,並且用\latexline{qedhere}來指定證畢符號的位置。如果不加指定,將會自動另起一行。

\begin{codeshow}
\begin{proof}
For an right triangle, we have:
  \[a^2+b^2=c^2 \qedhere\]
\end{proof}
\end{codeshow}

\section{數學符號與字體}
\subsection{數學字體}
原生的數學字體命令:
\begin{center}
\begin{minipage}{\linewidth}
\centering
\tabcaption{原生數學字體表}
\label{tab:mathfont}
\begin{tabular}{>{\ttfamily\char92}l>{$}l<{$}}
\hline
mathrm\{ABCDabcde 1234\} & \mathrm{ABCDabcde 1234} \\
\hline
mathit\{ABCDabcde 1234\} & \mathit{ABCDabcde 123} \\
\hline
mathnormal\{ABCDabcde 1234\} & \mathnormal{ABCDabcde 1234} \\
\hline
mathcal\{ABCDabcde 1234\} & \mathcal{ABCDabcde 1234} \\
\hline
\end{tabular}
\end{minipage}
\end{center}

需要其他宏包支持的數學字體:
\begin{center}
\begin{minipage}{\linewidth}
\centering
\tabcaption{宏包數學字體表}
\label{tab:mathfont-pk}
\begin{tabular}{>{\ttfamily}ll}
\hline
\char92mathscr\{ABCDabcde 1234\} & mathrsfs\\
$\mathscr{ABCDabcde 1234}$ & \\
\hline
\char92mathfrak\{ABCDabcde 1234\} & amsfonts或者amssymb\\
$\mathfrak{ABCDabcde 1234}$ & \\
\hline
\char92mathbb\{ABCDabcde 1234\} & amsfonts或者amssymb\\
$\mathbb{ABCDabcde 1234}$ & \\
\hline
\end{tabular}
\end{minipage}
\end{center}

\subsection{定界符}
\label{subsec:delimiter}
\tref{tab:delimiter}給出了一些數學環境中使用的定界符。

\begin{table}[!htb]
\centering
\caption{定界符}
\label{tab:delimiter}
\begin{tabular}{@{}*{3}{>{$}p{2em}<{$} @{} >{\ttfamily}p{7em}}}
( & ( & [ & [ or \char92 lbrack & \uparrow & \char92 uparrow \\
) & ) & ] & ] or \char92 rbrack & \downarrow & \char92 downarrow \\
\{ & \{ or \char92 lbrace & \} & \} or \char92 rbrace & \updownarrow & \char92 updownarrow \\
\langle & \char92 langle & \rangle & \char92 rangle & \backslash & \char92 backslash \\
\lfloor & \char92 lfloor & \rfloor & \char92 rfloor & \Updownarrow & \char92 Updownarrow \\
\lceil & \char92 lceil & \rceil & \char92 rceil & \Uparrow & \char92 Uparrow \\
\Vert & \char92 | or \char92 Vert & | & | or \char92 vert & \Downarrow & \char92 Downarrow \\
\hline
\multicolumn{6}{c}{-- 以下需要amssymb宏包 --} \\
\multicolumn{3}{c}{$\ulcorner$ \quad \texttt{\char92 ulcorner}} & \multicolumn{3}{c}{$\urcorner$ \quad \texttt{\char92 urcorner}} \\
\multicolumn{3}{c}{$\llcorner$ \quad \texttt{\char92 llcorner}} & \multicolumn{3}{c}{$\lrcorner$ \quad \texttt{\char92 lrcorner}}
\end{tabular}
\end{table}

使用\latexline{left}, \latexline{right}還有\latexline{middle}能夠使定界符自適應式子的高度:
\begin{codeshow}
\[P\left(X \middle\vert Y=0\right)
=\left.\int_0^1 p(t)\ud t\middle/ N\right.\]
\end{codeshow}

如果希望手動指定定界符的尺寸,這時使用後:
\begin{codeshow}
% 加l, r, m對應上述三種自適應命令
$(\big(\Big(\bigg(\Bigg<\qquad
\bigl[\frac{x+y}{x^2}\bigr]$
\end{codeshow}

有時\latexline{left.}和\latexline{right.}能靈活地用於跨行控制,因為它們並非實際配對:
\begin{codeshow}
\begin{align*}
  x &=\left(\frac{1}{2}x\right.\\
  &\left.\vphantom{\frac{1}{2}}
  +y^2+z_1\right)
\end{align*}
\end{codeshow}

其中\latexline{vphantom}命令用於輸出一個高度虛位,使得第二行的自適應定界符與第一行同等大小。特別地,命令\latexline{mathstrut}表示一個有圓括號總高的虛位:
\begin{codeshow}
$\sqrt{b}\sqrt{y}\qquad
\sqrt{\mathstrut b}\sqrt{\mathstrut y}$
\end{codeshow}

\subsection{希臘字母}
希臘字母表如\tref{tab:greekletter}所示。表中包含了小寫希臘字母、大寫希臘字母,其中部分希臘字母的輸入方式與英文字母一致。
\begin{table}[!htb]
\centering
\caption{希臘字母表}
\label{tab:greekletter}
\renewcommand\arraystretch{1}
\begin{tabular}{*{4}{>{$}c<{$}!{}>{\ttfamily\char`\\}p{6em} @{}}}
\alpha & alpha & \theta & theta & o & \multicolumn{1}{p{6em}}{o} & \upsilon & upsilon \\
\beta & beta & \vartheta & vartheta & \pi & pi & \phi & phi \\
\gamma & gamma & \iota & iota & \varpi & varpi & \varphi & varphi \\
\delta & delta & \kappa & kappa & \rho & rho & \chi & chi \\
\epsilon & epsilon & \lambda & lambda & \varrho & varrho & \psi & psi \\
\varepsilon & varepsilon & \mu & mu & \sigma & sigma & \omega & omega \\
\zeta & zeta & \nu & nu & \varsigma & varsigma & \eta & eta \\
\xi & xi & \tau & tau & \multicolumn{4}{c}{} \\
A & \multicolumn{1}{p{6em}}{A} & B & \multicolumn{1}{p{6em}}{B} & \Gamma & Gamma & \varGamma & varGamma \\
\Delta & Delta & \varDelta & varDelta & E & \multicolumn{1}{p{6em}}{E} & Z & \multicolumn{1}{p{6em}}{Z} \\
H & \multicolumn{1}{p{6em}}{H} & \Theta & Theta & \varTheta & varTheta & I & \multicolumn{1}{p{6em}}{I} \\
\Lambda & Lambda & \varLambda & varLambda & M & \multicolumn{1}{p{6em}}{M} & N & \multicolumn{1}{p{6em}}{N} \\
\Xi & Xi & \varXi & varXi & O & \multicolumn{1}{p{6em}}{O} & \Pi & Pi \\
\varPi & varPi & P & \multicolumn{1}{p{6em}}{P} & \Sigma & Sigma & \varSigma & varSigma \\
T & \multicolumn{1}{p{6em}}{T} & \Upsilon & Upsilon & \varUpsilon & varUpsilon & \Phi & Phi \\
\varPhi & varPhi & X & \multicolumn{1}{p{6em}}{X} & \Psi & Psi & \varPsi & varPsi \\
\Omega & Omega & \varOmega & varOmega & \multicolumn{4}{c}{}
\end{tabular}
\end{table}

\subsection{二元運算符}
二元運算符包括常見的加減乘除,還有集合的交、並、補等運算。\tref{tab:operator}只列出常用的二元運算符,更多的請參考symbols-a4文檔。
\begin{table}[!htb]
\centering
\caption{二元運算符:\latexline{mathbin}}
\label{tab:operator}
\renewcommand\arraystretch{1}
\begin{tabular}{@{}*{2}{>{$}c<{$}!{} >{\ttfamily\char92}p{5em} @{}}*{2}{>{$}p{2em}<{$} @{} >{\ttfamily\char92}p{6em} @{}}}
+ & \multicolumn{1}{p{6em}}{+} & - & \multicolumn{1}{p{6em}}{-} & \times & times & \div & div \\
\pm & pm & \mp & mp & \circ & circ & \triangleright & triangleright \\
\cdot & cdot & \star & star & \ast & ast & \triangleleft & triangleleft \\
\cup & cup & \cap & cap & \setminus & setminus & \bullet & bullet \\
\oplus & oplus &\ominus & ominus & \otimes & otimes & \oslash & oslash \\
\odot & odot & \bigcirc & bigcirc & \vee & vee,lor & \wedge & wedge,land \\
\bigcup & bigcup & \bigcap & bigcap & \bigvee & bigvee & \bigwedge & bigwedge
\end{tabular}
\end{table}

\subsection{二元關係符}
二元關係符常常被用於判斷兩個數的大小關係,或者集合中的從屬關係。\tref{tab:relation-operator}和\tref{tab:amsrelation-operator}只列出常用的二元關係符,更多的請參考symbols-a4文檔。
\begin{table}[!htb]
\centering
\caption{二元關係符:\latexline{mathrel}}
\label{tab:relation-operator}
\renewcommand\arraystretch{1}
\begin{tabular}{@{}*{4}{>{$}c<{$}!{} >{\ttfamily\char92}p{6em} @{}}}
< & \multicolumn{1}{p{6em}}{<} & > & \multicolumn{1}{p{6em}}{>} & \le & le(q) & \ge & ge(q) \\
\ll & ll & \gg & gg & \equiv & equiv & \neq & neq \\
\prec & prec & \succ & succ & \preceq & preceq & \succeq & succeq \\
\sim & sim & \simeq & simeq & \cong & cong & \approx & approx \\
\subset & subset & \supset & supset & \subseteq & subseteq & \supseteq & supseteq \\
\in & in & \ni & ni & \notin & notin & \propto & propto \\
\parallel & parallel & \perp & perp & \smile & smile & \frown & frown \\
\asymp & asymp & \bowtie & bowtie & \vdash & vdash & \dashv & dashv
\end{tabular}
\end{table}

\tref{tab:amsrelation-operator}中的二元關係符需要\pkg{amssymb}宏包。
\begin{table}[!htb]
\centering
\caption{amssymb二元關係符}
\label{tab:amsrelation-operator}
\renewcommand\arraystretch{1}
\begin{tabular}{@{}*{4}{>{$}c<{$}!{} >{\ttfamily\char92}p{6em} @{}}}
\leqslant & leqslant & \geqslant & geqslant & \because & because & \therefore & therefore \\
\nless & nless & \ngtr & ngtr & \lessdot & lessdot & \gtrdot & gtrdot \\
\lessgtr & lessgtr & \gtrless & gtrless & \lesseqqgtr & lesseqqgtr & \gtreqqless & gtreqqless \\
\subseteqq & subseteqq & \supseteqq & supseteqq & \subsetneqq & subsetneqq & \supsetneqq & supsetneqq
\end{tabular}
\end{table}

\subsection{箭頭與長等號}
在\tref{tab:delimiter}中給出了幾個箭頭符號,但是不夠全,這裏給出總表如\tref{tab:arrow}。
\begin{table}[!htb]
\centering
\caption{箭頭}
\label{tab:arrow}
\renewcommand\arraystretch{1}
\begin{tabular}{@{}*{2}{>{$}c<{$}!{} >{\ttfamily\char92}p{10em} @{}}}
\leftarrow & leftarrow & \longleftarrow & longleftarrow \\
\rightarrow & rightarrow & \longrightarrow & longrightarrow \\
\leftrightarrow & leftrightarrow & \longleftrightarrow & longleftrightarrow \\
\Leftarrow & Leftarrow & \Longleftarrow & Longleftarrow \\
\Rightarrow & Rightarrow & \Longrightarrow & Longrightarrow \\
\Leftrightarrow & Leftrightarrow & \Longleftrightarrow & Longleftrightarrow \\
\mapsto & mapsto & \longmapsto & longmapsto \\
\nearrow & nearrow & \searrow & searrow \\
\swarrow & swarrow & \nwarrow & nwarrow \\
\leftharpoonup & leftharpoonup & \rightharpoonup & rightharpoonup \\
\leftharpoondown & leftharpoondown & \rightharpoondown & rightharpoondown \\
\rightleftharpoons & rightleftharpoons & \iff & iff (bigger space)
\end{tabular}
\end{table}

\LaTeX\ 定義了邏輯命令\latexline{iff}, \latexline{implies}, \latexline{impliedby},與箭頭符大小相同但是兩側間距更大:
\begin{codeshow}
$x=y \implies a=b$\\
$x=y \impliedby a=b$\\
$x=y \iff a=b$
\end{codeshow}

依舊另外給出一個基於\pkg{amssymb}宏包的附\tref{tab:amsarrow}。
\begin{table}[!htb]
\centering
\caption{amssymb箭頭}
\label{tab:amsarrow}
\renewcommand\arraystretch{1}
\begin{tabular}{@{}*{2}{>{$}c<{$}!{} >{\ttfamily\char92}p{10em} @{}}}
\dashleftarrow & dashleftarrow & \dashrightarrow & dashrightarrow \\
\circlearrowleft & circlearrowleft & \circlearrowright & circlearrowright \\
\leftrightarrows & leftrightarrows & \rightleftarrows & leftrightarrows \\
\nleftarrow & nleftarrow & \nLeftarrow & nLeftarrow \\
\nrightarrow & nrightarrow & \nRightarrow & nRightarrow \\
\nleftrightarrow & nleftrightarrow & \nLeftrightarrow & nLeftrightarrow
\end{tabular}
\end{table}

最後,宏包\pkg{extarrows}給出了一些實用的長箭頭與長等符號:

\begin{codeshow}
$\xlongequal{\Delta}$\quad
$\xLeftrightarrow{\Delta}$\\
$\xleftrightarrow{x=\tan t}$\\
$\xLongleftarrow{x} \xLongrightarrow{y}$
\end{codeshow}

\subsection{其他符號}
注意冒號如果從鍵盤輸入,會識別為關係符,例如$:=$。在表示比例時也可以借用,或者外加\latexline{mathbin}命令$a\mathbin{:}b$。數學中可能用到的冒號,請使用\latexline{colon}命令,像$x\colon y\to\infty$這樣。

類似西文斷詞的\latexline{-}命令,在數學環境中使用\latexline{*}命令可以提醒\LaTeX\ 斷詞。\LaTeX\ 如果在此處斷詞,會自動補一個$\times$叉乘號。你也可以自定義來讓\LaTeX\ 補點乘號:
\begin{latex}
\renewcommand{\*}{\discretionary{\,\mbox{$\cdot$}}{}{}}
\end{latex}

最後是一些其他的難以歸類的符號,也不全是數學領域會用到的,只不過它們可以在數學環境下輸出出來,以及被\pkg{amssymb}宏包所支持。如\tref{tab:othersym}和\tref{tab:amsothersym}。
\begin{table}[!htb]
\centering
\caption{其他符號}
\label{tab:othersym}
\renewcommand\arraystretch{1}
\begin{tabular}{@{}*{4}{>{$}c<{$}!{} >{\ttfamily\char92}p{5.5em} @{}}}
\dots & dots & \cdots &cdots &
\vdots & vdots & \ddots & ddots \\
\forall & forall & \exists & exists &
\Re & Re & \aleph & aleph \\
\angle & angle & \infty & infty &
\triangle & triangle & \nabla & nabla \\
\hbar & hbar & \imath & imath &
\jmath & jmath & \ell & ell \\
\spadesuit & spadesuit & \heartsuit & heartsuit &
\clubsuit & clubsuit & \diamondsuit & diamondsuit \\
\flat & flat & \natural & natural &
\sharp & sharp & \multicolumn{2}{l}{} \\
\hline
\multicolumn{8}{l}{非數學符號:} \\
\multicolumn{1}{p{2em}}{\pounds} & pounds & \multicolumn{1}{p{2em}}{\S} & S &
\multicolumn{1}{p{2em}}{\copyright} & copyright & \multicolumn{1}{p{2em}}{\P} & P \\
\multicolumn{1}{p{2em}}{\dag} & dag & \multicolumn{1}{p{2em}}{\ddag} & ddag &
\multicolumn{1}{p{2em}}{\textregistered} & \multicolumn{3}{l}{\ttfamily \char`\\ textregistered}
\end{tabular}
\end{table}

\begin{table}[!htb]
\centering
\caption{amssymb其他符號}
\label{tab:amsothersym}
\renewcommand\arraystretch{1}
\begin{tabular}{@{} >{$}c<{$}!{} >{\ttfamily\char92}p{6em} @{}*{2}{>{$}p{2em}<{$} @{} >{\ttfamily\char92}p{8em} @{}}}
\square & square & \blacksquare & blacksquare & \hslash & hslash \\
\bigstar & bigstar & \blacktriangle & blacktriangle & \blacktriangledown & blacktriangledown \\
\lozenge & lozenge & \blacklozenge & blacklozenge & \measuredangle & measuredangle \\
\mho & mho & \varnothing & varnothing & \eth & eth
\end{tabular}
\end{table}

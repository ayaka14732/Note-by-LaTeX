%!TEX root = ../LaTeX-cn.tex
\chapter{\mbox{寫給讀者}*}

我見過許多朋友初試\LaTeX\ ,他們都感到非常不能理解。主要有以下幾個疑問:
\begin{feae}
\item “我平常使用MS Word,似乎也能完成科技排版工作,那麼為什麼我還需要\LaTeX\ ?”\hfill \textit{——見“為什麼需要\LaTeX\ ?”一節。}
\item “\LaTeX\ 看上去不像是排版工具,更像是編程語言。我討厭用寫代碼一樣的方式來寫文章。”\hfill \textit{——文本文件使你更專注於內容而不是排版細節。}
\item “\LaTeX\ 能生成doc文件嗎?我平時上交作業/提交匯報時難道使用不便修改pdf文件麼?”\hfill \textit{——見“\LaTeX\ 生成的文件格式?”一節。}
\end{feae}

本章希望能解決讀者的這些疑問,讓讀者對於\LaTeX\ 有基礎的瞭解,再決定是否需要學習。當然,如果你是被迫進入了\LaTeX\ 這個坑,你也可以閲讀本章,或許本章能讓你喜歡上\LaTeX\ 呢。

\section{什麼是\LaTeX\ ?}
先講\TeX\ (讀音類似於“泰赫”)。

\TeX\ 是Knuth\footnote{Donald Ervin Knuth(高德納,1938--),現代計算機科學的先驅者,斯坦福大學計算機系的終身榮譽教授,圖靈獎和馮諾依曼獎得主,\TeX\ 和 \hologo{METAFONT} 的發明人。同時也是業內經久不衰的著作\emph{The Art of Computer Programming}的作者。}研發的免費、開源的排版系統,其初衷是為了“改變排版界糟糕的排版技術”,並用於排版他的著作《計算機程序設計藝術》。

\TeX\ 對於讀者來説應該是底層的內容,如果你有興趣,可以閲讀Knuth著的\emph{The \TeX\ book},這本書是學習\TeX\ 最權威的材料,沒有之一。在本手冊的參考文獻中也給出了其他\TeX\ 學習的資料。\dpar

再講\LaTeX\ (讀音類似於“拉泰赫”)。

\LaTeX\ 是基於\TeX\ 的宏\footnote{“宏(Macro)”是一個計算機概念,指用單個命令或操作完成一系列底層命令或操作的組合。}集,其作者是Dr.~Lamport\footnote{Leslie Lamport(1941--),美國計算機科學家,圖靈獎和馮諾依曼獎得主。};其姓氏開頭兩個字母La與底層排版系統\TeX\ 相結合,就組成了名稱\LaTeX\ 。\LaTeX\ 在\TeX\ 基礎上定義了眾多的宏命令,使得用户可以更方便地進行排版。本手冊的參考文獻中就有他的作品。

\LaTeX\ 現在的版本是\LaTeXe ,意思是2.x版而沒到3.0版。錯位排版的字母E和字母A暗示了它是排版系統。在無法這樣輸出的場合,請寫作LaTeX和LaTeX~2e。

\section{\TeX\ 與\LaTeX\ 的優缺點}
\TeX\ 的優點:\qd{穩定、精確、美觀}。底層的\TeX\ 系統已經很多年沒有進行大的變動了,因為它注重\uline{穩定};\TeX\ 系統可以讓你把排版內容通過數字參數的方式寫到任意的位置,量化的參數意味着\uline{精確};\TeX\ 底層的空距調整機制,以及對於數學公式近乎完美的支持,則確保了排版效果的\uline{美觀}。

\LaTeX\ 是基於\TeX\ 的,自然不會拋棄上述\TeX\ 的優點。具體包括:
\begin{feai}
\item 排版出來就是印刷品。專業而美觀。
\item 易用、全面的數學排版支持,無出其右。
\item 撰寫文檔時不會被文檔排版細節干擾精力。你可以使用之前自定義的模板,或者方便地在文字組織完畢後調整你的模板,以輕鬆達到滿意的效果。
\item 複雜的排版功能支持,比如圖表目錄、索引、參考文獻管理、高度自定義的目錄樣式、雙欄甚至多欄排版。
\item 豐富的功能以及易尋的幫助文檔。眾多的\LaTeX\ 宏包賦予了\LaTeX\ 強大的擴展功能,它們都自帶文檔供你學習。
\item 源文件是\RED{文本文件}\footnote{文本文件的另一個優點是易於進行版本控制,比如利用git. 你可以方便地比較你相比上次修改了什麼內容,也可以方便地恢復到之前某個時刻的版本。}。你可以在任何設備、任何文本編輯器中書寫文檔內容,無須擔心複製時格式的變化;最後粘貼到同一個tex文件中編譯即可。
\item 跨平台,免費,開源。
\end{feai}

那缺點呢?我認為主要有:
\begin{feai}
\item 入門門檻高。想要熟練地使用\LaTeX\ 並輕鬆地編寫有自己的風格的文檔,不是一兩天就能夠達到的。
\item 並非“所見即所得”,需要編譯才能看到效果。編譯查錯有時令人惱火。
\item 完善一個自己的模板可能需要很長的時間。儘管\LaTeX\ 原生定義的模板能夠滿足絕大多數場合的需要。
\item 排版長表格有些複雜。但作為補充,在表格內插入數學公式是非常簡單的。
\end{feai}

\section{為什麼需要\LaTeX\ ?}
你可能基於以下原因學習\LaTeX\ :
\begin{feae}
\item 你的投稿對象要求你使用\LaTeX\ 排版,而不是MS Word——這種情況對沒聽説過\LaTeX\ 的你來説,真是糟糕透了。
\item 你需要在多個設備上撰寫同一份文檔。但你發現把內容在多個文檔間複製粘貼時,格式總是會出現問題。
\item 你受夠了MS Word自帶的公式編輯器,或者你覺得購買的插件MathType的效果也不盡如人意。但你經常需要排版公式。
\item 你想參加某個科學競賽,比如MCM,然後你發現你的朋友用的一個叫\LaTeX\ 的東西似乎還不錯。
\item 你想出版一本書,或者投稿你的作品——結果他們告知你如果你使用\LaTeX\ 而不是MS Word攥寫原稿件,他們會更快地把作品印刷出來。
\item 呃……也許,你只是喜歡學習新事物?
\end{feae}

對於科研工作者或者在校研究生,我認為\LaTeX\ 是非常優秀的工具。如果你是本科生,或者更年輕的羣體,你也可以先學習\LaTeX\ ,因為到了研究生和工作中,學習這類基礎工具的時間可能就非常有限了。

\section{MS Word難道不優秀嗎?}
我想説的是,\qd{MS Word當然是優秀的軟件}。但是它與\LaTeX\ 的定位不同,所以它們分別適用於不同場合。前者注重簡單組織內容,後者注重排版效果。

在\uline{排版書籍、科學文檔}方面,\LaTeX\ 非常專業、美觀,公式支持性極佳,幾乎所有參數你都可以量化調整。如果你想\uline{高度自定義一份文件},比如擁有特殊幾何、顏色元素,且易於更改模板的簡歷,\LaTeX\ 可以完全獨當一面。在這些這方面,MS Word是無法匹敵\LaTeX\ 的。

但是如果你只是為了生成\uline{非正式的文檔},比如1--2頁的作業稿;或者只是一份\uline{易於別人修改的非科學稿件},比如一份需要同事修改的演講稿……那你無須使用\LaTeX\ 。這些方面,\LaTeX\ 無疑是比不上MS Word的。

\section{\LaTeX\ 生成的文件格式?}
一般廣為使用的是pdf,以及dvi的格式。\LaTeX\ 無法生成doc或者docx擴展名文件,因為那是屬於MS的商用格式,兩者的工作機理也完全不同。

所以很遺憾,如果你身處一個要求你“必須提交docx”的環境中,那麼\LaTeX\ 對你並不是一個好選擇。但我想指出的是,這是穩定、優秀的pdf格式沒有得到你身處環境認可的遺憾——pdf也可以方便地添加批註,並在不同設備的顯示上有更好的穩定性。
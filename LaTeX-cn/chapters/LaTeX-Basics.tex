%!TEX root = ../LaTeX-cn.tex
\chapter{\LaTeX{}基礎}
\section{第一份文稿}

編輯器的配置大概是需要講解一下的,畢竟對於初學者來説是很頭疼的事情。本手冊就以\TeX\ Studio為例進行配置。首先你應該安裝一個\TeX{} Live,它是完全免費的,網址:\url{http://tug.org/texlive/}.

雖然它體積較大,但是卻是最一勞永逸、最不需要花時間去配置的方法,同時它大概也是功能支持最強的\LaTeX\ 發行版。

打開\TeX\ Studio後,選擇選項$\rightarrow$ 設置\TeX\ Studio $\rightarrow$ 構建$\rightarrow$ 默認編譯器,選擇\xelatex{}. 這主要是基於中文文檔編譯的考慮,同時\xelatex 也能很好地編譯英文文檔。我建議始終使用它作為默認編譯器。\dpar

之後你可以在窗口輸入一篇小文檔,並保存為tex擴展名的文件進行測試:
\begin{latex}
\documentclass{ctexart}
\begin{document}
    Hello, world!
    你好,世界!
\end{document}
\end{latex}

點擊編譯按鈕生成,F7查看。生成的pdf在你的tex文件保存目錄中。具體各行的含義我們會在後文介紹。

\section{認識\LaTeX}
\subsection{命令與環境}
\LaTeX\ 中的\co{命令}通常是由一個反斜槓加上命令名稱,再加上花括號內的參數構成的(有的命令不帶參數,例如:\latexline{TeX})。
\begin{latex}
\documentclass{ctexart}
\end{latex}

如果有一些選項是備選的,那麼通常會在花括號前用方括號標出。比如:
\begin{latex}
\documentclass[a4paper]{ctexart}
\end{latex}

還有一種重要指令叫做\co{環境}。它被定義於控制命令\latexline{begin\{environment\}} 和\latexline{end\{environment\}}間的內容。比如:
\begin{latex}
\begin{document}
...內容...
\end{document}
\end{latex}

環境如果有備選參數,只需要寫在\latexline{begin[...]\{name\}}這裏就行。

注意:不帶花括號的命令後面如果想打印空格,請加上\RED{一對內部為空的花括號}再鍵入空格。否則空格會被忽略。例如:\verb+\LaTeX{} Studio+.

\subsection{保留字符}

\LaTeX\ 中有許多字符有着特殊的含義,在你生成文檔時不會直接打印。例如每個命令的第一個字符:反斜槓。單獨輸入一個反斜槓在你的行文中不會有任何幫助,甚至可能產生錯誤。\LaTeX\ 中的保留字符有:
\begin{center}
\texttt{\# \$ \% \^ \& \_ \{ \} \char92}
\end{center}

它們的作用分別是:
\begin{para}
\item[\#{}:] 自定義命令時,用於標明參數序號。
\item[\${}:] 數學環境命令符。
\item[\%{}:] 註釋符。在其後的該行命令都會視為註釋。如果在回車前輸入這個命令,可以防止行末\LaTeX\ 插入一些奇怪的空白符。
\item[\^{}:] 數學環境中的上標命令符。
\item[\&{}:] 表格環境中的跳列符。
\item[\_{}:] 數學環境中的下標命令符。
\item[\{與\}:] 花括號用於標記命令的必選參數,或者標記某一部分命令成為一個整體。
\item[\char92{}:] 反斜槓用於開始各種\LaTeX\ 命令。
\end{para}

以上除了反斜槓外,均能用前加反斜槓的形式輸出。即你只需要鍵入:
\begin{center}
\verb|\# \$ \% \^ \& \_ \{ \}|
\end{center}

唯獨反斜槓的輸出比較頭痛,你可以嘗試:
\begin{codeshow}
$\backslash$ \textbackslash
\texttt{\char92}
\end{codeshow}

其中命令\latexline{char[num]}是一個特殊的命令,使用環境需要是tt字體環境,用於輸出ASCII碼對應的字符;92對應的即反斜槓。你也可以用\latexline{char`}後加字符的方式輸出你想輸出的命令,但需要包裹在\latexline{texttt}或者\latexline{ttfamily}內。如果想輸出的字符是保留字符,需要再加一個反斜槓。
\begin{verbatim}
\texttt{\char`~} % 輸出一個波浪線
\texttt{\char`\\} % 輸出保留字反斜槓
\texttt{\char`@} % 實際上可直接輸入@
\end{verbatim}

另外需要説明的是,上例提及的波浪線{\texttt{\~}}用來輸出一個禁止在該處斷行的空格,也不能夠直接輸出。嘗試:
\begin{codeshow}
a $\sim$ b
a\~ b
a\~{} b
a\textasciitilde b
\end{codeshow}

\subsection{導言區}
任何一份\LaTeX{}文檔都應當包含以下結構:
\begin{latex}
\documentclass[`\itshape options`]{doc-class}
\begin{document}
    ...
\end{document}
\end{latex}

其中,在語句\latexline{begin\{document\}}之前的內容稱為\co{導言區}。導言區可以留空,以可以進行一些文檔的準備操作。你可以粗淺地理解為:\RED{導言區即模板定義}。\dpar

文檔類的參數doc-class和可選選項{\textit{options}}有\tref{tab:documentclass}取值:
\begin{table}[!htb]
    \centering
	\caption{文檔類和選項}
	\label{tab:documentclass}
	\begin{tabular}{p{5em} @{\ -\ } p{24em}}
		\hline
		\multicolumn{2}{l}{\bfseries doc-class文檔類\footnotemark} \\
		\hline
		article   & 科學期刊,演示文稿,短報告,邀請函。\\
		proc      & 基於article的會議論文集。\\
		report    & 多章節的長報告、博士論文、短篇書。\\
		book      & 書籍。\\
		slides    & 幻燈片,使用了大號Scans Serif字體。\\
		\hline
		\multicolumn{2}{l}{\bfseries\itshape options} \\
		\hline
		字體     & 默認10pt,可選11pt和12pt.\\
		頁面方向 & 默認豎向portrait,可選橫向landscape。\\
		紙張尺寸 & 默認letterpaper,可選用a4paper, b5paper等。\\
		分欄     & 默認onecolumn,還有twocolumn。\\
		雙面打印 & 有oneside/twoside兩個選項,用於排版奇偶頁。article/report默認單面。\\
		章節分頁 & 有openright/openany兩個選項,決定是在奇數頁開啓新頁或是任意頁開啓新頁。注意article是沒有chapter(``章'')命令的,默認任意頁。\\
		公式對齊 & 默認居中,可改為左對齊fleqn;默認編號居右,可改為左對齊leqno。\\
		草稿選項 & 默認final,可改為draft,使行溢出的部分顯示為黑塊。\\
		\hline
	\end{tabular}
\end{table}

在本文中,多數的文檔類提及的均為report/book類。如果有article類將會特別指明。其餘的文檔類不予説明。本手冊排版即使用了report類。\dpar

在導言區最常見的是\co{宏包}的加載工作,命令形如:\latexline{usepackage\{package\}}。通俗地講,宏包是指一系列已經制作好的功能``模塊'',在你需要使用一些原生\LaTeX\ 不帶有的功能時,只需要調用這些宏包就可以了。比如本文的代碼就是利用\pkg{listings}宏包實現的。

宏包的具體使用將參在各部分內容説明中進行講解。如果你想學習一個宏包的使用,按Win+R組合鍵呼出運行對話框,輸入texdoc加上宏包名稱即可打開宏包幫助pdf文檔。例如:\verb|texdoc xeCJK|。

\footnotetext{此外還有\pkg{beamer}宏包定義的beamer文檔類,常用於創建幻燈片。}

\subsection{錯誤的排查}
\label{subsec:debug}
在編輯器界面上,下方的日誌是顯示編譯過程的地方。在你編譯通過後,會出現這樣的字樣:
\begin{feai}
	\item {\qd{Errors錯誤}}:嚴重的錯誤。一般地,編譯若通過了,該項是零。
	\item {\qd{Warnings警告}}:一些不影響生成文檔的瑕疵。
	\item {\qd{Bad Boxes壞箱}\footnote{Box是\LaTeX{}中的一個特殊概念,具體將在\hyperref[sec:box]{這裏}進行講解。}}:指排版中出現的長度問題,比如長度超出(Overfull)等。後面的Badness表示錯誤的嚴重程度,程度越高數值越大。這類問題需要檢查,排除Badness高的選項。
\end{feai}

你可以向上翻閲日誌記錄(即.log文件),來找到Warning開頭的記錄,或者Overfull/ Underfull開頭的記錄。這些記錄會指出你的問題出在哪一行(比如line 1-2) 或者在pdf的哪一頁(比如active [12]。注意,這個12表示計數器計數頁碼,而不是文件打印出來的真實頁數)。此外你還需要了解:
\begin{feai}
	\item 值得指出的是,由於\LaTeX{}的編譯原理(第一次生成aux文件,第二次再引用它),目錄想要合理顯示\qd{需要連續編譯兩次}。在連續編譯兩次後,你會發現一些Warnings會在第二次編譯後消失。在\TeX\ Studio中,你可以只單擊一次“構建並查看”,它會檢測到文章的變化並自動決定是否需要編譯兩次。
	\item 對於大型文檔,尋找行號十分痛苦。你需要學會合理地拆分tex文件,參閲\secref{sec:include}的內容。
\end{feai}

這裏也推薦宏包\pkg{syntonly},在導言區加入它支持的\latexline{syntaxonly}命令,會只排查語法錯誤而不生成任何文檔,這可以使你更快地編譯。不過它似乎不太穩定,例如本文檔可以正常編譯,但是使用該命令時則會出錯。

\subsection{文件輸出}
\LaTeX{}的輸出一般推薦pdf格式,由\LaTeX\ 直接生成dvi的方法並不推薦。

你在tex文檔的文件夾下可能看到的其他文件類型:
\begin{tabbing}
	.sty{\hspace{2em}}\=宏包文件\\
	.cls	\> 文檔類文件。\\
	.aux    \> 用於儲存交叉引用信息的文件。因此,在更新交叉引用(公式編號、\\
	\> 大綱級別)後,需要編譯兩次才能正常顯示。\\
	.log    \> 日誌。記錄上次編譯的信息。\\
	.toc    \> 目錄文件。\\
	.lof    \> 圖形目錄。\\
	.lot    \> 表格目錄。\\
	.idx    \> 如果文檔中包含索引,該文件用於儲存索引信息。\\
	.ind	\> 索引記錄文件。\\
	.ilg	\> 索引日誌文件。\\
	.bib	\> \bibtex 參考文獻數據文件。\\
	.bbl	\> \bibtex 生成的參考文獻記錄。\\
	.bst	\> \bibtex 模板。\\
	.blg	\> \bibtex 日誌。\\
	.out	\> \texttt{hyperref}宏包生成的pdf書籤記錄。
\end{tabbing}

有時\LaTeX\ 的編譯出現異常,你需要刪除文件夾下除了tex以外的文件再編譯。此外,在某些獨佔程序打開了以上的文件時(比如用Acrobat打開了pdf),編譯可能出現錯誤。請在編譯時確保關閉這些獨佔程序。

\section{標點與強調}
英文符號$|<>+=$一般用於數學環境中,如果在文本中使用,請在它們兩側加上“\$”。如果你在\LaTeX\ 中直接輸入大於、小於號而不把它們放在數學環境中,它們並不會被正確地打印。你應該使用\latexline{textgreater}, \latexline{textless}命令。

在部分科技文章中,中文的句號可能使用全角圓點“.”\footnote{這個標點是 U+FF0E,稱為 FULLWIDTH FULL STOP。},而不是平常的“。”,也不是正常的英文句點“.”。這個符號很難正常輸入;你可以先輸入正常句點,最後再替換。

\subsection{引號}
英文單引號並不使用兩個\verb|'|符號組合。左單引號是重音符\verb|`|(鍵盤上數字1左側),而右單引號是常用的引號符。英文中,\RED{左雙引號就是連續兩個重音符}。

英文下的引號嵌套需要藉助\latexline{thinspace}命令分隔,比如:
\begin{codeshow}[listing side text, listing options={escapeinside=++}]
``\thinspace`Max' is here.''
\end{codeshow}
% 這裏臨時修改了listing的escapeinside, 使得這一行代碼也可以被 codeshow 環境表示, 也使得行文風格更加統一
中文下的單引號和雙引號你可以用中文輸入法直接輸入。

\subsection{破折、省略號與短橫}
英文的短橫分為三種:
\begin{feai}
\item 連字符:輸入一個短橫:\verb|-|,效果如daughter-in-law
\item 數字起止符:輸入兩個短橫:\verb|--|,效果如:page 1--2
\item 破折號:輸入三個短橫:\verb|---|,效果如:Listen---I'm serious.
\end{feai}

中文的破折號你也許可以直接使用日常的輸入方式。中文的省略號同樣。但是注意,英文的省略號使用\latexline{ldots}這個命令而不是三個句點。

\subsection{強調:粗與斜}
\LaTeX\ 中專門有個叫做\latexline{emph\{text\}}的命令,可以強調文本。對於通常的西文文本,上述命令的作用就是斜體。如果你對一段已經這樣轉換為斜體的文本再使用這個命令,它就會取消斜體,而成為正體。

西文中一般採用上述的斜體強調方式而不是粗體,例如在説明書名的時候可能就會使用以上命令。關於字體的更多內容參考\hyperref[sec:font]{字體}這一節。

\subsection{下劃線與刪除線}
\LaTeX\ 原生提供的\latexline{underline}命令簡直爛的可以,建議你使用\pkg{ulem}宏包下的\texttt{uline}命令代替,它還支持換行文本。\pkg{ulem}宏包還提供了一些實用命令:

\begin{codeshow}
\uline{下劃線} \\
\uuline{雙下劃線} \\
\dashuline{虛下劃線} \\
\dotuline{點下劃線} \\
\uwave{波浪線} \\
\sout{刪除線} \\
\xout{斜刪除線}
\end{codeshow}

需要注意的是,\pkg{ulem}宏包重定義了\latexline{emph}命令,\RED{使得原來的加斜強調變成了下劃線、原來的兩次強調就取消強調變成了兩次強調就雙下劃線}。通過宏包的normalem選項可以取消這個更改:\verb|\usepackage[normalem]{ulem}|。

\subsection{其他}
角度符號或者温度符號需要藉助數學模式\verb|$...$|輸入:

\begin{codeshow}
$30\,^{\circ}$三角形 \\
$37\,^{\circ}\mathrm{C}$
\end{codeshow}

歐元符可能需要用到\pkg{textcomp}宏包支持的\latexline{texteuro}命令。

其次是千位分隔位,比如\verb|1\,000\,000|。如果你不想它在中間斷行就在外側再加上一個\latexline{mbox}命令:\latexline{mbox\{1\char`\\,000\char`\\,000\}}。也可以使用\pkg{numprint}或者\pkg{siunitx}宏包中的相關命令。

再次是注音符號,比如\^o,也常用於拼音聲調,參考\hyperref[app:phonetic]{注音符號表}部分的附錄內容。如果你想輸入音標,請使用tipa宏包\footnote{tipa會重定義\latexline{!}命令,因此請使用\latexline{negthinspace}代替;或在\pkg{xeCJK}與\pkg{amsmath}宏包前加載,並使用safe選項。具體可以參見本手冊的 Head.tex 文件。},同樣參考附錄A。

最後,介紹\pkg{hologo}宏包,它可以輸出許多\TeX\ 家族標誌。其實\LaTeX\ 原生自帶了\latexline{LaTeX}, \latexline{TeX}等命令。而hologo宏包支持的命令有:

\begin{codeshow}
% 大寫H表示符號的首字母也大寫
\hologo{XeLaTeX} \Hologo{BibTeX}
\end{codeshow}

\section{格式控制}
首先了解一下\hologo{LaTeX}的長度單位:
\begin{fead}
  \item[pt] point,磅。\label{sec:length}
  \item[pc] pica。1pc=12pt,四號字大小
  \item[in] inch,英寸。1in=72.27pt
  \item[bp] bigpoint,大點。1bp=$\frac{1}{72}$in
  \item[cm] centimeter,釐米。1cm=$\frac{1}{2.54}$in
  \item[mm] millimeter,毫米。1mm=$\frac{1}{10}$cm
  \item[sp] scaled point。\Hologo{TeX} 的基本長度單位,1sp=$\frac{1}{65536}$pt
  \item[em] 當前字號下,大寫字母M的寬度。
  \item[ex] 當前字號下,小寫字母x的高度。
\end{fead}

然後是幾個常用的長度宏,更多的長度宏使用會在表格、分欄等章節提到。
\begin{latex}
\textwidth % 頁面上文字的總寬度,即頁寬減去兩側邊距。
\linewidth % 當前行允許的行寬。
\end{latex}

有時候你可以使用可變長度,比如“\texttt{5pt plus 3pt minus 2pt}”,表示一個能收縮到3pt也能伸長到8pt的長度。直接使用倍數也是允許的,例如:1.5\latexline{parindent}等。

我們通常使用\latexline{hspace}和\latexline{vspace}這兩個命令控制特殊的空格,具體的使用方法參考\hyperref[sec:hvspace]{水平和豎直距離}這一節。

\subsection{空格、換行與分段}
在\LaTeX\ 中,多個空格會被視為一個,多個換行也會被視為一個。如果你想要禁止\LaTeX\ 在某個空格處的換行,將空格用\texttt{\char126}命令替代即可,比如“\texttt{Fig.\char126 8}”。

通常的換行方法非常簡單:\hologo{LaTeX}會自動轉行,然後在每一段的末尾,只需要輸入兩個回車即可完成分段。如果需要一個空白段落(實質是一個空白行),先輸入兩個回車,再輸入\latexline{mbox\{\}},最後再輸入兩個回車即可。你可以用\latexline{par}來產生一個帶縮進的新段。

在下劃線一節的例子中已經給出了強制換行的方式,即兩個反斜:\latexline{\char`\\}. 不過這樣做的缺點在於下一行段首縮進會消失,這個命令也的確一般\RED{不用於}正文換行;\textbf{正文中想要換行,請直接使用兩個回車}。

段落之間的距離由\latexline{parskip}控制,默認\texttt{0pt plus 1pt}. 
\begin{latex}
\setlength{\parskip}{0pt}
\end{latex}

宏包\pkg{lettrine}能夠產生首字下沉的效果:
\begin{codeshow}
\lettrine{T}{his} is an example. Hope you like this package, and enjoy your \LaTeX\ trip!
\end{codeshow}

\subsection{分頁}
用\latexline{newpage}命令開始新的一頁。

用\latexline{clearpage}命令清空浮動體隊列\footnote{參見\hyperref[sec:float]{浮動體}這一節的內容。},並開始新的一頁。

用\latexline{cleardoublepage}命令清空浮動體隊列,並在偶數頁上開始新的一頁。

注意:以上命令都是基於\latexline{vfill}的。如果要連續新開兩頁,請在中間加上一個空的箱子(\latexline{mbox\{\}}),如\latexline{newpage\char`\\mbox\{\}\char`\\newpage}。

\subsection{縮進、對齊與行距}
英文的段首不需要縮進。但是對中文而言,段首縮進需要藉助\pkg{indentfirst}宏包來完成。你可能還需要使用\latexline{setlength{\char`\\parindent}\{2em\}}這樣的命令來設置縮進距離。如果在句首強制取消縮進,你可以在段首使用\latexline{noindent}命令。

\LaTeX\ 默認使用兩端對齊的排版方式。你也可以使用\envi{flushleft}, \envi{flushright}, \envi{center}這三種環境來構造居左、居中、居右三種效果。特殊的\latexline{centering}命令常常用在環境內部(或者一對花括號內部),以實現居中的效果。但請儘量用\envi{center}環境代替這個老舊的命令。類似的命令還有\latexline{raggedleft}來實現居右,\latexline{raggedright}來實現居左。更多的空格控制請參考\hyperref[sec:hvspace]{這一節}。

插入製表位、懸掛縮進、行距等複雜的調整參考\hyperref[sec:hvspace]{這部分}的內容。

\section{字體與顏色}
\label{sec:font}
這一節只討論行文中字體使用。數學環境內字體使用請參考\hyperref[sec:mathfont]{這一節}的內容。

\subsection{字族、字系與字形}
字體(typeface)的概念非常令人惱火,在電子化時代,基本上也都以字體(font)作為替代的稱呼。宋體、黑體、楷體,這屬於\co{字族};對應到西文就是羅馬體、等寬體等。加粗、加斜屬於\co{字系和字形}。五號、小四屬於\co{字號}。這三者大概可以並稱\co{字體}\footnote{本文中的字族、字系等稱呼難以找到統一標準,可能並不是準確的名稱。}。

\subsection{中西文“斜體”}
首先需要明確一點:\RED{漢字沒有加斜體}。平常我們看到的加斜漢字,通常是幾何變換得到的結果,非常的粗糙,並不嚴格滿足排版要求;而真正的字形是需要精細的設計的。同時,漢字字體裏面也很少有加粗體的設計。

西文一般設有加斜,但是這與“斜體”並不是同一回事。加斜是指某種字族的Italy字系;而斜體,是指Slant字族。在行文中表強調時使用的是前者;在Microsoft Word等軟件中看到的傾斜的字母\textit{I},也代表前者。

\subsection{原生字體命令}
\LaTeX\ 提供了基本的字體命令,包括\tref{tab:fontcommand}中顯示的內容。
\begin{table}[!ht]
\centering
\caption{\LaTeX\ 字體命令表}
\label{tab:fontcommand}
\begin{tabular}{p{3em}<{\centering} @{\ -\quad} l @{\quad-\quad} p{18em}}
\hline
字族 & \latexline{rmfamily} & 把字體置為{\rmfamily Roman}羅馬字族。\\
     & \latexline{sffamily} & 把字體置為{\sffamily Sans Serif}無襯線字族。\\
     & \latexline{ttfamily} & 把字體置為{\ttfamily Typewriter}等寬字族。\\
\hline
字系 & \latexline{bfseries} & 粗體{\bfseries BoldSeries}字系屬性。\\
     & \latexline{mdseries} & 中粗體{\mdseries MiddleSeries}字系屬性。\\
\hline
字形 & \latexline{upshape}  & 豎直{\upshape Upright}字形。 \\
     & \latexline{slshape}  & 斜體{\slshape Slant}字形。 \\
     & \latexline{itshape}  & 強調體{\itshape Italic}字形。 \\
     & \latexline{scshape}  & 小號大寫體{\scshape Scap}字形。 \\
\hline
\multicolumn{3}{l}{\ttfamily 如果臨時改變字體,使用\latexline{textrm}, \latexline{textbf}這類命令。}\\
\hline
\end{tabular}
\end{table}

字族、字系、字形三種命令是互相獨立的,可以任意組合使用。但這種複合字體的效果有時候無法達到(因為沒有對應的設計),比如\latexline{scshape}字形和\latexline{bfseries}字系。\LaTeX\ 會針對這種情況給出警告,但仍可以編譯,只是效果會不同於預期。

如果在文中多次使用某種字體變換,可以將其自定義成一個命令。這時請使用text系列的命令而不要使用family, series或shape系列的命令。否則需要多加一組花括號防止“泄露”。以下二者等價:
\begin{latex}
\newcommand{\concept}[1]{\textbf{#1}}
\newcommand{\concept}[1]{{\bfseries #1}}
\end{latex}

更多自定義命令的語法請參考\hyperref[sec:newcommand]{這一節}。

然後就是字號的命令。行文會有一個默認的“標準”字號,比如你在documentclass的選項中設置的12pt(如果你設置了的話)。\LaTeX\ 給出了一系列“相對字號命令”,列出如\tref{tab:fontsize}。此外,\pkg{ctex}宏包的\latexline{zihao}命令,參數$0$--$8$以及$-0$--$-8$表示初號到八號、小初到小八\footnote{日常使用的小四為12pt,五號為10.5pt。}。
\begin{table}[!ht]
\centering
\caption{相對字號命令表}
\label{tab:fontsize}
\begin{tabular}{|l|*{3}{l|}}
\hline
命令         & 10pt & 11pt & 12pt \\
\hline
\latexline{tiny}         & 5pt  & 6py  & 6pt  \\
\latexline{scriptsize}   & 7pt  & 8pt  & 8pt  \\
\latexline{footnotesize} & 8pt  & 9pt  & 10pt \\
\latexline{small}        & 9pt  & 10pt & 11pt \\
\latexline{normalsize}   & 10pt & 11pt & 12pt \\
\latexline{large}        & 12pt & 12pt & 14pt \\
\latexline{Large}        & 14pt & 14pt & 17pt \\
\latexline{LARGE}        & 17pt & 17pt & 20pt \\
\latexline{huge}         & 20pt & 20pt & 25pt \\
\latexline{Huge}         & 25pt & 25pt & 25pt \\
\hline
\end{tabular}
\end{table}

如果你想設置特殊的字號,使用:
\begin{latex}
\fontsize{font-size}{line-height}{\selectfont <text>}
\end{latex}

其中font-size填數字,單位pt;一般而言,line-height填\latexline{baselineskip}\footnote{這個命令的意義是行與行之間的基線間距(即行距),默認是1.2倍文字高。}。

默認全文的字體使用\latexline{rmfamily}族的字體。你可以通過重定義的方式更改它,使\latexline{rmfamily, \char`\\textrm}命令都指向新的字體。甚至把默認字體改為sf/tt字族。
\begin{latex}
\renewcommand{\rmdefault}{`\textit{font-name}`}
% 默認字體改為sf字族,也可用\ttdefault
\renewcommand{\familydefault}{\sfdefault}
\renewcommand{\sfdefault}{`\textit{font-name}`}
% 如果你排版CJK文檔,還需要更改CJK的默認字體
\renewcommand{\CJKfamilydefault}{\CJKsfdefault}
\end{latex}

\subsection{西文字體}
\LaTeX\ 預包含字體如\tref{tab:alphafont}(參考\url{http://www.tug.dk/FontCatalogue/}):
\begin{table}[!hbt]
\centering
\caption{部分\LaTeX\ 西文字體}
\label{tab:alphafont}
\begin{tabular}{>{\ttfamily}ll}
\hline
命令 & \texttt{字體名} \\
\hline
cmr & \myfont{cmr}{Computer Modern Roman} (默認) \\
lmr & \myfont{lmr}{Latin Modern Roman} \\
pbk & \myfont{pbk}{Bookman} \\
ppl & \myfont{ppl}{Palatino} \\
lmss & \myfont{lmss}{Latin Modern Roman Serif} \\
phv & \myfont{phv}{Helvetica} \\
lmtt & \myfont{lmtt}{Latin Modern} \\
\hline
\end{tabular}
\end{table}

以上可以這樣使用:
\begin{latex}
\newcommand{\myfont}[2]{{\fontfamily{#1}\selectfont #2}}
\renewcommand{\rmdefault}{ptm} % 可更改默認字體,同理可改sfdefault等
% 以上在導言區定義。在正文中:
Let's change font to \myfont{ppl}{Palatino}!
\end{latex}

在\xelatex 編譯下,一般使用\pkg{fontspec}宏包來選擇\textbf{本地安裝}的字體。注意:該宏包可能會明顯增加編譯所需的時間。
\begin{latex}
\usepackage{fontspec}
  \newfontfamily{\lucida}{Lucida Calligraphy}
  \lucida{This is Lucida Calligraphy}
\end{latex}

該宏包的\latexline{setmathrm/sf/tt}與\latexline{setboldmathrm}命令可以支持你更改數學環境中調用的字體。

另外,你也可以通過簡單地加載\pkg{txtfont}宏包,設置西文字體為Roman體,且同時會為你設置好數學字體。其他的簡單字體宏包還有\pkg{cmbright},提供的CM Bright與\TeX\ 默認字體Computer Modern協調的不錯;以及提供Palatino字體的\pkg{pxfonts}。另外的字體宏包在此不再介紹。

\subsection{中文支持與CJK字體}
中文方面,\pkg{ctex}宏包直接定義了新的中文文檔類ctexart, ctexrep與ctexbook,以及ctexbeamer幻燈文檔類。例如本手冊\texttt{Head.tex}中:
\begin{latex}
\documentclass[a4paper, zihao=-4, linespread=1]{ctexrep}
  \renewcommand{\CTEXthechapter}{\thechapter}
\end{latex}

以上設置字號為小四,行距因子為1(故行距為$1\times 1.2=1.2$倍,其中1.2是\LaTeX\ 默認的基線間距)。而a4paper選項繼承與原生文檔類report,可見ctex文檔類還是很好地保留了原生文檔類的特徵。\RED{值得注意的是,ctex文檔類會用\texttt{\char92 CTEX}開頭的計數器命令代替原有的},除非你使用\texttt{scheme=plain}來讓ctex文檔類\uline{僅支持中文而不做任何文檔細節更改}。具體的使用參考ctex宏包文檔。

\pkg{ctex}宏包支持以下字體命令:
\begin{center}
\tabcaption{\texttt{ctex}宏包支持的字體命令}
\begin{tabular}{*{4}{ll}}
宋體 & \latexline{songti} & 黑體 & \latexline{heiti} & 仿宋 & \latexline{fangsong} & 楷書 & \latexline{kaishu} \\
雅黑 & \latexline{yahei} & 隸書\textsuperscript{\dag} & \latexline{lishu} & 幼圓\textsuperscript{\dag} & \latexline{youyuan} &\multicolumn{2}{l}{} \\
\multicolumn{8}{l}{\quad\dag\ \textit{標註了此符號的字體不受ubuntu字庫支持}。}
\end{tabular}
\end{center}

再者參考\xelatex 編譯下的\pkg{xeCJK}宏包的使用。在使用\xelatex 時,如果你使用ctex文檔類,它會在底層調用\pkg{xeCJK}宏包,所以你無須再顯式地加載它。當然你也可以使用原生文檔類,然後逐一漢化參數內容。

\TeX\ Live 配合\xelatex 時, 調用字體非常慢。Windows下,把xelatex.exe與TeXStudio設為管理員運行,能大幅縮短編譯用時。另外,安裝新字體後,管理員命令行\texttt{fc-cache}能夠刷新字體緩存(很慢),有時也能改善用時\footnote{提供這兩種方式的網頁鏈接:\href{https://tex.stackexchange.com/questions/325278/xelatex-runs-slow-on-windows-machine}{StackExchange頁面}。}。

比如在導言區:
\begin{latex}
\usepackage[slantfont,boldfont]{xeCJK}
  \xeCJKsetup{CJKMath=true}
  \setCJKmainfont[BoldFont=Source Han Serif SC Bold]{SimHei}
% 這裏把SimHei直接寫成中文“黑體”也可以
% 也可以直接通過 otf 等字體名調用
\end{latex}

其中,加載xeCJK宏包時使用了slantfont和boldfont兩個選項,表示允許設置中文的斜體和粗體字形。在setCJKmainfont命令中,把SimSun(宋體)設置為了主要字體,SimHei(黑體)設置為主要字體的粗體字形,即textbf或者bfseries命令的變換結果。你也可以使用SlantFont來設置它的斜體字形。

除了setCJKmainfont,還有setCJKsansfont(對應\latexline{textsf}),setCJKmonofont(對應\latexline{texttt}),以及setCJKmathfont(對應數學環境下的CJK字體,但需要載xeCJKsetup中設置CJKMath=true)。

上面提到的xeCJKsetup有下列可以定製的參數,下劃線為默認值:
\begin{feai}
\item CJKspace=true/\uline{false}:是否保留行文中CJK文字間的空格,默認忽略空格。
\item CJKMath=true/\uline{false}:是否支持數學環境CJK字體。如果想在數學環境中直接輸入漢字,請開啓該選項;否則在數學環境內,需要將漢字寫在\latexline{textrm}、或者\pkg{amsmath}宏包支持的\latexline{text}命令中。
\item CheckSingle=true/\uline{false}:是否檢查CJK標點單獨佔用段落最後一行。此檢查在倒數二、三個字符為命令時可能失效。
\item LongPunct=\verb|{——……}|:設置CJK長標點集,默認的只有中文破折號和中文省略號。長標點不允許在內部產生斷行。你也可以用\texttt{+=}或者\texttt{-=}號來修改CJK長標點集。
\item MiddlePunct=\verb|{——·}|:設置CJK居中標點集,默認的只有中文破折號和中文間隔號(中文輸入狀態下按數字1左側的重音符號鍵)。居中標點保證標點兩端距前字和後字的距離等同,並禁止在其之前斷行。你同樣可以使用\texttt{+=/-=}進行修改。
\item AutoFakeBold=true/\uline{false}:是否啓用全局偽粗體。如果啓用,在setCJKmainfont等命令中,將用AutoFakeBold=2參數代替原有的BoldFont=SimHei這種參數。其中,數字2表示將原字體加粗2倍實現偽粗體。
\item AutoFakeSlant=true/\uline{false}:是否啓用全局偽斜體。仿上。
\end{feai}

如果預定義一種CJK字體,可以在導言區使用如下命令。比如這裏定義了宋體,後文中直接使用\latexline{songti}來調用SimSun字體:
\begin{latex}
% 參數:[family]\font-switch[features]{font-name}
\newCJKfontfamily[song]\songti{SimSun}
\end{latex}

如果要臨時使用一種CJK字體,使用\latexline{CJKfontspec}命令。其中的FakeSlant和FakeBold參數根據全局偽字體的啓用情況決定;如果未啓用則使用BoldFont、SlantFont參數指定具體的字體。
\begin{latex}
{\CJKfontspec[FakeSlant=0.2,FakeBold=3]{SimSun} text}
\end{latex}

對於Windows系統,想要獲知電腦上安裝的中文字體,使用CMD命令:
\begin{verbatim}
fc-list -f "%{family}\n" :lang=zh-cn >d:\list.txt
\end{verbatim}

然後到\verb|d:\list.txt|文件中查看中文字體列表。如果沒有找到想要使用的字體,請:

\begin{feae}
  \item 使用 \texttt{fc-cache} 命令刷新字體緩存後再嘗試。如果你安裝了較多字體,刷新可能較慢。
  \item 如果刷新緩存無效,考慮重新安裝對應的字體。注意在安裝字體時,通過右鍵點擊(而不是雙擊打開)字體文件,然後選擇“為所有用户安裝”。如果雙擊字體文件後安裝,可能會導致 \LaTeX\ 無法找到字體。
\end{feae}

\subsection{顏色}
使用\pkg{xcolor}宏包來方便地調用顏色。比如本文中代碼的藍色:
\begin{latex}
\usepackage{xcolor}
  \definecolor{keywordcolor}{RGB}{34,34,250}
% 指定顏色的text
{\color{`\textit{color-name}`}{text}}
\end{latex}

\pkg{xcolor}宏包預定義的顏色:
\begin{center}
\tabcaption{\texttt{xcolor}宏包預定義顏色}
\begin{tabular}{*{6}{l|}l}
\scol{black} & \scol{darkgray} & \scol{lime} & \scol{pink} & \scol{violet} & \scol{blue} & \scol{gray} \\
\scol{magenta} & \scol{purple} & \scol{white} & \scol{brown} & \scol{green} & \scol{olive} & \scol{red}\\
\scol{yellow} & \scol{cyan} & \scol{lightgray} & \scol{orange} & \multicolumn{3}{|l}{\scol{teal}}
\end{tabular}
\end{center}

還可以通過“調色”做出新的效果:

\begin{codeshow}
\textcolor{red!70}{百分之70紅色}\\
\textcolor{blue!50!black!20!white}
  {50藍20黑30白}
\end{codeshow}

還有一些方便的顏色命令,比如帶背景色的箱子,參考\secref{subsec:colorbox}。

\section{引用與註釋}
電子文檔的最大優越在於能夠使用超鏈接,跳轉標籤、目錄,甚至訪問外部網站。這些功能實現都需要“引用”。
\subsection{標籤和引用}
使用\latexline{label}命令插入標籤(在MS Word中稱為“題注”),然後在其他地方用\latexline{ref}或者\latexline{pageref}命令進行引用,分別引用標籤的序號、標籤所在頁的頁碼。
\begin{latex}
\label{section:this}
\ref{section:this}
\pageref{section:this}
\end{latex}

宏包\pkg{amsmath}提供了\latexline{eqref}命令,默認效果如(3.1),實質上是調用了原生的\latexline{ref}命令。

但是更常用的是\pkg{hyperref}宏包。由於它經常與其他宏包衝突,一般把它放在導言區的最後。比如本手冊:
\begin{latex}
\usepackage[colorlinks,bookmarksopen=true,
    bookmarksnumbered=true]{hyperref}
\end{latex}

宏包選項也可以以\latexline{hypersetup}的形式另起一行書寫,鍵值包括:
\begin{para}
\item[colorlinks] 默認false,即加上帶顏色的邊框,\footnote{這個邊框在打印時並不會打印出來。}而不是更改文字的顏色。默認linkcolor=red, anchorcolor=black, citecolor=green, urlcolor=magenta. 
\item[hidelinks] 無參數,取消鏈接的顏色和邊框。
\item[bookmarks] 默認true,用於生成書籤。
\item[bookmarksopen] 默認false,是否展開書籤。
\item[bookmarksopenlevel] 默認全部展開。設置為secnumdepth對應的值可以指定展開到這一級。比如對report指定2,就是展開到section為止。
\item[bookmarksnumbered] 默認false,書籤是否帶章節編號。
\item[unicode] 無參數,使用UTF-8編碼時可以指定的選項。
\item[pdftitle] pdf元數據:標題。
\item[pdfauthor] pdf元數據:作者。
\item[pdfsuject] pdf元數據:主題。
\item[pdfkeywords] pdf元數據:關鍵詞。
\item[pdfstartview] 默認值Fit,設置打開pdf時的顯示方式。Fit適合頁面,FitH適合寬度,FitV適合高度。
\end{para}

如果章節標題中帶有特殊內容無法正常顯示在pdf書籤中,這樣使用:

\begin{verbatim}
\section{質能公式\texorpdfstring{$E=mc^2$}{E=mc\textasciicircum 2}}
\end{verbatim}

在加載了\pkg{hyperref}宏包後,可以使用的命令有:
\begin{latex}
% 文檔內跳轉
\hyperref[`\textit{label-name}`]{`\textit{print-text}`}
\autoref{`\textit{label-name}`} % 自動識別label上方的命令
% 鏈接網站
\href{URL}{print-text}
\url{URL} %彩色可點擊
\nolinkurl{URL} % 黑色可點擊
\end{latex}

其中\latexline{autoref}命令會先檢查\latexline{label}引用的計數器,再在檢查其\texttt{autref}宏是否存在。比如圖表環境會檢查是否有\latexline{figureautorefname}這個宏,如果有則引用之;而正常的\latexline{ref}命令只會引用\latexline{figurename}。以下列出\pkg{hyperref}宏包支持的計數器宏(請自行插入):
\begin{table}[!hbt]
\centering
\caption{\texttt{autoref}命令支持的計數器宏}
\label{tab:autoref}
\begin{tabular}{|*{2}{>{\ttfamily\char`\\}lc|}}
\hline
\multicolumn{1}{|c}{命令} & 默認值 & \multicolumn{1}{c}{命令} & 默認值 \\
\hline
figurename & Figure & tablename & Table \\
partname & Part & appendixname & Appendix \\
equationname & Equation & Itemname & item \\
chaptername & chapter & sectionname & section \\
subsectionname & subsection & subsubsectionname & subsubsection \\
paragraphname & paragraph & Hfootnotename & footnote \\
AMSname & Equation & theoremname & Theorem \\
page & \multicolumn{3}{l|}{page. 但常使用\latexline{autopageref}命令代替。} \\
\hline 
\end{tabular}
\end{table}

比如,通過重定義\latexline{figureautorefname},就能用“圖3.1”的效果代替默認的“Figure 3.1”:
\begin{latex}
\renewcommand\figureautorefname{圖}
\end{latex}

另一個宏包\pkg{nameref}不滿足於只引用編號,提供了引用對象的標題內容的功能。使用\latexline{nameref}命令可以利用位於標題下方的標籤來引用標題內容。

關於頁碼引用,如果想要生成“第$\times$頁,共$\times$頁”的效果,可能需要藉助\pkg{lastpage}宏包。它提供的標籤LastPage可以保證在輸出頁面的最後(如果你自行添加標籤,可能還會有後續浮動體),因此可以:

\begin{codeshow}
This is page \thepage\ of \pageref{LastPage}
\end{codeshow}

\subsection{腳註、邊注與尾註}
\subsubsection{腳註}
腳註是一種簡單標註,使用方法是:
\begin{latex}
\footnote{This is a footnote.}
\end{latex}

在某些環境內(如表格),腳註無法正常使用,可以先用\latexline{footnotemark}依次插入位置,再在tabular/table環境外用\latexline{footnotetext}依次指明腳註的內容。

minipage環境是支持腳註,在其內部或正文內這樣可以寫表格腳註:

\begin{codeshow}
\begin{minipage}{\linewidth}
\begin{tabular}{l}
This is an exmaple\footnotemark. 
\end{tabular}
\footnotetext{You don't need more.}
\end{minipage}
\end{codeshow}

行文中切忌過多地使用腳註,它會分散讀者的注意力。默認情況下腳註按章編號。腳註相關的命令:
\begin{latex}
% 在大綱或者\caption命令中使用腳註,需要加\protect
\caption{Title\protect\footnote{This is footnote.}}
% 腳註之間的距離:\footnotesep
% 每頁腳註之上橫線:\footnoterule,默認值:
\renewcommand\footnoterule{\rule{0.4\columnwidth}{0.4pt}}
% 調整腳註到正文的間距,例如:
\setlength{\skip\footins}{0.5cm}
\end{latex}

更多的參考\pkg{footmisc}宏包,比如其選項\texttt{perpage}讓腳註編號每頁清零。

\subsubsection{邊注}
\LaTeX\ 的邊注命令\latexline{marginpar}不會進行編號。必選參數表示在頁右顯示邊注;可選參數表示如果邊注在偶數頁,則在頁左顯示。例如右邊這個音符:\marginpar{\twonotes}
\begin{latex}
這一行有邊注\marginpar[左側]{右側}
\end{latex}

如果想要改變邊注的位置,使用\latexline{reversemarginpar}命令。此外,有關邊注的長度命令\latexline{marginparwidth/sep/push}分別控制邊注的寬、邊注到正文的距離、邊注之間的最小距離。可以使用\pkg{geometry}宏包來設置前兩者,參考\secref{sec:geometry}。

\subsubsection{尾註}
尾註用於註釋較長、無法使用腳註的場合,需要\pkg{endnotes}宏包。

\subsection{援引環境}
援引環境有quote和quotation兩個。前者首行不縮進;後者首行縮進,且支持多段文字。

\begin{codeshow}
魯智深其師有偈言曰:
\begin{quote}
逢夏而擒,遇臘而執。
聽潮而圓,見信而寂。
\end{quote}
圓寂之後,其留頌曰:
\begin{quotation}
平生不修善果,只愛殺人放火。
忽地頓開金繩,這裏扯斷玉鎖。

咦!錢塘江上潮信來,今日方知我是我。
\end{quotation}
\end{codeshow}

另外一個詩歌援引環境叫verse,\label{envi:verse}是懸掛縮進的。一般很少用到。

\begin{codeshow}
Rabindranath Tagore wrote this in 
his \emph{The Gardener}: 
\begin{verse}
Constant thrusts from your eyes
keep my pain fresh for ever.
\end{verse}
\end{codeshow}

\subsection{摘要}
article和report文檔類支持摘要,在\latexline{maketitle}命令之後可以使用\envi{abstract}環境。在單欄模式下,其相當於一個帶標題的\envi{quotation}環境,而這個標題可以通過重定義\latexline{abstractname}更改;雙欄下則相當於\latexline{section*}命令定義的一節。

\subsection{參考文獻}
\label{subsec:cite}
參考文獻主要使用的命令是\latexline{cite},與\latexline{label}相似。通過\pkg{natbib}宏包的使用可以定製參考文獻標號在文中的顯示方式等格式,下面\pkg{natbib}宏包的選項含義為:數字編號、排序且壓縮、上標、外側方括號,總體像這樣:\textsuperscript{\ttfamily [1,3-5]}。\footnote{這裏的LaTeX代碼實際為:\latexline{textsuperscript\{\char`\\ttfamily [1,3-5]\}}}
\begin{latex}
\documentclass{ctexart}
% 如果是book類文檔,把\refname改成\bibname
\renewcommand{\refname}{參考文獻}
\usepackage[numbers,sort&compress,super,square]{natbib}
\begin{document}
This is a sample text.\cite{author1.year1,author2.year2}
This is the text following the reference.
% “99”表示以最多兩位數來編號參考文獻,用於對齊
\begin{thebibliography}{99}
    \addtolength{\itemsep}{-2ex} % 用於更改行距
    \bibitem{author1.year1}Au1. ArtName1[J]. JN1. Y1:1--2
    \bibitem{author2.year2}Au2. ArtName2[J]. JN2. Y2:1--2
\end{thebibliography}
\end{document}
\end{latex}

當然以上只是權宜之計的書寫方法。更詳盡的參考文獻使用(\bibtex\ 方法)在\hyperref[sec:bibtex]{\bibtex{}這一節}進行介紹。

如果想要將參考文獻章節正常編號,並加入到目錄中,可以使用 \pkg{tocbibind} 宏包\label{pkg:tocbibind}。注意,此時需要重命名\latexline{tocbibname}(而不是 \latexline{refname} 或 \latexline{bibname})來指定參考文獻章節的標題。例如:
\begin{latex}
\usepackage[nottoc,numbib]{tocbibind}
\renewcommand{\tocbibname}{References}
\end{latex}

該宏包對於將索引、目錄本身、圖表目錄編入目錄頁同樣有效。選項\texttt{nottoc}表示目錄本身不編入,\texttt{notlof/lot}表示圖/表目錄不編入,\texttt{notindex}表示索引不編入,\texttt{notbib}表示參考文獻不編入。而選項\texttt{numindex/bib}表示給索引/參考文獻章節正常編號。選項\texttt{none}表示禁用所有。

\section{正式排版:封面、大綱與目錄}

\subsection{封面}
封面的內容在導言區進行定義,一般寫在所有宏包、自定義命令之後。主要用到的如:
\begin{latex}
\title{Learning LaTeX}
\author{wklchris}
\date{text}
\end{latex}

然後在\envi{document}環境內第一行,寫上:\latexline{maketitle},就能產生一個簡易的封面。其中\latexline{title}和\latexline{author}是必須定義的,\latexline{date}如果省略會自動以編譯當天的日期為準,格式形如:January 1, 1970。 如果你不想顯示日期,可以寫\latexline{date\{\}}。

標題頁的腳註用\latexline{thanks}命令完成。

\subsection{大綱與章節}
\LaTeX\ 中,將文檔分為若干大綱級別。分別是:
\begin{para}
\item[\latexline{part}] 部分。這個大綱不會打斷chapter的編號。
\item[\latexline{chapter}] 章。基於article的文檔類不含該大綱級別。
\item[\latexline{section}] 節。
\item[\latexline{subsection}] 次節。默認report/book文檔類本級別及以下的大綱不進行編號,也不納入目錄。
\item[\latexline{subsubsection}] 小節。默認article文檔類本級別及以下的大綱不進行編號,也不納入目錄。
\item[\latexline{paragraph}] 段。極少使用。
\item[\latexline{subparagraph}] 次段。極少使用。
\end{para}

對應的命令例如:\latexline{section\{第一節\}}。

以上各級別在\LaTeX\ 內部以“深度”參數作為標識。第一級別part的深度是$-1$,以下級別深度分別是$0,1,\ldots$,類推。注意到由於article文檔類缺少chapter大綱,其part深度又是從$0$開始的,故section及以下的深度數值與book/report文檔類是一致的。\dpar

另外的一些使用技巧:
\begin{latex}
% 大綱編號到深度2,並納入目錄
\setcounter{tocdepth}{2}
% 星號命令:插入不編號大綱,也不納入目錄
\chapter*{序}
% 將一個帶星號的大綱插入目錄
\addcontentsline{toc}{chapter}{序}
% 可選參數用於在目錄中顯示短標題
\section[Short]{Loooooooong}
% 自定義章節標題名
\renewcommand{\chaptername}{CHAPTER}
\end{latex}

book文檔類還提供了以下的命令:
\begin{para}
\item[\latexline{frontmatter}] 前言。頁碼為小寫羅馬字母,其後的章節不編號,但生成頁眉頁腳和目錄項。
\item[\latexline{mainmatter}] 正文。頁碼為阿拉伯數字;其後的章節編號,頁眉頁腳和目錄項正常運作。
\item[\latexline{backmatter}] 後記。頁碼格式不變,繼續計數。章節不編號,但生成頁眉頁腳和目錄項。
\end{para}

關於附錄\latexline{appendix}部分的大綱級別問題不在此討論,請參考\hyperref[sec:appendix]{這一節}。在book文檔類中,附錄一般放在正文與後記之間;當然你也可以在非book文檔類中使用附錄。關於章節樣式自定義的問題,則請看\hyperref[sec:titlesec]{這裏}。

\subsection{目錄}
目錄在大綱的基礎上生成,使用命令\latexline{tableofcontents}即可插入目錄。目錄在加載了\pkg{hyperref}宏包後,可以實現點擊跳轉的功能。你可以通過重定義命令更改\latexline{contentsname},即“目錄”的標題名。
\begin{latex}
\renewcommand{\contensname}{目錄}
\end{latex}

你也可以插入圖表目錄,分別是\latexline{listoffigures}, \latexline{listoftables}。通過重定義\latexline{listfigurename}和\latexline{listtablename}可以更改圖表目錄的標題。如果要更改目錄的顯示的大綱級別深度,設置計數器:
\begin{latex}
\setcounter{tocdepth}{2} % 這是到subsection
\end{latex}

想要將目錄本身編入目錄項,使用 \pkg{tocbibind} 宏包,參考\pageref{pkg:tocbibind}。

目錄的高級自定義需要藉助\pkg{titletoc}宏包,參考\secref{sec:titletoc}。

\section{計數器與列表}

\subsection{計數器}
\LaTeX\ 中的自動編號都藉助於內部的計數器來完成。包括:
\begin{fead}
\item[章節] part, chapter, section, subsection, subsubsection, paragraph, subparagraph
\item[編號列表] enumi, enumii, enumiii, enumiv
\item[公式和圖表] equation, figure, table
\item[其他] page, footnote, mpfootnote\footnote{\latexline{mpfootnote}命令用於實現minipage環境的腳註。}
\end{fead}

\RED{用\texttt{\char92 the}接上計數器名稱的方式來調用計數器},比如\latexline{thechapter}。如果只是想輸出計數器的數值,可以指定數值的形式,如阿拉伯數字、大小寫英文字母,或大小寫羅馬數字。常用的命令包括:
\begin{latex}
\arabic{counter-name}
\Alph \alph \Roman \roman
% ctex文檔類還支持\chinese
\end{latex}

比如本文的附錄,對章和節的編號進行了重定義。注意:\textbf{章的計數器包含了節在內}。以下的命令寫在\envi{appendices}環境中(或者\latexline{appendix}命令後),因此對於此外的編號不產生影響;同理你也可以這樣對列表編號進行局部重定義。
\begin{latex}
\renewcommand{\thechapter}{\Alph{chapter}}
\renewcommand{\thesection}
    {\thechapter-\arabic{section}}
\renewcommand{\thefootnote}{[\arabic{footnote}]}
\end{latex}

計數器的命令:
\begin{latex}
% 父級計數器變化,則子級計數器重新開始計數
\newcounter{`\textit{counter-name}`}[`\textit{parent counter-name}`]
\setcounter{`\textit{counter-name}`}{`\textit{number}`}
\addtocounter{`\textit{counter-name}`}{`\textit{number}`}
% 計數器步進1,並歸零所有子級計數器
\stepcounter{`\textit{counter-name}`}
\end{latex}

\subsection{列表}
\LaTeX\ 支持的預定義列表有三種,分別是無序列表\envi{itemize}, 自動編號列表\envi{enumerate}, 還有描述列表\envi{description}.

\subsubsection{\textit{itemize}環境}
例子:

\begin{codeshow}
\begin{itemize}
  \item This is the 1st.
  \item[-] And this is the 2nd.
\end{itemize}
\end{codeshow}

每個\latexline{item}命令都生成一個新的列表項。通過方括號的可選參數,可以定義項目符號。默認的項目符號是圓點(\latexline{textbullet})。更多的方法參考\hyperref[sec:list]{這一節}。

\subsubsection{\textit{enumerate}環境}
例子:

\begin{codeshow}
\begin{enumerate}
  \item First
  \item[Foo] Second
  \item Third
\end{enumerate}
\end{codeshow}

方括號的使用會打斷編號,之後的編號順次推移。更多的方法參考\hyperref[sec:list]{這一節}。

\subsubsection{\textit{description}環境}
例子:

\begin{codeshow}
\begin{description}
  \item[LaTeX] Typesetting System.
  \item[wkl] A Man.
\end{description}
\end{codeshow}

默認的方括號內容會以加粗顯示。更多的方法參考\hyperref[sec:list]{這一節}。

\section{浮動體與圖表}
\label{sec:float}

\subsection{浮動體}
浮動體將圖或表與其標題定義為整體,然後動態排版,以解決圖、表卡在換頁處造成的過長的垂直空白的問題。但有時它也會打亂你的排版意圖,因此使用與否需要根據情況決定。

圖片的浮動體是\envi{figure}環境,而表格的浮動體是\envi{table}環境。一個典型的浮動體例子:
\begin{latex}
\begin{table}[!htb]
    \centering
    \caption{table-cap}
    \label{table-name}
    \begin{tabular}{...}
        ...
    \end{tabular}
\end{table}
\end{latex}

其中,浮動體環境的參數\verb|!htb|含義是:!表示忽略內部參數(比如內部參數對一頁中浮動體數量的限制);h、t、b分別表示插入此處、插入頁面頂部、插入頁面底部,故htb表示優先插入此處,再嘗試插入到某頁頂,最後嘗試插入到頁底。此外還有參數p,表示允許為浮動體單獨開一頁。\LaTeX\ 的默認參數是tbp. 請不要單獨使用htbp中的某個參數,以免造成不穩定。

\latexline{caption}命令給表格一個標題,寫在了表格內容(即\envi{tabular}環境)之前,表示標題會位於表格上方。對於圖片,一般將把此命令寫在圖片插入命令的下方。注意:\RED{label命令請放在caption下方,否則可能出現問題}。\dpar

浮動體的自調整屬性可能導致它“一直找不到合適的插入位置”,然後多個浮動體形成排隊(因為靠前的浮動體插入後,靠後的才能插入)。如果在生成的文檔中發現浮動體丟失的情況,請嘗試更改浮動參數、去掉部分浮動體,或者使用\latexline{clearpage}命令來清空浮動隊列,以正常開始隨後的內容。

如果希望浮動體不要跨過section,使用:
\begin{latex}
\usepackage[section]{placeins}
\end{latex}

其實質是重定義了\latexline{section}命令,在之前加上了\latexline{FloatBarrier}。你也可以自行在每個想要阻止浮動體跨過的位置添加。

\subsection{圖片}
圖片的插入使用\pkg{graphicx}宏包和\latexline{includegraphics}命令,例子:
\begin{latex}
\begin{center}
    \includegraphics[width=0.8\linewidth]{ThisPic}
\end{center}
\end{latex}

可選參數指定了圖片寬度為0.8倍該行文字寬。類似地可以指定height(圖片高),scale(圖片縮放倍數),angle(圖片逆時針旋轉角度),origin(圖片旋轉中心lrctbB\footnote{這六個字母分別代表左、右、中、頂、底,以及基線。})這樣的命令。前三個命令不建議同時使用。旋轉的圖片基線會變化,故一般用totalheight代替height. 

對於\texttt{Thispic}這個參數的寫法,\xelatex 支持pdf, eps, png, jpg圖片擴展名。你可以書寫帶擴展名的圖片名稱ThisPic.png,也可以不帶擴展名。如果不給出擴展名,將按上述四個擴展名的順序依次搜索文件。\dpar

\subsubsection{圖片子文件夾}
如果你不想把圖片放在\LaTeX\ 文檔主文件夾下,可以使用下面的命令加入新的圖片搜索文件夾:
\begin{latex}
\graphicspath{{c:/pics/}{./pic/}}
\end{latex}

用正斜槓代替Windows正常路徑中的反斜槓。你可以加入多組路徑,每組用花括號括起,並確保路徑以正斜槓結束。用\verb|./|指代主文件夾路徑,也可省略。\dpar

\subsubsection{含特殊字符的文件名*}
文件名中含有特殊字符,一般插圖命令中是不需要加入反斜槓進行處理的(也就是按原文件直接輸入)。但是,圖片標題的 \latexline{caption} 命令需要轉義。如果可以直接輸入,那麼需要配合反斜槓進行轉義;在不能顯式輸入的場合, \latexline{detokenize} 命令可以應對。一個例子:
\begin{latex}
%在導言區定義:
%\newcommand{\includegraphicswithcaption}[2][]{
%  \includegraphics[#1]{#2}
%  \caption{\detokenize{#2}}
%}
\includegraphicswithcaption[width=0.6\linewidth]{hello_world.png}
\end{latex}

\subsubsection{圖文混排}
圖文混排可參考\pkg{wrapfig}宏包,後文的\hyperref[sec:box]{箱子}一節即是例子。
\begin{latex}
% \usepackage{wrapfig}
\begin{wrapfigure}[linenum]{place}[overhang]{picwidth}
    \includegraphics ...
    \caption ...
\end{wrapfigure}
\end{latex}

各參數的含義
\begin{inlinee}
\item {\bfseries\itshape linenum:} (可選)圖片所佔行數,一般不指定
\item {\bfseries\itshape place:} 圖片在文字段中的位置——R, L, I, O分別代表右側、左側、近書脊、遠書脊
\item {\bfseries\itshape overhang:} (可選)允許圖片超出頁面文本區的寬度,默認是0pt。在該項可以使用\latexline{width}代替圖片的寬度,填入\latexline{width}將允許把圖片全部放入頁邊區域
\item {\bfseries\itshape picwidth:} 指定圖片的寬度,默認情形下圖片的高度會自動調整。
\end{inlinee}

\subsection{表格}
\LaTeX\ 原生的表格功能非常有限,甚至不支持單元格跨行和表格跨頁。但是這些可以通過宏包\pkg{longtable}, \pkg{supertabular}, \pkg{tabu}等宏包解決。跨行的問題只需要\pkg{multirow}宏包。下面是一個例子(沒有寫在浮動體中):

\begin{codeshow}
\begin{center}
  \begin{tabular}[c]{|l|c||p{3em}
    r@{-}} \hline\hline
    A & B & C & d\\D & E & F & g\\
    \cline{1-2}
    \multicolumn{2}{|c|}{G}&H&i\\
    \hline
  \end{tabular}
\end{center}
\end{codeshow}

各參數的説明如下:
\begin{feai}
\item 可選參數\textbf{對齊方式}:[t] 表示表格上端與所在行的網格線對齊。如果與它同一行的有文字的話,文字是與表格上端同高的。如果使用參數 [b],就是下端同高。[c] 是中央同高。t=top, b=buttom, c=center.
\item 必選參數\textbf{列格式}:用豎線符號“|”來表示豎直表線,連續兩個“|”表示雙豎直表線。最右邊留空了,表示沒有豎直表線。或者你可以使用“@\{\}”表示沒有豎直表線。你也可以用“@\{-\}”這樣的形式把豎直表線替換成“-”,具體效果不再展示。而此處的l、c、r分別表示從左往右一共三列,分別\textbf{左對齊、居中對齊、右對齊}文字。在使用l、c、r時,表格寬度會自動調整。你可以用"p{2em}"這樣的命\textbf{指定某一列的寬度},這時文字自動左對齊。注意:單元格中的文字默認向上水平表線對齊,即豎直居上。
\item 在tabular環境內部,命令\latexline{hline}來繪製水平表線。命令\latexline{cline\{i-j\}}用於繪製橫跨從i到j列的水平表線。兩個連續的\latexline{hline}命令可以畫雙線,但是雙線之間相交時可能存在問題。
\item 在tabular環境內部,命令"\texttt{\&}"用於把光標跳入該行下一列的單元格。每行的最後請使用兩個反斜槓命令跳入下一行。命令\latexline{hline}或\latexline{cline}不能算作一行,因此它們後面沒有附加換行命令。
\item 在tabular環境內部,跨列命令\latexline{multicolumn\{number\}\{format\}\{text\}}用於以format格式合併該行的number個單元格,並在合併後的單元格中寫入文本text。如果一行有了跨列命令,請注意相應地減少"\texttt{\&}"的數量。
\end{feai}

文章中出現了表格,幾乎就一定會加載\pkg{array}宏包。在\pkg{array}宏包支持下,\texttt{cols}參數除了l, c, r, p\{\}, @\{\}以外,還可以使用:
\begin{feai}
\item \texttt{m\{\}, b\{\}}: 指定寬度的豎直居中,居下的列。
\item \verb|>{decl}, <{decl}|: 前者用在lcrpmb參數之前,表示該列的每個單元格都以此decl命令開頭;後者用於結尾。比如\footnote{例中的\latexline{centering}命令後可加入\latexline{arraybackslash}以應對可能的表格換行命令異常。}:
\begin{latex}
\begin{tabular}{|>{\centering\ttfamily}p{5em}
    |>{$}c<{$}|}
...
\end{tabular}
\end{latex}
\item \verb|!{symbol}|:  使用新的豎直表線,類似於原生命令\texttt{@\{\}},不同在於\verb|!{}|命令可以在列間保持合理的空距,而\verb|@{}|會使兩列緊貼。
\end{feai}

你甚至可以自定義lcrpmb之外的列參數,但需要保證是單字母。比如定義某一列為數學環境:
\begin{latex}
\newcolumntype{T}{>{$}c<{$}}
\end{latex}

\subsubsection{\texttt{array, multirow}宏包}
來一個\pkg{array}宏包下的例子:
\begin{codeshow}
% 記得\usepackage{array}
\begin{tabular}{|>{\setlength
  \parindent{5mm}}m{1cm}|
  >{\large\bfseries}m{1.5cm}|
  >{$}c<{$}|}
  \hline A & 2 2 2 2 2 2 & C\\
  \hline 1 1 1 1 1 1  & 10 & \sin x \\ \hline
\end{tabular}
\end{codeshow}

然後一個跨行跨列的例子。\RED{如果同時跨行跨列,必須把multirow命令放在multicolumn內部}。用\latexline{multirow}和\latexline{multicolumn}作用於單獨的1行或1列,能臨時改變某單元格的對齊方式。如果用星號代替列樣式,表示自適應寬度。

\begin{codeshow}
% \usepackage{multirow}
\begin{center}
\begin{tabular}{|c|c|c|}
  \hline
  \multirow{2}{2cm}{A Text!}
    & ABC & DEF \\
  \cline{2-3} & abc & def \\
  \hline
  \multicolumn{2}{|c|}
    {\multirow{2}*{Nothing}} & XYZ \\
  \multicolumn{2}{|c|}{} & xyz \\
  \hline
\end{tabular}
\end{center}
\end{codeshow}

表格的第一個單詞是默認不斷行的,這在單元格很窄而第一個詞較長時會出現問題。可以通過下述方法解決:

\begin{codeshow}
% \usepackage{array}
\newcolumntype{P}[1]{>{#1
  \hspace{0pt}\arraybackslash}
  p{14mm}}
% \arraybackslash用於修復換行符
\begin{center}
\begin{tabular}{|P{\raggedleft}|}
  \hline Superconsciousness \\ \hline
\end{tabular}
\end{center}
\end{codeshow}

此外,\qd{表格還可以嵌套},以方便地“拆分單元格”。注意下例中如何確保嵌套單元格表線顯示正常:

\begin{codeshow}
\begin{tabular}{|c|l|c|}
\hline
a & bbb & c \\ \hline
a & \multicolumn{1}{@{}l@{}|}
{\begin{tabular}{c|c}
a & b \\ \hline
aa & bb \\
\end{tabular}}
& c \\ \hline
a & b & c \\ \hline
\end{tabular}
\end{codeshow}

用\latexline{firsthline}和\latexline{lasthline}能夠解決行內表格豎直方向對齊問題。

\subsubsection{\texttt{makecell}宏包}
宏包\pkg{makecell}提供了一種方便在單元格內換行的方式,並可以配合參數tblrc;帶星表示有更大的豎直空距。此外,命令\latexline{multirowcell}由\pkg{multirow}宏包與該宏包共同支持。命令\latexline{thead}則有更小的字號,通常用於表頭。
\begin{codeshow}
\begin{tabular}{|c|c|}
\hline
\thead{雙行\\表頭} & \thead{雙行\\表頭}\\
\hline
\multirowcell{2}{簡單\\粗暴} & \makecell[l]{ABCD\\EF} \\
\cline{2-2} & \makecell*{更大的豎直空距} \\
\hline
\end{tabular}
\end{codeshow}

該宏包還提供了\latexline{Xhline}和\latexline{Xcline}命令,可以指定橫線的線寬。例如模仿三線表\footnote{更正規的三線表繪製,參考後文的\pkg{booktabs}宏包。}:
\begin{codeshow}
\begin{tabular}{ccc}
\Xhline{2pt}
\multirow{2}*{X} & 
\multicolumn{2}{c}{Hey}\\
\Xcline{2-3}{0.4pt}
& Left & Right \\
\Xhline{1pt} 
a & A & B \\
b & C & D \\
\Xhline{2pt}
\end{tabular}
\end{codeshow}

\subsubsection{\texttt{diagbox}宏包}
該宏包提供了分割表頭的命令\latexline{diagbox}。雖然斜線表頭並不是規範的科技排版內容,但是在許多場合也可能用到。命令支持兩或三參數。
\begin{codeshow}
\begin{tabular}{c|cc}
\diagbox{左邊}{中間}{右邊} & A & B \\
\hline
1 & A1 & B1 \\
2 & A2 & B2 
\end{tabular}
\end{codeshow}

\subsubsection{其他}
關於表格的間距:
\begin{feai}
\item \latexline{tabcolsep}或者\latexline{arraycolsep}控制列與列之間的間距,取決於你使用\envi{tabular}還是\envi{array}環境。默認\texttt{6pt}。
\item 列格式@能夠去除列間的空距,比如“@\{\}”。而命令\latexline{extracolsep\{1pt\}}於@的參數中,那麼會將其右側的列間隔都增加\texttt{1pt}。
\item 表格內行距用\latexline{arraystretch}控制,默認為1。
\end{feai}

一些其他的使用技巧:
\begin{feae}
\item 輸入同格式的列:例如\envi{tabular}參數\verb+|*{7}{c|}r|+,相當於7個居中和1個居右。
\item 表格重音:原本的重音命令\verb|\`|, \verb|\'|與\verb|\=|,改為\verb|\a`|, \verb|\a'|與\verb|\a=|。
\item 控制整表寬度:\pkg{tabularx}宏包提供\latexline{begin\{tabular*\}\{width\}[pos]\{cols\}},比如你可以把width取值為\latexline{0.8\char`\\linewidth}之類。
\item 如想實現單元格內換行,使用\pkg{makecell}宏包支持的\latexline{makecell}命令。
\item 宏包\pkg{dcolumn}提供了新的列對齊方式D,並調用\pkg{array}宏包。故你可以利用後者支持的命令,這樣定義:
\begin{latex}
% 表示輸入小數點、顯示為小數點、支持小數點後2位
\newcolumntype{d}{D{.}{.}{2}}
% 使用 d{2} 這樣的參數進行控制 
\newcolumntype{d}[1]{D{.}{.}{#1}}
\end{latex}
注意:\begin{itemize}
\item 表頭請用\latexline{multicolumn{1}{c}}類似的語句進行處理。
\item 第三參數不能幫你截取、舍入,只用於預設列寬;小心超寬。
\item 第三參數可以是“-1”,表示小數點居中;可以形如“2.1”,表示在小數點左側預留2位寬、右側預留1位寬。
\end{itemize}
\end{feae}

\subsection{非浮動體圖表和並排圖表}
如果不使用浮動體,又想給圖、表添加標題,請在導言區加上:
\begin{latex}
\makeatletter
\newcommand\figcaption{\def\@captype{figure}\caption}
\newcommand\tabcaption{\def\@captype{table}\caption}
\makeatother
\end{latex}

這部分是底層的\TeX\ 代碼,在此就不多介紹了。在如上定義後,你可以在浮動體外使用\latexline{figcaption}和\latexline{tabcaption}命令。注意:為了防止標題和圖表不在一頁,可以用\envi{minipage}環境把它們包起來。

同樣的,如果排版並排圖片,請用\envi{minipage}把每個圖包起來,指定寬度,然後放在浮動體內。注意靈活運用\verb|\\[10ex]|這樣的命令來排版$2\times 2$的圖片。

如果需要給每個圖片定義小標題,參考\pkg{subfig}宏包的相關內容。這裏給一個簡單的例子:
\begin{latex}
\begin{figure}
\centering
\subfloat[...]{\label{sub-fig-1}
    \begin{minipage}
    \centering
    \includegraphics[width=...]{...}
    \end{minipage}}
\quad\subfloat[...]
\end{latex}

\section{頁面設置}
\label{sec:geometry}
\subsection{紙張、方向和邊距}
主要藉助\pkg{geometry}宏包。先看一張頁面構成,如\fref{fig:geo-paper}:
\begin{figure}
\centering
\input{./tikz/geometry-paper.tex}
\figcaption{頁面構成示意圖}
\label{fig:geo-paper}
\end{figure}

geometry宏包的具體的選項參數有:
\begin{para}
\item[paper=<papername>]: 其中紙張尺寸有[a0--a6, b0--b6, c0--c6]paper, ansi[a--e]paper, letterpaper, executivepaper, legalpaper.
\item[papersize=\{<width>,<height>\}]: 自定義尺寸。也可以單獨對paperwidth或者paperheigth賦值。
\item[landscape]: 切換到橫向紙張。默認的是portrait.
\end{para}

body部分分為兩個概念:一個是總文本區(total body),另一個是主文本區(body). 總文本區可以由主文本區加上頁眉(head)、頁腳(foot)、側頁邊(marginalpar)組成。默認的選項為includehead,表示總文本區包含頁眉。要包括其他內容,可使用:\texttt{includefoot, includeheadfoot, includemp, includeall},以及以上各個參數將include改為ignore後的參數。

總文本區在默認狀態下佔紙張總尺寸的0.7,由scale=0.7控制,你也可以分別用\texttt{hscale}和\texttt{vscale}指定寬和高的佔比。用具體的長度定義也是可以的,使用\texttt{(total)width}和\texttt{(total)height}定義總文本區尺寸,或者用\texttt{textwidth}和\texttt{textheight}定義主文本區的尺寸\footnote{當totalwidth和textwidth都定義時,優先採用後者的值。}。或者直接用\texttt{total=\{width,height\}, body=\{width,height\}}定義。甚至你可以用\texttt{lines=<num>}行數指定textheight。

頁邊的控制最為常用,分別用\texttt{left/inner, right/outer, top, bottom}來定義四向的頁邊。其中\texttt{inner, outer}參數只在文檔的twoside參數啓用時才有意義。你可以用\texttt{hmarginratio}來給定left(inner)與right(outer)頁邊寬的比例,默認是單頁1:1、雙頁2:3。top和bottom之間的比由\texttt{vmarginratio}給定。你也可以用\texttt{vcentering, hcentering, centering}來指定頁邊比例為1:1. 在文檔的左側(內側),可以指定裝訂線寬度\texttt{bindingoffset},使頁邊不會被侵入。

頁眉和頁腳是位於top和bottom頁邊之內的文檔元素。對於頁眉和頁腳的高度,分別使用\texttt{headheight/head, footskip/foot}參數指定。\texttt{hmargin, vmargin}來指定側兩和頂底的邊距。它們到主文本區的參數分別是\texttt{headsep, footnotesep, marginparsep}. 你可以用\texttt{nohead, nofoot, nomarginpar}參數來清除總文本區中的頁眉,頁腳和側頁邊。

對於在文檔類documentclass命令中能使用的參數,geometry有不少也能做。比如\texttt{twoside, onecolumn, twocolumn}。甚至還能用文檔類中不能用的\texttt{columnsep}(啓用多欄分隔線)。

最後,這是幾個小例子:
\begin{latex}
% 與Microsoft Word的默認樣式相同:
\usepackage[hmargin=1.25in,vmargin=1in]{geometry}
% 書籍中靠書脊一側的邊距較小:
\usepackage[inner=1in,outer=1.25in]{geometry}
\end{latex}

\subsection{頁眉和頁腳}
頁眉和頁腳的控制主要藉助fancyhdr宏包。\LaTeX\ 中的頁眉頁腳定義主要藉助了兩個命令,一個是\latexline{pagestyle},參數有:
\begin{para}
\item[empty] 無頁眉頁腳。
\item[plain] 無頁眉,頁腳只包含一個居中的頁碼。
\item[headings] 無頁眉,頁腳包含章/節名稱與頁碼。
\item[myheadings] 無頁眉,頁腳包含頁碼和用户定義的信息。
\end{para}

另一個命令是\latexline{pagenumbering},與計數器一樣,擁有\texttt{arabic, [Rr]oman, [Aa]lph}五種頁碼形式。

\pkg{fancyhdr}宏包給出了一個叫fancy的\latexline{pagestyle},將頁眉和頁腳分別分為左中右三個部分,分別叫\latexline{lhead}, \latexline{chead}, \latexline{rhead}, 以及類似的[lcr]foot. 頁眉頁腳處的橫線粗細也可以定義,默認頁眉為0.4pt、頁腳為0pt. 下面是一個例子:
\begin{latex}
\usepackage{fancyhdr}
\pagestyle{fancy}
    \lhead{}
    \chead{}
    \rhead{\bfseries wklchris}
    \lfoot{Leftfoot}
    \cfoot{\thepage}
    \rfoot{Rightfoot}
\renewcommand{\headrulewidth}{0.4pt}
\renewcommand{\footrulewidth}{0.4pt}
\end{latex}

加載這個宏包,更多地是為了解決雙頁(twoside)文檔的排版問題。對於雙頁文檔,\pkg{fancyhdr}宏包給出了一套新的指令:用E, O表示單數頁和雙數頁,L, C, R表示左中右,H, F表示頁眉和頁腳。其中H, F需要配合\latexline{fancyhf}命令使用。如果不使用H, F這兩個參數,也可以使用\latexline{fancyhead},  \latexline{fancyfoot}兩個命令代替。一個新的例子:
\begin{latex}
\fancyhead{} % 清空頁眉
    \fancyhead[RO,LE]{\bfseries wklchris}
\fancyfoot{} % 清空頁腳
    \fancyfoot[LE,RO]{Leftfoot}
    \fancyfoot[C]{\thepage}
    \fancyfoot[RE,LO]{Rightfoot}
\end{latex}

該宏包在定義雙頁文檔時,採用瞭如下的默認設置:
\begin{latex}
\fancyhead[LE,RO]{\slshape \rightmark}
\fancyhead[LO,RE]{\slshape \leftmark}
\fancyfoot[C]{\thepage}
\end{latex}

上例中的\latexline{rightmark}表示較低級別的信息,即當前頁所在的section,形式如“1.2 sectionname ”,對於article則是subsection;而\latexline{leftmark}表示較高級別的信息,即對應的chapter,對於article則是section. 命令\latexline{leftmark}包含了頁面上\latexline{markboth}\footnote{\latexline{markboth}是一個會被\latexline{chapter}等命令調用的命令,默認右參數是空。注意,帶星號的大綱不調用這一命令,你需要這樣書寫:\latexline{chapter*\{This\char`\\markboth\{This\}\{\}\}}。}下的最後一條命令的左參數,比如該頁上出現了section 1--2,那麼leftmark就是“Section 2”;命令\latexline{rightmark}則包含了頁面上的第一個\latexline{markboth}命令的右參數或者第一個\latexline{markright}命令的唯一參數,比如可能是“Subsection 1.2”。

這聽起來可能難以理解,但是\latexline{markboth}命令有兩個參數,分別對應顯示在文檔的左頁和右頁(但是默認右參數留空,用\latexline{markright}去指定右頁),故有左右之分;而\latexline{markright}命令只有一個參數。你可以試着再去理解一下雙頁文檔下的宏包的默認設置。利用這一點來重定義chaptermark(book/report), sectionmark, subsectionmark(article)命令,舉個例子:
\begin{latex}
% 這裏的參數#1是指輸入的section/chapter的標題
% 效果:“1.2. The section”
\renewcommand{\sectionmark}[1]{\markright{\thesection.\ #1}}
% 效果:“CHAPTER 2. The chapter”
\renewcommand{\chaptermark}[1]{\markboth{\MakeUppercase{%
    \chaptername}\ \thechapter.\ #1}{}}
\end{latex}

如果你對於默認的\latexline{pagestyle}不滿意,可以用\latexline{fancypagestyle}命令進行更改。例如更改plain頁面類型:
\begin{latex}
\fancypagestyle{plain}{
    \fancyhf{} % 清空頁眉頁腳
    \fancyhead[c]{\thesection}
    \fancyfoot{\thepage}}
\end{latex}

\section{抄錄與代碼環境}
抄錄是指將鍵盤輸入的字符(包括保留字符和空格)不經過\TeX\ 解釋,直接輸出到文檔。默認的字體參數是等寬字族(ttfamily)。用法是\latexline{verb(*)}命令或者\envi{verbatim(*)}環境,區別在於帶星號的會將空格以“\textvisiblespace”(\latexline{textvisiblespace})的形式標記出來。

注意,\latexline{verb}命令是一個特殊的命令,可以用一組花括號括住抄錄內容,也可以任意兩個同樣的符號(但不能是*)。比如:
\begin{latex}
\verb|fooo{}bar|
\verb+fooo{}bar+
\end{latex}

\latexline{verb(*)}以及\envi{verbatim(*)}環境很脆弱,不能隱式地用於自定義環境,也一般不能用作命令的參數。\pkg{verbatim}宏包提供了更多的抄錄支持,\pkg{fancyvrb}宏包提供了\latexline{SaveVerb}, \latexline{UseVerb}命令,以及便於實現居中的\envi{BVerbatim}環境(置於\envi{center}環境內即可),詳情讀者可自行查閲。

宏包\pkg{shortverb}支持以一對符號代替\latexline{verb}命令,比如豎線號:
\begin{latex}
% \usepackage{shortverb}
\MakeShortVerb|
Verbatim between this pair of verts: |#\?*^|
\end{latex}

代碼環境的輸出,比如本文中帶行號的代碼塊,參見\hyperref[sec:coding]{這一節}。

\begin{multicols}{2}[\section{分欄}]
這部分內容使用文檔類的\texttt{two\-column}可選參數就能實現。在\LaTeX\ 的雙欄模式下,\latexline{newpage}命令只能進行換欄操作,而\latexline{clearpage}命令才會換進行換頁操作。同時,文中隨時可以使用\latexline{twocolumn}或者\latexline{onecolumn}命令執行\RED{換頁、清空浮動隊列,並切換分欄模式}。在雙欄上方的跨欄內容,如摘要,可以寫在\latexline{twocolumn[\ldots]}可選參數中。

欄之間的間距由\latexline{columnsep}控制;欄寬為\latexline{columnwidth},但請不要手工修改這個值。它可以被用作參數傳遞給其他命令。欄之間的分隔線寬由長度\latexline{columnseprule}給出,默認值為0pt,一般需要可以將其設置為0.4pt。 

如果在同一頁內需要分欄與單欄並存,或者想要分成多欄,可以嘗試使用\pkg{multicol}宏包。它提供一個支持任意多欄、但是邊注和浮動體\footnote{帶星號的浮動體或許可以使用,如\envi{figure*},但參數\texttt{h}會失效。}無法使用的環境。比如本節:
\begin{latex}
\begin{multicols}{2}
  [\section{分欄}]
  ...
\end{multicols}
\end{latex}

同時,該宏包會對齊每一欄的下邊緣;在該環境下,使用\latexline{columnbreak}來強制切換到新的一欄。還需要指出的是,該宏包並不保證各欄之間每行的網格都是對齊的。如果你需要此功能,可以參考\pkg{grid}宏包。
\end{multicols}

\section{文檔拆分}
\label{sec:include}
文檔拆分只需要在主文件中使用\latexline{input\{filename.tex\}}或\latexline{include\{filename\}}命令,後者不寫擴展名默認為.tex。兩者區別在於\latexline{include}命令將會插入\latexline{clearpage}再讀取文件。

其中 filename 可以是相對路徑,如 \verb|./chapter1|,\verb|../tikz|;也可以是絕對路徑,如 \verb|D:/latex/chapter1|。但是注意,在使用\latexline{include\{filename\}}時,儘量不要選擇在該文檔父目錄中的文件。例如,在 \verb|dir/subdir/main.tex| 中使用\latexline{include\{dir/filename\}},會因安全問題無法在文件夾 \verb|dir/| 中生成輔助文件 \verb|filename.aux|,進而導致編譯失敗。在這時應該嘗試調整文章層級結構,如果有十分必要的理由這麼做的話,在編譯的時候可以使用
\begin{verbatim}
  openout_any=a xelatex(pdflatex) mainfilename
\end{verbatim}
臨時修改 \verb|openout_any| 的值,該選項可以在 \verb|texmf.cnf| 中查看它的説明,但是不推薦在 \verb|texmf.cnf| 中修改,這可能會導致系統安全問題, 而\latexline{input\{filename.tex\}}不受這一點的影響。更多説明可以看\href{https://tex.stackexchange.com/questions/2209/how-can-i-include-the-file-somedir-file-tex-in-the-file-somedir-subdir-another}{\TeX SX 上的回答}以及\href{http://www.texfaq.org/FAQ-includeother}{\TeX faq上的説明}。

拆分的優勢在於可以根據chapter(或其他)分為多個文件,省去了長文檔瀏覽時的一些不便。你也可以把整個導言區做成一個文件,然後在不同的\LaTeX\ 文檔中反覆使用,即充當模板的功能。你還可以把較長的tikz繪圖代碼寫到一個tex中,在需要時\latexline{input}即可。

在導言區定義\latexline{includeonly}加上filename,可以確保只引入列表中的文件。在被引入文件的最後加入\latexline{endinput}命令,其後的內容會被忽略。

一種較規範的拆分文件的文件頭,以本文的章節放在次級目錄中為例:
\begin{latex}
%!TEX root = ../LaTeX-cn.tex
\end{latex}

\section{西文排版及其他}
\subsection{連寫}
\LaTeX\ 排版以及正規排版中,如果你輸入ff, fl, fi, ffi等內容,它們默認會連寫。在字母中間插入空白的箱子以強制不連寫:f~\latexline{mbox\{\}}~l。

\subsection{斷詞}
行末的英文單詞太長,\LaTeX\ 就會以其音節斷詞。如果你想指定某些單詞的斷詞位置,使用如下命令斷詞。例子:
\begin{latex}
\hyphenation{Hy-phen-a-tion FORTRAN}
\end{latex}

這個例子允許Hyphenation, hyphenation在短橫處斷詞,同時\textbf{禁止}FORTRAN, Fortran, fortran斷詞。如果你在行文中加入\verb|\-|命令,則可以實現允許在對應位置斷詞的效果。比如:

\begin{codeshow}
I will show you this:
su\-per\-cal\-i\-frag\-i\-lis\-%
tic\-ex\-pi\-al\-i\-do\-cious
\end{codeshow}

如果你不想斷詞,比如電話號碼,巧妙利用\latexline{mbox}命令吧:
\begin{latex}
My telephone number is: \mbox{012 3456 7890}
\end{latex}

\subsection{硬空格與句末標點}
如果你想在某個不帶參數的命令後輸入空格,請接上一個空的花括號確保空格能夠正常輸出。例如:\latexline{這是\char`\\TeX\{\} Live. }

在\LaTeX\ 中還有一個命令“\texttt{\char92\textvisiblespace}”,用於產生一個硬空格(區別於軟空格\texttt{\char92space}),所以你也可以用\latexline{TeX\char`\\ \textvisiblespace Live}。

西文排版下,\LaTeX\ 會判斷一種\co{句末標點},即小寫字母后的“.”,“?”或者“!”三個英文標點。句末標點後如果鍵入空格,\LaTeX\ 會自動增加空格的距離。如果句子以大寫字母結尾,\LaTeX\ 會認為這是人名而不增加空格,這時候需要手動添加命令\latexline{@}:

\begin{codeshow}
OK. That's fine.\\
OK\@. That's fine.
\end{codeshow}

相反,有些並非句末標點的情況會被識別為句末標點,這時候需要在標點後插入一個\latexline{\textvisiblespace}或者\verb|~|來縮小間距;區別在於前者允許斷行,後者不允許。

\begin{codeshow}
Prof. Smith is a nice man.\\
Prof.~Smith is a nice man.
\end{codeshow}

在標點後使用\latexline{frenchspacing}命令,可以調整為極小的空距。這個命令在排版參考文獻列表時可能被使用。

在\xelatex 編譯模式下的中文字符,與西文或者符號之間會產生默認的空距\footnote{這個問題在ctex文檔類下似乎被已解決。}。如果你不想要這個空距,把中文放在\latexline{mbox}內即可,比如:

\begin{codeshow}
\mbox{例子}-1
\end{codeshow}

\subsection{特殊符號}
符號的總表可以參照symbols-a4文檔,運行texdoc symbols-a4即可調出。包括希臘字母在內的一些數學符號將會在下一章介紹。這裏給出基於\pkg{wasysym}宏包的一些常用符號:
\begin{center}
  \centering
  \tabcaption{wasysym宏包符號}
  \begin{tabular}{*{3}{c >{\ttfamily\char92}p{5.5em}}}
     \permil     & permil   & \male     & male  & \female       & female \\
     \checked    & checked  & \XBox     & XBox  & \CheckedBox   & CheckedBox \\
     \hexstar    & hexstar  & \phone    & phone & \twonotes     & twonotes
  \end{tabular}
\end{center}

%!TEX root = ../LaTeX-cn.tex
\chapter{\LaTeX\ 進階}

本章的內容多數與宏包的使用相關。記得使用texdoc命令查看宏包的使用手冊,這是學習宏包最好的手段,沒有之一。

\section{自定義命令與環境}
\label{sec:newcommand}
自定義命令是\LaTeX\ 相比於字處理軟件MS Word之流最強大的功能之一。它可以大幅度優化你的文檔體積,用法是:
\begin{latex}
\newcommand{`\textit{cmd}`}[`\textit{args}`][`\textit{default}`]{`\textit{def}`}
\end{latex}

現在來解釋一下各個參數:
\begin{para}
\item[cmd:] 新定義的命令,不能與現有命令重名。
\item[args:] 參數個數。
\item[default:] 首個參數,即\texttt{\#{}1}的默認值。你可以定義只有一個參數、且參數含默認值的命令。
\item[def:] 具體的定義內容。參數1以\texttt{\#{}1}代替,參數2以\texttt{\#{}2}代替,以此類推。
\end{para}

如果重定義一個現有命令,使用\latexline{renewcommand}命令,用法與\latexline{newcommand}一致。簡單的例子:
\begin{latex}
% 加粗:\concept{text}
\newcommand{\concept}[1]{\textbf{#1}}
% 加粗#2並把#1#2加入索引,默認#1為空。
% 比如\cop{Sys}或者\cop[Sec.]{Sys}
\newcommand{\cop}[2][]{\textbf{#2}}\index{#1 #2}}
\end{latex}

如果想定義一個用於數學環境的命令,藉助\latexline{ensuremath}命令。它保證其參數會在數學模式下運轉, 且即使已位於數學模式中也不會報錯。
\begin{latex}
\renewcommand\qedsymbol{\ensuremath{\Box}}
\end{latex}

自定義環境的命令是\latexline{newenvironment},也可以定義多個參數。注意後段定義中不能使用參數,但你可以“先保存後調用”。例子:
\begin{latex}
\newenvironment{QuoteEnv}[2][]
    {\newcommand\Qauthor{#1}\newcommand\Qref{#2}}
    {\medskip\begin{flushright}\small ——~\Qauthor\\
    \emph{\Qref}\end{flushright}}
\end{latex}

下面是效果:
\begin{codeshow}
\begin{QuoteEnv}[William Butler]{When you are old}
But one man loved the pilgrim soul in you.
And loved the sorrows of your changing face.
\end{QuoteEnv}
\end{codeshow}

\section{箱子:排版的基礎}
\label{sec:box}

\begin{wrapfigure}{R}{0.4\textwidth}
\includegraphics{Texcharbox.pdf}
\caption{箱子的參數}
\label{fig:boxpara}
\end{wrapfigure}

\LaTeX\ 排版的基礎單位就是“箱子(box)”,例如整個頁面是一個矩形的箱子,側邊欄、主正文區、頁眉頁腳也都是箱子。在正常排版中,文字應當位於箱子內部;如果單行文字過長、沒能正確斷行,造成文字超出箱子,這便是Overfull的壞箱(bad box);如果內容太少,導致文字不能美觀地填滿箱子,便是Underfull的壞箱。

如\fref{fig:boxpara}所示,箱子的三個參數:高度(height)、寬度(width)和深度(depth)。分隔高度和深度的是基線。

\subsection{無框箱子}
命令\latexline{mbox}產生一個無框的箱子,寬度自適應。有時用它來強制“結合”一系列命令,使之不在中間斷行。比如\TeX\ 這個命令的定義(其中\latexline{raisebox}命令在後面介紹):
\begin{latex}
\mbox{T\hspace{-0.1667em}\raisebox{-0.5ex}{E}\hspace{-0.125em}X}
\end{latex}

或者也可以使用命令\latexline{makebox[width][pos]\{text\}},寬度由width參數指定。pos參數的取值可以是l, s, r即居左、兩端對齊、居右,還有豎直方向的t, b兩個參數。

無框小頁的使用方法是\texttt{minipage}環境,參數類似\latexline{parbox}:
\begin{latex}
\begin{minipage}[pos]{width}
\end{latex}

\subsection{加框箱子}
命令\latexline{fbox}產生加框的箱子,寬度自動調整,但不能跨行。命令\latexline{framebox}類似上面介紹的\latexline{makebox}。如果是想在數學環境下完成加框,使用\latexline{boxed}命令。

width參數中,可以用\latexline{width}, \latexline{height}, \latexline{depth}, \latexline{totalheight}分別表示箱子的自然寬度、自然高度、自然深度和自然高深度之和。

\begin{codeshow}
\fbox{This is a frame box} \\
\framebox[2\width]{double-width}\\
\begin{equation}\boxed{x^2=4}
\end{equation}
\end{codeshow}

加寬盒子的寬度、以及內容到盒子的距離可以自行定義。默認定義是:
\begin{latex}
\setlength{\fboxrule}{0.4pt} \setlength{\fboxsep}{3pt}
\end{latex}

加框小頁使用\envi{boxedminipage}環境(需要\pkg{boxedminipage}宏包)。

\subsection{豎直升降的箱子}
命令\latexline{raisebox}可以把文字提升或降低,它有兩個參數:

\begin{codeshow}
A\raisebox{-0.5ex}{n} example.
\end{codeshow}

\subsection{段落箱子}
段落箱子的強大之處在於它提供自動換行的功能,當然你需要指定寬度。
\begin{latex}
\parbox[pos]{width}{text}
\end{latex}

以及例子:

\begin{codeshow}
This is \parbox[t]{3.5em}{an long
example to show} how \parbox[b]
{4em}{`parbox' works perfectly}.
\end{codeshow}

\subsection{縮放箱子}
宏包\pkg{graphicx}提供了一種可縮放的箱子\latexline{scalebox\{h-sc\}[v-sc]\{pbj\}},注意其中水平縮放因子是必要參數。縮放內容可以是文字也可以是圖片,例子:
\begin{codeshow}
\LaTeX---\scalebox{-1}[1]{\LaTeX}\\
\LaTeX---\scalebox{1}[-1]{\LaTeX}\\
\LaTeX---\scalebox{-1}{\LaTeX}\\
\LaTeX---\scalebox{2}[1]{\LaTeX}
\end{codeshow}

此外還有\latexline{resizebox\{width\}\{heigh\}\{text\}}命令。

\subsection{標尺箱子}
命令\latexline{rule[lift]\{width\}\{height\}}能夠畫出一個黑色的矩形。你可以在單元格中使用width, height其一為0的該命令,作一個隱形的“支撐”來限定單元格的寬或高。而\latexline{strut}命令則用當前字號大小設置高度與深度。例如:

\begin{codeshow}
\begin{tabular}{|c|}
  \hline
  \rule[-1em]{1em}{1ex}text
  \rule{0pt}{38pt} \\
  \hline
  2nd text\strut--- \\
  \hline
\end{tabular}
\end{codeshow}

\subsection{覆蓋箱子}
有時候需要把一段文字覆蓋到另一段上面,使用\latexline{llap}或\latexline{rlap}。什麼?你從沒這麼幹過?但或許有一天你需要呢?

\begin{codeshow}
你看不清這些字\llap{是什麼}\\
\rlap{這些}你也看不清
\end{codeshow}

\subsection{旋轉箱子}
宏包\pkg{graphicx}提供了\latexline{rotatebox}命令,參數與插圖命令相同。
\begin{codeshow}
\rotatebox[origin=c]{90}{專}治頸椎病。
\end{codeshow}

\subsection{顏色箱子}
\label{subsec:colorbox}
\pkg{xcolor}宏包支持的顏色箱子命令有:

\begin{codeshow}
\textcolor{red}{紅色}強調\\
\colorbox[gray]{0.95}{淺灰色背景} \\
\fcolorbox{blue}{cyan}{%
\textcolor{blue}{藍色邊框+文字,
  青色背景}}
\end{codeshow}

命令\latexline{fcolorbox}可以調整\latexline{fboxrule, \char`\\fboxsep}參數,而\latexline{colorbox}只能調整後者。參考前面的加框箱子一節。

強大的\pkg{tcolorbox}宏包專門定義了眾多的箱子命令,參考\secref{subsec:tcolorbox}。

\section{複雜距離}
\label{sec:hvspace}
\subsection{水平和豎直距離}
長度單位參考\hyperref[sec:length]{這裏}介紹過的內容。水平距離命令有兩種,一種禁止在此處斷行,如\tref{tab:nobreak-hspace};另一種允許換行,如\tref{tab:break-hspace}。
\begin{table}[!htb]
\centering
\caption{禁止換行的水平距離}
\label{tab:nobreak-hspace}
\begin{tabular}{p{12em}p{8em}p{6em}}
  \latexline{thinspace}或\latexline{,} & 0.1667em & \rule{8pt}{2pt}\thinspace\rule[4pt]{8pt}{2pt} \\
  \latexline{negthinspace}或\latexline{!} & -0.1667em & \rule{8pt}{2pt}\negthinspace\rule[4pt]{8pt}{2pt} \\
  \latexline{enspace} & 0.5em & \rule{8pt}{2pt}\enspace\rule[4pt]{8pt}{2pt} \\
  \latexline{nobreakspace}或\char`~{} & 空格 & \rule{8pt}{2pt}\nobreakspace\rule[4pt]{8pt}{2pt}
\end{tabular}
\end{table}

\begin{table}[!htb]
\centering
\caption{允許換行的水平距離}
\label{tab:break-hspace}
\begin{tabular}{p{12em}p{8em}p{6em}}
  \latexline{quad}          & 1em           & \rule{8pt}{2pt}\quad\rule[4pt]{8pt}{2pt} \\
  \latexline{qquad}         & 2em           & \rule{8pt}{2pt}\qquad\rule[4pt]{8pt}{2pt} \\
  \latexline{enskip}        & 0.5em         & \rule{8pt}{2pt}\enskip\rule[4pt]{8pt}{2pt} \\
  \latexline{\textvisiblespace} & 空格 & \rule{8pt}{2pt}\ \rule[4pt]{8pt}{2pt}
\end{tabular}
\end{table}

使用\latexline{hspace\{length\}}命令自定義空格的長度,其中\textit{length}的取值例如:\texttt{-1em, 2ex, 5pt plus 3pt minus 1pt, 0.5\char92{}linewidth}等。如果想要這個命令在斷行處也正常輸出空格,使用帶星命令\latexline{hspace*}。

類似地使用\latexline{vspace}和\latexline{vspace*}命令,作為豎直距離的輸出。

要定義新的長度宏,使用\latexline{newlength}命令;要重設現有長度宏的值,可以選擇使用\latexline{setlength}命令;要調整長度宏的值,則使用\latexline{addtolength}命令。
\begin{latex}
\newlength{\mylatexlength}
\setlength{\mylatexlength}{10pt}
\addtolength{\mylatexlength}{-5pt}
\end{latex}

此外,\LaTeX\ 還定義了三個豎直長度\latexline{smallskip}, \latexline{medskip}, 和\latexline{bigskip}:

\begin{codeshow}
\parbox[t]{3em}{TeX\par TeX}
\parbox[t]{3em}{TeX\par\smallskip TeX}
\parbox[t]{3em}{TeX\par\medskip TeX}
\parbox[t]{3em}{TeX\par\bigskip TeX}
\end{codeshow}

\subsection{填充距離與彈性距離}
命令\latexline{fill}用於填充距離,需要作為\latexline{hspace}或\latexline{vspace}的參數使用。另外還有單獨使用的命令\latexline{hfill}與\latexline{vfill},作用相同。

彈性距離指以一定比例計算得到的多個空白,命令是\latexline{stretch}。例子:

\begin{codeshow}
Left\hspace{\fill}Right\\
Left\hspace{\stretch{1}}Center
\hspace{\stretch{2}}Right
\end{codeshow}

你還可以使用類似\latexline{hfill}的\latexline{hrulefill}和\latexline{dotfill}命令:

\begin{codeshow}
L\hfill R\\
L\hrulefill Mid\dotfill R
\end{codeshow}

\subsection{行距}
\LaTeX\ 的行距由基線計算,可以使用命令\latexline{linespread\{num\}},默認的基線距離\latexline{baselineskip}是1.2倍的文字高。所以默認行距是1.2倍;如果更改linespread為1.3,那麼行距變為$1.2\times 1.3=1.56$倍——這也是ctex文檔類的做法。

此外還有\latexline{lineskiplimit}和\latexline{lineskip}命令。有時候在兩行之間,可能包含較高的內容(比如分式$\dfrac{1}{2}$),使得前一行底部與後一行頂部的距離小於limit值,則此時行距會從由\latexline{linespread}改為由\latexline{lineskip}控制。本手冊採用:
\begin{latex}
\setlength{\lineskiplimit}{3pt}
\setlength{\lineskip}{3pt}
\end{latex}

\subsection{製表位*}
製表位使用\envi{tabbing}環境,需要指出,這是一個極其容易造成壞箱的環境。幾個要點:
\begin{para}
\item[\char92{}=] 在此處插入製表位。
\item[\char92{}>] 跳入下一個製表位。
\item[\char92{}\char92{}] 製表環境內必須手動換行和縮進。
\item[\char92{}kill] 若行末用\verb|\kill|代替\verb|\\|,那麼該行並不會被實際輸出到文檔中。
\end{para}

一個醜陋的例子:
\begin{codeshow}[listing and text]
\begin{tabbing}
\hspace{4em}\=\hspace{8em}\=\kill
製表位 \> 就是這樣 \> 使用的 \\
隨時 \> 可以添加 \> 新的: \= 就這樣 \\
也可以 \= 隨時重設 \= 製表位 \\
這是 \> 新的 \> 一行
\end{tabbing}
\end{codeshow}

\subsection{懸掛縮進*}
這種縮進在實際排版中並不常用,經常是列表需要的場合才使用,但那可以藉助列表宏包\pkg{enumitem}進行定義。這裏介紹的是正文中的懸掛縮進使用。

如果需要對單獨一段進行懸掛縮進,例如使用:
\begin{latex}
\hangafter 2
\hangindent 6em
\end{latex}

\hangafter 2
\hangindent 6em
這兩行放在某一段的上方,作用是控制緊隨其後的段落從第2行開始懸掛縮進,並且設置懸掛縮進的長度是\texttt{6em}。

如果需要對連續的多段進行懸掛縮進,可以改造編號列表環境或者\envi{verse}環境\footnote{事實上這是一個排版詩歌的環境,參考前文的\hyperref[envi:verse]{這裏}。}來實現。或者嘗試:

\begin{codeshow}
正文...

{\leftskip=3em\parindent=-1em
\indent 這是第一段。注意整體需要放在
一組花括號內,且花括號前應當有空白行
。第一段前需要加indent命令,最後一段
的末尾需額外空一行,否則可能出現異常。

這是第二段。

\ldots

這是最後一段。別忘了空行。

}
\end{codeshow}

\subsection{整段縮進*}
宏包\pkg{changepage}提供了一個\envi{adjustwidth}環境,它能夠控制段落兩側到文本區(而不是頁邊)兩側的距離。
\begin{latex}
\begin{adjustwidth}{1cm}{3cm}
本段首行縮進需要額外手工輸入。本環境距文本區左側1cm,右側3cm。
\end{adjustwidth}
\end{latex}

也可以嘗試賦值\latexline{leftskip}等命令,對奇偶頁處理更有效。

\section{自定義章節樣式}
\label{sec:titlesec}
這一節主要涉及\pkg{titlesec}宏包的使用。章節樣式調整使用\latexline{titlelabel},\latexline{titleformat*}命令。前者需要配合計數器使用,後者簡單地設置章節標題的字體樣式。例如:
\begin{latex}
\titlelabel{\thetitle.\quad}
\titleformat*{\section}{\itshape}
\end{latex}

章節樣式由標籤和標題文字兩部分構成。標籤一般表明了大綱級別以及編號,比如“第一章”、“Section 3.1”等。標題文字比如“自定義章節樣式”這幾個字。還記得嗎?在report與book類的subsection及以下,article類的paragraph及以下是默認沒有編號的。因此對應的級別也沒有標籤,除非人工進行設置。

對於需要詳細處理標籤、標題文字兩部分的情況,\pkg{titlesec}宏包還提供了一個\latexline{titleformat}命令。調用方式:
\begin{latex}
\titleformat{`\itshape command`}[`\itshape shape`]{`\itshape format`}{`\itshape label`}{`\itshape sep`}
    {`\itshape before-code`}[`\itshape after-code`]
\end{latex}

它們對應的含義如下:
\begin{para}
\item[command:] 大綱級別命令,如\latexline{chapter}等。
\item[shape:] 章節的預定義樣式,分為9種:
  \begin{para}
  \item[hang] 缺省值。標題在右側,緊跟在標籤後。
  \item[block] 標題和標籤封裝排版,不允許額外的格式控制。
  \item[display] 標題另起一段,位於標籤的下方。
  \item[runin] 標題與標籤同行,且正文從標題右側開始。
  \item[leftmargin] 標題和標籤分段,且位於左頁邊。
  \item[rightmargin] 仿上。右頁邊。
  \item[drop] 文本包圍標題。
  \item[wrap] 類似drop,文本會自動調整以適應最長的一行。
  \item[frame] 類似display,但有框線。
  \end{para}
\item[format:] 用於設置標籤和標題文字的字體樣式。這裏可以包含豎直空距,即標題文字到正文的距離。
\item[label:] 用於設置標籤的樣式,比如“第\latexline{chinese\char`\\thechapter}章”大概是ctexbook類的默認樣式設置。
\item[sep:] 標籤和標題文字的水平間距,必須是\LaTeX\ 的長度表達。當shape取display時,表示豎直空距;取frame時表示標題到文本框的距離。
\item[before:] 標題前的內容。
\item[after:] 標題後的內容。對於hang, block, display,此內容取豎向;對於runin, leftmargin, 此內容取橫向;否則此內容被忽略。
\end{para}

宏包還給出了\latexline{titlespacing}與\latexline{titlespacing*}兩個命令。使用方式是:
\begin{latex}
\titlespacing*{`\itshape command`}{`\itshape left`}{`\itshape before-sep`}{`\itshape after-sep`}[`\itshape right-sep`]
\titlespacing{`\itshape command`}{`\itshape left`}{`\itshape *m`}{`\itshape *n`}[`\itshape right-sep`]
\end{latex}

各參數的含義:
\begin{para}
\item[command:] 大綱級別命令,如\latexline{chapter}
\item[label:] 縮進值。在left/right margin下表示標題寬;在wrap中表示最大寬;在runin中表示標題前縮進的空距。
\item[before-sep:] 標題前的垂直空距。
\item[after-sep:] 標題與正文之間的空距。hang, block, display中是垂直空距;runin, wrap, drop, left/right margin中是水平空距。
\item[right-sep:] 可選。僅對hang, block, display適用。
\item[*m/*n:] 在\latexline{titlespacing}命令中的\textit{m, n}分別表示before與after sep的變動範圍倍數,基數是默認值。
\end{para}

宏包中還有一個 \latexline{titleclass} 命令,用來定義新的章節命令(例如 \verb|\subchapter|)或者重申明已有的章節命令:
\begin{latex}
% 使 \part 命令不單獨佔據一頁
\titleclass{\part}{top}
% 新定義一個 \subchapter 命令
\titleclass{\subchapter}{straight}[\chapter]
\newcounter{subchapter}
\renewcommand{\thesubchapter}{\Alph{subchapter}}
\end{latex}
其中,第二參數表示章節類型,可以是 \texttt{page}(獨佔一頁),\texttt{top}(另開新頁),或者 \texttt{straight}(普通)。

最後,宏包還給出了\latexline{titleline}命令,用來繪製填充整行、同時又嵌有其他對象的行。對象可以嵌入到左中右lcr三個位置。如果你只是想填充一行而不嵌入對象,使用\latexline{titlerule}及其帶星號的命令形式。
\begin{latex}
% 嵌入對象的線
\titleline[c]{CHAPTER 1}
% 單純填充一行
\titlerule[`\itshape height`]
\titlerule*[`\itshape width`]{`\itshape text`}
\end{latex}

最後,給出本手冊中的樣式定義,作為例子。這個例子稍微有些複雜,只使用到了\latexline{titleformat}相關的命令。
\begin{latex}
\newcommand{\chaformat}[1]{%
    \parbox[b]{.5\textwidth}{\hfill\bfseries #1}%
    \quad\rule[-12pt]{2pt}{70pt}\quad
    {\fontsize{60}{60}\selectfont\thechapter}}
% chapter樣式定義中的\chaformat以章名作為隱式參數
\titleformat{\chapter}[block]{\hfill\LARGE\sffamily}
    {}{0pt}{\chaformat}[\vspace{2.5pc}\normalsize
    \startcontents\printcontents{}{1}
    {\setcounter{tocdepth}{2}}]
\titleformat*{\section}{\centering\Large\bfseries}
\titleformat{\subsubsection}[hang]
    {\bfseries\large}{\rule{1.5ex}{1.5ex}}{0.5em}{}
\end{latex}

本例沒有定義subsection樣式。如果你想給subsection級別標號(即賦予它標籤),使用:\latexline{setcounter\{secnumdepth\}\{3\}}\footnote{report/book類part級別深度為0,遞增;article類part為-1,無chapter級別。故section及以下深度一致。}。

臨時更改\latexline{secnumdepth}可以生成不編號的章節,但章節名仍會被使用在目錄和\latexline{markboth}中——有時這比帶星號的章節命令更巧妙一些。

\section{自定義目錄樣式}
\label{sec:titletoc}
這一節主要涉及\pkg{titletoc}宏包,它與\pkg{titlesec}宏包的文檔寫在同一個pdf中。上節的例子(即本手冊Chapter)涉及\latexline{startcontents}與\latexline{printcontents}命令,旨在每一章的開始插入本章的一個目錄。

首先是目錄的標題,可以通過renewcommand更改。分別是 \latexline{contentsname}, \latexline{listfigurename}, \latexline{listtablename}三個。

再來看命令\latexline{dottecontents}與命令\latexline{titlecontents}:
\begin{latex}
\dottecontents{`\itshape section`}[`\itshape left`]{`\itshape above-code`}
    {`\itshape label-width`}{`\itshape leader-width`}
\titlecontents{`\itshape section`}[`\itshape left`]{`\itshape above-code`}{`\itshape numbered-entry-format`}
    {`\itshape numberless-entry-format`}{`\itshape filler-page-format`}[`\itshape below-code`]
\end{latex}

各參數的含義:
\begin{para}
\item[section:] 目錄對象。可以填chapter, section, 或者figure, table.
\item[left:] 目錄對象左側到左頁邊區的距離。請作必選使用。
\item[above-code:] 格式調整命令。可以包含垂直對象,也可以用\latexline{contentslabel},即指定本級別目錄標籤箱子的寬度。
\item[label-width:] 標籤寬。
\item[leader-width:] 填充符號寬。默認的填充符號是圓點。
\item[numered-entry-format:] 如果有標籤,則在目錄文本前輸入的格式。
\item[numberless-entry-format:] 如果沒有標籤輸入的格式。
\item[filler-page-format:] 填充格式。一般藉助\pkg{titlesec}中的\latexline{titlerule*}命令。
\item[below-code:] 在entry之後輸入的格式,比如垂直空距。
\end{para}

本手冊目錄樣式定義,其中section級別使用了填充命令\latexline{titlerule*}:
\begin{latex}
\titlecontents{chapter}[1.5em]{}{\contentslabel{1.5em}}
    {\hspace*{-2em}}{\hfill\contentspage}
\titlecontents{section}[3.3em]{}
    {\contentslabel{1.8em}}{\hspace*{-2.3em}}
    {\titlerule*[8pt]{$\cdot$}\contentspage}
\titlecontents*{subsection}[2.5em]{\small}
    {\thecontentslabel{}}{}
    {, \thecontentspage}[;\qquad][.]
\end{latex}

\section{自定義圖表}
\label{sec:figtab}
\subsection{長表格}
包括\pkg{supertabular}, \pkg{longtable}, \pkg{tabu}在內的多個宏包都能完成長表格的排版,大致的功能會包括:
\begin{para}
\item[表頭控制:] 首頁的表頭樣式,以及轉頁後表頭的樣式。
\item[轉頁樣式:] 在表格跨頁時,頁面最下方插入的特殊行,比如to be continued.
\end{para}

這裏主要介紹\pkg{longtable}宏包。主要命令:
\begin{para}
\item[\latexline{endhead}] 定義每頁頂端的表頭。在表頭行用該命令代替\latexline{\char`\\\char`\\}命令來換行即可。
\item[\latexline{endfirsthead}] 如果首頁的表頭與其他頁不同,使用該命令。
\item[\latexline{endfoot}] 定義每頁底端的表尾。
\item[\latexline{endlastfoot}] 另外定義末頁底端的表尾。
\item[\latexline{caption}] 與原生tabular的該命令一致。如果你不想顯示錶格編號,使用帶星的該命令;如果不想讓其加入表格目錄,在可選參數中留空\latexline{caption[]\{...\}}。
\item[\latexline{label}] 注意\verb|\label|命令不能被用在多頁對象中,請在表體中或者firsthead/lastfoot中使用。
\item[\latexline{LTleft}] 表格左側到主文本區邊緣的距離,默認是\latexline{fill}。你可以用:

\latexline{setlength\char`\\LTleft\{0pt\}}來取消這個距離,進行居左。
\item[\latexline{LTright}] 類似。
\item[\latexline{LTpre}] 表格上部到文本的距離,默認是\latexline{bigskipamount}\footnote{這個命令通常是一行左右的豎直距離,12pt$\pm$4pt左右。}。
\item[\latexline{LTpost}] 類似。
\item[\latexline{\char`\\ \char91\ldots\char93}] 在換行後插入豎直空距。
\item[\latexline{\char`\\{*}}] 禁止在該行後立刻進行分頁。
\item[\latexline{kill}] 該行不顯示,但用於計算寬度。
\item[\latexline{footnote(mark/text)}] 命令\latexline{footnote}不能用於表頭或表尾;在表頭和表尾中,使用\latexline{footnotemark}命令,並在表外用\latexline{footnotetext}寫明腳註內容。
\end{para}

\pkg{longtable}宏包支持的表格可選參數是clr,不能使用t或b. 此外,\RED{longtable中的跨列可能需要編譯多次才能正常顯示。}最後給出一個例子:

\begin{longtable}{@{*}r||p{3cm}@{*}}
KILLED & LINE! \kill

\caption[\texttt{longtable} Example]{This is an example}\\
\hline
\multicolumn{2}{@{}c@{}}{This is the headfirst\footnotemark}\\
First Col & Second Col \\
\hline\hline
\endfirsthead

\caption*{--Continued Longtable--}\\
\hline\hline
\multicolumn{2}{@{}c@{}}{This is the head of other page}\\
First Here & Second Here \\
\hline
\endhead

\hline\hline
This is the & bottom. \\
\hline
\endfoot

\hline
\textbf{That's all} & \textbf{and thanks}. \\
\hline
\endlastfoot

\footnotetext{Footnotemark: first footnote in table head.}
This is an example & and you can \\
see how longtable will & work. \\
Space after line are & allowed. \\[25ex]
You can adjust LTright and & LTleft \\
if you want to. I'd like & to set \\
LTleft as ``0pt'', but it all & depends on\\
you. And maybe you can try & footnote \\
like this\footnote{Footnote example.} and also
& footnotetext\footnotemark\footnotetext{Footnotetext example.}. \\
As for footnotemark, you've seen it & in the firsthead.\\[15ex]
And I think maybe it's long & enough to make \\
a table across pages, so go to the & next page and \\
check whether the head at next page is & different from \\
that on this page. Also you can have a look & at lastfoot. \\[20ex]
So do you get how to use this & package? \\
Maybe you'll love it. So enjoy & \texttt{longtable}!
\end{longtable}

它的代碼如下:
\begin{latex}
\begin{longtable}{@{*}r||p{3cm}@{*}}
KILLED & LINE! \kill

\caption[\texttt{longtable} Example]{This is an example}\\
\hline
\multicolumn{2}{@{}c@{}}{This is the headfirst\footnotemark}\\
First Col & Second Col \\
\hline\hline
\endfirsthead

\caption*{--Continued Longtable--}\\
\hline\hline
\multicolumn{2}{@{}c@{}}{This is the head of other page}\\
First Here & Second Here \\
\hline
\endhead

\hline\hline
This is the & bottom. \\
\hline
\endfoot

\hline
\textbf{That's all} & \textbf{and thanks}. \\
\hline
\endlastfoot

\footnotetext{Footnotemark: first footnote in table head.}
This is an example & and you can \\
see how longtable will & work. \\
Space after line are & allowed. \\[25ex]
You can adjust LTright and & LTleft \\
if you want to. I'd like & to set \\
LTleft as ``0pt'', but it all & depends on\\
you. And maybe you can try & footnote \\
like this\footnote{Footnote example.} and also
& footnotetext\footnotemark
\footnotetext{Footnotetext example.}. \\
As for footnotemark, you've seen it & in the firsthead.\\[15ex]
And I think maybe it's long & enough to make \\
a table across pages, so go to the & next page and \\
check whether the head at next page is & different from \\
that on this page. Also you can have a look & at lastfoot.
\\[20ex]
So do you get how to use this & package? \\
Maybe you'll love it. So enjoy & \texttt{longtable}!
\end{longtable}
\end{latex}

\subsection{\texttt{booktabs}:三線表}
\pkg{booktabs}宏包提供\latexline{toprule}, \latexline{midrule}與\latexline{bottomrule}命令來繪製三線表。更多需要的橫線可以通過\latexline{midrule}添加。
\begin{codeshow}
\begin{tabular}{cccc}
\toprule
& \multicolumn{3}{c}{Numbers} \\
\cmidrule{2-4}
& 1 & 2 & 3 \\
\midrule
Alphabet & A & B & C \\
Roman & I & II& III \\
\bottomrule
\end{tabular}
\end{codeshow}

命令\latexline{cmidrule}如果連續使用,還能寫成\latexline{cmiderule(lr)}的形式,使其向內縮進一小段,造成相互“斷開”的樣子。\dpar

\subsection{彩色表格}
彩色表格依靠\pkg{colortbl}宏包,它會調用\pkg{array}和\pkg{color}宏包。但是可以在加載\pkg{xcolor}宏包時添加table選項來調用\pkg{colortbl}宏包。

首先是命令\latexline{columncolor},給表格某列加背景色。其中mode參數是指rgb/cmyk等。left/right-ex參數表示向兩側填充的距離,默認是\latexline{tablecolsep}。
\begin{latex}
\columncolor[mode]{colorname}[left-ex][right-ex]
\end{latex}

命令\latexline{rowcolor}和\latexline{cellcolor}分別用於更改表頭行的顏色和單個單元格的顏色,放置在表格內對應位置即可。在\pkg{xcolor}支持下還可以使用\latexline{rowcolors}命令,但放在表格開始之前:
\begin{latex}
% 表線為單橫,從第2行開始,奇數行綠,偶數行青
\rowcolors[\hline]{2}{green}{cyan}
\begin{tabular}...
\end{latex}

要臨時開關奇偶行顏色,使用\latexline{show/hide rowcolors}命令。

彩色表格中跨行,需要把跨行命令放在最後一行,並跨負數行:
\begin{codeshow}
\rowcolors{2}{green}{cyan}
\begin{tabular}{ll}
\hline Col 1 & Col 2\\
& A\\ \multirow{-2}*{Hey} & B\\
\hline
\end{tabular}
\end{codeshow}

\subsection{子圖表}
子圖表輸出用\pkg{subcaption}宏包,它需要與\pkg{caption}宏包共同加載。比如:
\begin{latex}
\usepackage{caption,subcaption}
  \captionsetup[sub]{labelformat=simple}
  \renewcommand{\thesubtable}{(\alph{subtable})}
% 用\ref引用得到如“圖1.1(a)”的效果
\begin{table}
\caption{Parents}
\begin{subtable}[b]{0.5\linewidth}
  \centering
  \begin{tabular}{|c|c|}
  A & B \\ \end{tabular}
  \caption{First}\label{...}
\end{subtable}  
\begin{subtable}[b]{0.5\linewidth}
  \centering
  \begin{tabular}{|c|c|}
  A & B \\ C & D \end{tabular}
  \caption{Second}
\end{subtable}  
\end{table}
\end{latex}

效果如\tref{subtab:subcaption1}與\tref{subtab:subcaption2}。更多的請參考\pkg{caption}宏包。
\begin{table}[!htb]
\caption{Parents}
\begin{subtable}[b]{0.5\linewidth}
  \centering
  \begin{tabular}{|c|c|}
  A & B \\ \end{tabular}
  \caption{First}\label{subtab:subcaption1}
\end{subtable}  
\begin{subtable}[b]{0.5\linewidth}
  \centering
  \begin{tabular}{|c|c|}
  A & B \\ C & D \end{tabular}
  \caption{Second}\label{subtab:subcaption2}
\end{subtable}  
\end{table}

\subsection{GIF 動態圖}
使用 \pkg{animate} 宏包(當然,\pkg{graphicx} 宏包也是需要的),可以將多張圖片以動態圖的形式插入 PDF。需要注意的是,\textbf{動態圖在一些功能較弱的 PDF 瀏覽器中可能無法正常工作},推薦使用 Adobe 系列 PDF 瀏覽器以保證正常瀏覽。代碼如下:
\begin{latex}
\begin{figure}[!hbt]
  \centering
  \animategraphics[controls, autoplay, loop,
  width=0.6\linewidth]{20}{Py3-matplotlib-}{0}{98}
\end{figure}
\end{latex}

以上代碼對應的動態圖\footnote{該例由 Python - matplotlib 繪製。可以參考\href{https://wklchris.github.io/Py3-matplotlib.html}{此頁面}的附錄。}給出如\fref{fig:GIF}所示:
\begin{figure}[!hbt]
  \centering
  \animategraphics[controls, autoplay, loop, width=0.6\linewidth]{20}{Py3-matplotlib-}{0}{98}
  \caption{動態圖示例}\label{fig:GIF}
\end{figure}

以上會搜索文件夾(包括你在 \pkg{graphicx} 中設置的文件夾),找到圖片依次序編號的從“Py3-matplotlib-0.png”到“Py3-matplotlib-98.png”的這99張圖片,以每秒20幀為默認播放速度加載。參數 \texttt{controls} 表示在圖片下方附加控制按鈕,可以暫停/播放,正放/倒放,手動瀏覽幀,以及更改播放速度。參數 \texttt{autoplay} 表示當閲讀者瀏覽到動態圖所在頁面時,動態圖會自動開始播放。參數 \texttt{loop} 表示播放到尾幀後自動重播。最後,你可以像一般圖片加載一樣,指定它的 \texttt{width/height}。

注意:如果你只有GIF圖像,但安裝了\href{https://www.imagemagick.org/script/download.php}{ImageMagick},可以在圖像文件夾下使用命令行命令:

\begin{verbatim}
convert Py3-matplotlib.gif -coalesce Py3-matplotlib.png
\end{verbatim}

來將單個 GIF 轉為符合上述要求的多個 png 圖像。

\section{自定義編號列表}
\label{sec:list}
編號列表的自定義主要使用\pkg{enumitem}宏包。主要的計數器有:
\begin{feai}
\item enumerate:
  \begin{feai}
    \item \textbf{Counter:} enumi, enumii, enumiii, enumiv
    \item \textbf{Label:} labelenumi, labelenumii, \ldots
  \end{feai}
\item itemize: 只有Label,該列表沒有Counter。 
  \begin{feai}
    \item \textbf{Label:} labelitemi, labelitemii, \ldots
  \end{feai}
\item description: 只有\latexline{descriptionlabel}定義,默認:
\begin{latex}
\newcommand*{\descriptionlabel}[1]{\hspace\labelsep
    \normalfont\bfseries #1} % \labelsep標籤間距,默認0.5em
\end{latex}
\end{feai}

列表\envi{enumerate}, \envi{itemize}的默認參數見\tref{tab:enumitemdes}。

\begin{table}[!hbt]
\centering
\caption{編號列表默認參數表}
\label{tab:enumitemdes}
\renewcommand{\arraystretch}{0.9}
\begin{tabular}{@{}c@{\,}c*{2}{>{\small}l@{}>{\ttfamily\char`\\}l}@{}}
\hline
環境 & 層 & Label & \multicolumn{1}{l}{默認} & Counter & \multicolumn{1}{l}{默認}	\\
\hline
\multirow{4}*{\envi{enumerate}} & 1 & \latexline{labelenumi} & theenumi. & \latexline{theenumi} & arabic\{enumi\}\\
& 2 & \latexline{labelenumi} & (theenumii) & \latexline{theenumii} & alph\{enumi\} \\
& 3 & \latexline{labelenumiii} & theenumiii. & \latexline{theenumiii} & roman\{enumi\} \\
& 4 & \latexline{labelenumiv} & theenumiv. & \latexline{theenumiv} & Alph\{enumiv\} \\
\hline
\multirow{4}*{\envi{itemsize}} & 1 & \latexline{labelitemi} & \multicolumn{3}{p{.5\textwidth}}{\ttfamily\char`\\ textbullet \dotfill\fbox{\textbullet}} \\
& 2 & \latexline{labelitemii} & \multicolumn{3}{p{.5\textwidth}}{\ttfamily\char`\\ textendash \dotfill\fbox{\textendash}}  \\
& 3 & \latexline{labelitemiii} & \multicolumn{3}{p{.5\textwidth}}{\ttfamily\char`\\ textasteriskcentered \dotfill\fbox{\textasteriskcentered}} \\
& 4 & \latexline{labelitemiv} & \multicolumn{3}{p{.5\textwidth}}{\ttfamily\char`\\ textperiodcentered \dotfill\fbox{\textperiodcentered}} \\
\hline
\end{tabular}
\end{table}

在\envi{enumerate}列表中,編號樣式按照:1.$\rightarrow$(a)$\rightarrow$i.$\rightarrow$A的順序嵌套,分別代表\latexline{theenumi}, \latexline{theenumii}, \latexline{theenumiii}, \latexline{theenumiv}的值。你可以通過計數器命令來指定編號樣式,不過要額外加上一個星號,比如\latexline{arabic*}表示阿拉伯數字。一個例子:
\begin{codeshow}
\begin{enumerate}\item First
  \begin{enumerate}\item Second
     \begin{enumerate}\item Third
       \begin{enumerate}
       \item Fourth Layer
\end{enumerate}\end{enumerate}
\end{enumerate}\end{enumerate}
% 改為首層小寫羅馬數字,放於圓括號
\renewcommand{\theenumi}
  {\roman{enumi}}
\renewcommand{\labelenumi}
  {(\theenumi)}
\begin{enumerate}
\item First-layer symbol has changed!
\end{enumerate}
\end{codeshow}

你也可以在\pkg{ctex}宏包被調用(包括ctex文檔類被使用)時,在導言區加入:
\begin{latex}
\AddEnumerateCounter{\chinese}{\chinese}{}
\end{latex}

這樣就可以將漢字指定為編號樣式了。

宏包\pkg{enumitem}可添加參數於列表後,像\verb|\begin{list}[options]|:
\begin{para}
\item[label] 定義\envi{enumerate}環境的編號樣式,或者\envi{itemize}環境的符號樣式。
\item[ref] 設置嵌套序號格式,比如\texttt{[ref=\char`\\emph\{\char`\\alph*\}]}表示引用的上層序號是強調後的小寫字母。你也可以寫:\texttt{[label=\char`\\alph\{enumi\}.\ \char`\\roman*]}。
\item[label*] 加在\envi{enumerate}上層序號上。比如上層是2,那麼就是2.1, 2.1.1……
\item[font/format] 設置label的字體。如果環境是description, 那麼就會設置\latexline{item}命令後方括號內的文本字體。
\item[align] 對齊方式默認right, 也可以選擇left/parleft.
\item[start] 初始序號。start=2表示初始序號是2, b, B, ii或II.
\item[resume] 不需賦值的布爾參數。表示接着上一個\envi{enumerate}環境的結尾進行編號。
\item[resume*] 不需賦值的布爾參數。表示完全繼承上一個\envi{enumerate}環境的參數。如果你常常使用這個命令,也許你可以新定義一個列表環境。
\item[series] 給當前列表起名(比如mylist),可以在後文中用\texttt{resume=mylist}進行繼續編號。
\item[style] 定義\envi{description}列表的樣式。
\begin{para}
\item[standard:] label放在盒子中。
\item[unboxed:] label不放在盒子中,避免異常長度或空格。
\item[nextline:] 如果label過長,text會另起一行。
\item[sameline:] 無論label多長,text從label同一行開始。
\item[multiline:] label會被放在一個寬為leftmargin的parbox中。
\end{para}
\end{para}

在列表定義中可能碰到的參數如\fref{fig:enumitemsep}。其中加粗加斜的\LaTeX\ 不原生支持。
\begin{figure}[!hbt]
\includegraphics[width=0.9\linewidth]{enumitemsep.pdf}
\caption{列表長度參數總圖}
\label{fig:enumitemsep}
\end{figure}

\fref{fig:enumitemsep}中的豎直空距topsep, partopsep, parsep, itemsep,以及水平空距left/rightmargin, listparindent, labelwidth, labelsep, itemindent都是可以直接以\texttt{key=value}的形式寫在列表環境後做參數的。

命令\latexline{setlist},用於定義列表環境的樣式。比如可以更改原有的列表:
\begin{latex}
\setlist[enumerate]{label=\arabic* -,
    font=\bfseries, itemsep=0pt}
\setlist[itemize]{label=$\bullet$,
    font=\bfseries,leftmargin=\parindent}
\setlist[description]{font=\bfseries\uline}
\end{latex}

最後,説一下行內列表。在加載\pkg{enumitem}宏包時使用inline選項即可啓用,環境名是\envi{enumerate*}. 參數有:
\begin{para}
\item[before] 在行內列表插入前的文本,一般是冒號。
\item[itemjoin] 各\latexline{item}之間的文本,一般是逗號或者分號。
\item[itemjoin*] 倒數第二個與最後一個\latexline{item}間的文本,一般是``, and''或者“,還有”之類。
\end{para}

幾個小例子。\envi{description}環境:

\begin{codeshow}
\begin{description}
[font=\bfseries\uline]
    \item[This] is BFSERIES.
    \item[And] this also.
\end{description}
\end{codeshow}

編號數字左端與左頁邊平齊:
\begin{latex}
\begin{enumerate}[leftmargin=*]
\end{latex}

\noindent\begin{boxedminipage}{\linewidth}
\setlength{\parindent}{2em}
Here we go. This is a very long sentence and you will find that it goes to the second line in order to show how long its parindent is.
\begin{enumerate}[leftmargin=*]
\item The left sides
\item of the label number
\item have equal indent with
\item the text parindent.
\end{enumerate}
\end{boxedminipage}
\dpar

編號數字左端與段首縮進位置平齊:
\begin{latex}
\begin{enumerate}[labelindent=\parindent,leftmargin=*]
\end{latex}

\noindent\begin{boxedminipage}{\textwidth}
\setlength{\parindent}{2em}
Here we go. This is a very long sentence and you will find that it goes to the second line in order to show how long its parindent is.
\begin{enumerate}[labelindent=\parindent,leftmargin=*]
\item The left sides
\item of the label number
\item have equal indent with
\item the text parindent.
\end{enumerate}
\end{boxedminipage}
\dpar

編號項目正文與段首縮進位置平齊:
\begin{latex}
\begin{enumerate}[leftmargin=\parindent,start=3]
\end{latex}

\noindent\begin{boxedminipage}{\textwidth}
\setlength{\parindent}{2em}
Here we go. This is a very long sentence and you will find that it goes to the second line in order to show how long its parindent is.
\begin{enumerate}[leftmargin=\parindent,start=3,]
\item An item can be extremely long. You cannot know how its parindent works if it is too short to reach the second line.
\item This is short.
\end{enumerate}
\end{boxedminipage}
\dpar

標籤加框:
\begin{latex}
\begin{enumerate}[label=\fbox{\Roman*},labelindent=\parindent]
\end{latex}

\noindent\begin{boxedminipage}{\textwidth}
\setlength{\parindent}{2em}
Here we go. This is a very long sentence and you will find that it goes to the second line in order to show how long its parindent is.
\begin{enumerate}[label=\fbox{\Roman*},labelindent=\parindent]
\item An item can be extremely long. You cannot know how its parindent works if it is too short to reach the second line.
\item This is short.
\end{enumerate}
\end{boxedminipage}
\dpar

最後,本手冊使用瞭如下5種:
\begin{latex}
\begin{description}[font=\bfseries\uline,labelindent=\parindent,
    itemsep=0pt,parsep=0pt,topsep=0pt,partopsep=0pt]
\begin{description}[font=\bfseries\ttfamily,itemsep=0pt,
    parsep=0pt,topsep=0pt,partopsep=0pt]
\begin{enumerate}[font=\bfseries,labelindent=0pt,itemsep=0pt,
    parsep=0pt,topsep=0pt,partopsep=0pt]
\begin{itemize}[font=\bfseries,itemsep=0pt,parsep=0pt,
    topsep=0pt,partopsep=0pt]
% 行內列表定義
\newenvironment{inlinee}
    {\begin{enumerate*}[label=(\arabic*), font=\rmfamily,
    before=\unskip{:}, itemjoin={{;}}, itemjoin*={{,以及:}}]}
    {\end{enumerate*}。}
\end{latex}

\section{\bibtex\ / \biber\ 參考文獻}
\label{sec:bibtex}

首先説一下基礎的使用。通過重定義\latexline{refname}或\latexline{bibname},前者是article類,後者是book類。這點在\secref{subsec:cite}這一節已經介紹過。

關於怎樣將參考文獻正常編號並加入目錄中,請參考\secref{subsec:cite}這一節。

\begin{latex}
\renewcommand{\bibname}{參考文獻}
\end{latex}

在文獻目錄之前、文獻標題之下,用\latexline{bibpreamble}插入一段文字:
\begin{latex}
\renewcommand{\bibpreamble}{以下是參考文獻:}
\end{latex}

用\latexline{bibfont}更改參考文獻的字體:
\begin{latex}
\renewcommand{\bibfont}{\small}
\end{latex}

用\latexline{citenumfont}定義在正文中引用時,文獻編號的字體:
\begin{latex}
\renewcommand{\citenumfont}{\itshape}
\end{latex}

用\latexline{bibnumfmt}定義文獻目錄的編號,默認是[1], \ldots 形式。比如改成加點形式:
\begin{latex}
\renewcommand{\bibnumfmt}[1]{\textbf{#1.}}
\end{latex}

文獻項之間的間距更改,調整\latexline{bibsep}即可:
\begin{latex}
\setlength{\bibsep}{1ex}
\end{latex}

\subsection{\texttt{natbib}宏包}
個人認為文獻宏包首推\pkg{natbib},不再推薦\pkg{cite}. \pkg{natbib}宏包的加載選項:
\begin{para}
\item[round] (默認)圓括號。
\item[square/curly/angle] 方括號/花括號/尖括號。
\item[semicolon/comma] 分號/逗號作為文獻序號分隔符。
\item[authoryear] “作者+年代(AuY)”模式顯示參考文獻。
\item[numbers] “數字編號(num)”模式顯示參考文獻。
\item[super] 參考文獻顯示在上標。
\item[sort(\&compress)] 排序文獻序號(並壓縮\footnote{壓縮是指:連續三個或以上的序號會顯示為如2--4的形式。})。
\item[compress] 壓縮但不排序。
\item[longnamefirst] 長名稱在前,縮寫名稱在後。
\item[nonamebreak] 防止作者名稱中間出現斷行。可能造成Overfull壞箱,但能解決某些hyperref異常。
\item[merge] 允許*形式的引用。
\item[elide] 在merge選項引用中,省略相同的作者或年份。
\end{para}

你也可以通過宏包提供的\latexline{setcitestyle}命令:
\begin{feae}
\item 引用模式:authoryear, number與super三種,含義同上。
\item 引用分隔符:semicolon, comma, 或者用citesep=\{\textit{sep}\}來指定。
\item 作者與年代間的符號:aysep=\{\textit{sep}\}
\item 同作者下多個年代間的符號:yysep=\{\textit{sep}\}
\item 説明文字後的符號:notesep=\{\textit{sep}\}
\end{feae}

默認的參數是:
\begin{latex}
\setcitestyle{authoryear,round,comma,aysep={;},
    yysep={,},notesep={, }}
\end{latex}

除了\LaTeX\ 原生的\latexline{cite}命令,\pkg{natbib}宏包還提供了\tref{tab:natbib}所示的引用命令。

\begin{table}[!hbt]
\centering
\caption{\texttt{natbib}宏包命令表}
\label{tab:natbib}
\small
\begin{tabular}{l|>{\ttfamily}l!{$\Rightarrow$}l}
\hline
\multicolumn{3}{c}{使用\latexline{Citet}, \latexline{Citep}, \latexline{Citealt}, \latexline{Citealp}確保姓名首字母大寫} \\
\hline
\multicolumn{3}{l}{\latexline{citet} \& \latexline{citet*}} \\
\hline
\multirow{4}*{AuY}
& citet\{jon90\} & Jones et al. (1990) \\
& citet[chap.2]\{jon90\} & Jones et al. (1990, chap.2) \\
& citet\{jon90, jam91\} & Jones et al. (1990); James et al. (1991) \\
& citet*\{jon90\} & Jones, Baker, and Williams (1990) \\
\cline{2-3}
\multirow{2}*{num}
& citet\{jon90\} & Jones et al. [21] \\
& citet[chap.2]\{jon90\} & Jones et al. [21, chap.2] \\
\hline

\multicolumn{3}{l}{\latexline{citep} \& \latexline{citep*}} \\
\hline
\multirow{6}*{AuY}
& citep\{jon90\} & (Jones et al., 1990) \\
& citep[chap.2]\{jon90\} & (Jones et al., 1990, chap.2) \\
& citep[see][]\{jon90\} & (see Jones et al., 1990)\\
& citep[see][chap.2]\{jon90\} & (see Jones et al., 1990, chap.2)\\
& citep\{jon90, jon91\} & (Jones et al., 1990, 1991) \\
& citep*\{jon90\} & (Jones, Baker, and Williams, 1990) \\
\cline{2-3}
\multirow{5}*{num}
& citep\{jon90\} & [21] \\
& citep[chap.2]\{jon90\} & [21, chap.2] \\
& citep[see][]\{jon90\} & [see 21] \\
& citep[see][chap.2]\{jon90\} & [see 21, chap.2]\\
& citep{jon90a,jon90b} & [21, 32] \\
\hline

\multicolumn{3}{l}{\latexline{cite}} \\
\hline
AuY & \multicolumn{2}{c}{此模式下與\latexline{citet}相同} \\
\cline{2-3}
num & \multicolumn{2}{c}{此模式下與\latexline{citep}相同} \\
\hline\hline

\multicolumn{3}{l}{\latexline{citealt}: 與\latexline{citet}相似,但沒有括號。} \\
\hline
\multicolumn{3}{l}{\latexline{citealp}: 與\latexline{citep}相似,但沒有括號。} \\
\hline
\multicolumn{3}{l}{\latexline{citenum}: 引用文獻編號。} \\
\hline
\multicolumn{3}{l}{\latexline{citetext}: 打印一段文本。} \\
\hline
\multicolumn{3}{l}{\latexline{citeauthor} \& \latexline{citeauthor*}: 引用文獻的作者。帶星表示顯示該文獻的全部作者。} \\
\hline
\multicolumn{3}{l}{\latexline{citeyear} \& \latexline{citeyearpar}: 引用文獻的年份。par的意思是在文獻外加括號。} \\
\hline
\end{tabular}
\end{table}

\subsection{\bibtex 使用}
\bibtex 通過單獨的.bib擴展名文件管理文獻,使多文檔方便地共用一份文獻列表(可以指引用其中的部分文獻)成為可能。使用時請確保:
\begin{feae}
\item 確保你的文檔定義了\latexline{bibliographystyle}類型。\LaTeX\ 預定義的類型分為:
  \begin{para}
    \item[plain] 按照第一作者字母順序排序。“\textit{作者. 文獻名. 出版商或刊物, 出版地, 出版時間.} ”
    \item[unsrt] 按引用順序排序。
    \item[alpha] 按作者名稱和出版年份排序。
    \item[abbrv] 縮寫形式。
  \end{para}
\item 在文檔中插入了\latexline{cite}等命令。
\item 在參考文獻列表位置插入了\latexline{bibliography}命令\footnote{如果你在正文中使用\latexline{nocite\{ref-name\}}命令,可以把bib文件中未cite的文獻也加入到列表。想全部加入,在正文中使用\latexline{nocite\{*\}}命令。}。
\end{feae}

一個通用的\bibtex\ 使用方式:
\begin{latex}
\bibliographystyle{plain}
\begin{document}
    ...
    ... and published here\cite{Smith93TRB}.
    ...
    \bibliography{myBib}
\end{document}
\end{latex}

然後在你的myBib.bib文件中,你需要有類似這樣的條目。等號後使用花括號或引號均可。
\begin{latex}
% 如果引用期刊
@article{Smith1993TRB,
    author = {作者, 多個作者用and 連接},
    title = {標題},
    journal = {期刊名},
    number = {頁碼},
    year = {年份}}
% 如果引用書籍
@book{Smith1993TRB,
    author ="作者",
    year="年份2008",
    title="書名",
    publisher ="出版社名稱"}
\end{latex}

一些其他的注意事項:
\begin{feai}
  \item \textbf{通常你無須手動書寫 bib 內容}。許多文獻檢索頁面(比如 Google 學術)都支持導出\bibtex 文本,將其粘貼到你的 \texttt{.bib} 文件中即可。
  \item 除了上例介紹的文獻類型。更多的文獻類型以及它們的使用條目選項,參考\tref{tab:bibtype}。一個可能用到的場景是引用網頁,但它並沒有專門的類型。作為參考,我一般用\texttt{misc}類型,並在其\texttt{note}鍵中標出訪問時間。
  \item 等號後的花括號表示直接輸出,不通過 \bibtex 自動轉換大小寫。這常用於強制字母大小寫的場合,比如\texttt{title=\{UpperCaseVar\}}。
\end{feai}

\begin{table}[!htb]
\centering
\caption{\bibtex 文獻類型常用表}
\label{tab:bibtype}
\begin{tabular}{>{\ttfamily}ll}
\hline
article & 期刊文獻 \\
& 必要:author, title, journal, year \\
& 選填:volume, number, pages, month, note \\
\hline
book & 公開出版圖書 \\
& 必要:author/editor, title, publisher, year \\
& 選填:volume/number, series, address, edition, month, note \\
\hline
booklet & 無出版商或作者的圖書。必要:title\\
& 選填:author, howpublished, address, month, year, note \\
\hline
conference/ & 無出版商或作者的圖書。必要:title \\
inproceedings & 選填:author, howpublished, address, month, year, note \\
\hline
incollection & 書籍含獨立標題的章節,比如論文集的一篇 \\
& 必要:author, title, booktitle, publisher, year \\
& 選填:editor, volume/number, series, type, chapter, pages, \\
& address, edition, month, note \\
\hline
manual & 技術手冊。必要:title \\
& 選填:author, organization, address, edition, month, year, note \\
\hline
mastersthesis & 碩士論文\\
& 必要:author, title, school, year \\
& 選填:type, address, month, note\\
\hline
misc & 其他 \\
& 選填:author, title, howpublished, month, year, note \\
\hline
phdthesis & 博士論文 \\
& 必要:author, title, year, school\\
& 選填:address, month, keywords, note\\
\hline
techreport & 教育,商業機構的技術報告\\
& 必要:author, title, institution, year\\
& 選填:type, number, address, month, note\\
\hline
unpublished & 未出版的論文或圖書\\
& 必要:author, title, note\\
& 選填:month, year\\
\hline
\end{tabular}
\end{table}

最後,你可能需要\RED{編譯\xelatex , 再編譯\bibtex , 最後連續編譯兩次\xelatex 來完成你的文檔建立}。

\section{索引}
使用\pkg{makeidx}宏包來建立索引。索引標題通過重定義\latexline{indexname}更改。
\begin{feae}
\item 在導言區加載\pkg{makeidx}宏包,並輸入\latexline{makeindex}開始收集索引。
\item 在文中使用\latexline{index}命令來插入索引標籤。
\item 在需要插入索引列表的位置輸入\latexline{printindex}。
\end{feae}

索引命令\latexline{index}的用法如\tref{tab:index}。注意:這四種符號\verb+!|@"+如果要寫在參數中,請在它們之前添加一個雙引號。
\begin{table}
\centering
\tabcaption{索引命令\texttt{\char92 index}的使用}
\label{tab:index}
\begin{tabular}{>{\ttfamily}ll}
\hline
\multicolumn{1}{l}{\textbf{例子}} & \textbf{效果} \\
\hline
\multicolumn{2}{l}{!: 分級索引,最多三級} \\
hello & hello, 1 \\
hello!Foo & \hspace{1em}Foo, 2 \\
hello!Foo!bar & \hspace{2em}bar, 3 \\
\hline
\multicolumn{2}{l}{@: 格式化,“排序字串@顯示樣式”}\\
alpha@\$\char92 alpha\$ & $\alpha$, 4 \\
BOLD@\char92 textbf\{BOLD\} & \textbf{BOLD}, 5 \\
\hline
\multicolumn{2}{l}{|: 頁碼顯示}\\
wow|( & \multirow{2}*{wow, 6--13} \\
wow|) & \\ 
Meow|textbf & Meow, \textbf{14} \\
Meow|see\{hello\} & Meow, \textit{see} hello \\
Meow|seealso\{wow\} & Meow, \textit{see also} wow \\
\hline	
\end{tabular}
\end{table}

此外,\pkg{imakeidx}宏包可能更強,它允許索引分組:
\begin{latex}
\makeindex[title={Group 1}]
\makeindex[title={Group 2},name=another]
% 以上在導言區,且需要\usepackage{imakeidx}
    ...\index{...}
    ...\index[another]{...}
\printindex
\printindex[another]
\end{latex}

定製索引樣式可使用\pkg{imakeidx}宏包;另一個宏包\pkg{idxlayout}也能實現這些功能,不過需要放置在前者之後加載。

使用 \pkg{tocbibind} 宏包將索引章節正常編號或編入目錄項,參考\secref{pkg:tocbibind}部分。

關於索引,部分用户有製作詞彙表的需求,請參考\pkg{glossary}宏包。

\section{公式與圖表編號樣式}
\subsection{取消公式編號}
取消單行公式的編號,用\latexline{[\char`\\]}或者\envi{equation*}環境,代替\envi{equation}環境。

取消多行公式中某行的編號,使用\pkg{amsmath}宏包的\latexline{notag}或\latexline{nonumber}命令。命令放在對應行的末尾即可。該方法同樣適用於\envi{equation}環境。

取消多行公式中所有行的編號,請使用\envi{align*}環境而不是\envi{align}環境。你可以參考\secref{subsec:multieqnum}內容。

\subsection{增加公式編號}
\pkg{amsmath}宏包提供了增加編號的\latexline{tag}命令:

\begin{codeshow}
\[a^2>0 \tag{$\star$}\]
\begin{equation}
b^2 \geqslant 0
\tag*{[Axiom]}
\end{equation}
\end{codeshow}

其中\latexline{tag*}命令會去掉編號行間公式的小括號,從而定製性更強。如果想在\RED{多行公式中的某行}添加編號,使用\latexline{numberthis}命令。

\subsection{父子編號:公式1與公式1a}
有時你需要敍述一些推論,你不希望這些推論被編號為公式,但是不進行編號又難以敍述。這時可以嘗試\pkg{amsmath}宏包提供的\envi{subequations}環境:

\begin{codeshow}
\begin{subequations}
This is an upright text.
\begin{align}
A' &=B+C \\
X &=0 \nonumber \\
D' &=E \times F
\end{align}
Enjoy this environment.
\end{subequations}
\end{codeshow}

父子編號樣式的定義參考\hyperref[code:parenteqnum]{這裏}。還有一種利用計數器的方式,可以插入公式1和公式1'這樣的效果,但是實現起來稍顯麻煩。參考\hyperref[code:eq1plus]{這裏}。

\subsection{在新一節重新編號公式}
只需要\latexline{numberwithin}進行設置:
\begin{latex}
\numberwithin{equation}{section}
\end{latex}

對於article文檔類,顯示為(2.1), (2.2), \ldots ;對於chapter等文檔類,顯示為(1.2.1), (1.2.2), \ldots 這樣。

\subsection{公式編號樣式定義}
通過控制計數器的方式可以方便地自定義公式編號樣式:
\begin{latex}
% 更改為:(2-i), (2-ii)
\renewcommand{\theequation}{\thechapter-\roman{equation}}
% 父子公式編號樣式,在subequation環境內使用
% 效果:(4.1-i), (4.1-ii)
\renewcommand{\theequation}`\label{code:parenteqnum}`
    {\theparentequation-\roman{equation}}
\end{latex}

公式1和公式1'的實現方法可以這樣:\label{code:eq1plus}
\begin{latex}
\begin{equation}\label{eq:example}
    A=B,\ B=C
% 需要標記的公式內部給myeq計數器賦值
\setcounter{myeq}{\value{equation}}
\end{equation}
插入另一個式子:
\begin{equation}
    D>0
\end{equation}
由式\ref{eq:example}可以推出式\ref{eq:example'}:
\begin{equation}\label{eq:example'}
    \tag{\arabic{chapter}.\arabic{myeq}$'$}
    A=C
\end{equation}
\end{latex}

\section{附錄}
\label{sec:appendix}
附錄可以直接在文中需要開啓的地方,使用下述命令。此後的最高大綱級別編號會變為大寫英文字母:A, B, C, \ldots
\begin{latex}
\appendix
\end{latex}

你也可以使用\pkg{appendix}宏包。加載時常用的選項:
\begin{para}
\item[titletoc:] 目錄中顯示為``Appendix A''而不是隻有一個``A''。如果你不喜歡Appendix這個名稱,用重定義\latexline{appendixname}命令即可。
\item[header:] 在附錄頁的標題前插入同樣的名稱。
\end{para}

附錄一般出現在\envi{document}環境內部的最後,例如:
\begin{latex}
% 導言區:\usepackage[titletoc]{appendix}
\begin{appendices}
\renewcommand{\thechapter}{\Alph{chapter}}
\titleformat{\chapter}[display]{\Huge\bfseries}
    {附錄\Alph{chapter}}{1em}{}
\chapter{...} ...
\end{appendices}
\end{latex}

本手冊使用的是\latexline{appendix}命令,並在其後重定義了chapter的顯示樣式。讀者可以從源碼中參考。

\section{自定義浮動體*}
可以使用\pkg{newfloat}宏包自定義浮動體。形如:
\begin{latex}
\DeclareFloatingEnvironment[`\textit{options}`]{`\textit{float-name}`}
\end{latex}

選項包括:
\begin{para}
\item[name] 標籤內容。如name=插圖。
\item[listname] 目錄名。如listname=插圖目錄。
\item[fileext] 目錄文件擴展名,默認是\textit{float-name}前加lo。
\item[placement] 位置參數htbp。
\item[within] 父計數器名稱。比如chapter. 可以設為none.
\item[chapterlistsgaps] 賦值on/off. 是否允許浮動體目錄中,不同章節的浮動體間有額外的空距。
\end{para}

要輸出浮動體目錄,請插入命令\latexline{listof[float-name]s}。

\section{編程代碼與行號*}
\label{sec:coding}
\subsection{\texttt{listings}宏包}
編程代碼不是用\envi{verbatim}環境輸出的……\pkg{listings}宏包是個好選擇。你可以打開它的宏包文檔查看它支持的編程語言,包括C/C++, Python, Java等等。當然還有\LaTeX,如果它也算編程代碼的話。你也可以自定義一個全新的語言。預定義或自定義的語言均可用\latexline{lstset}命令來設置。
\begin{latex}
\lstdefinelanguage{`\textit{languagename}`}{`\textit{key=value}`}
% 設置
\lstset{language=`\textit{languagename}`, `\textit{key=value}`}
\end{latex}

可調整的參數包括:
\begin{para}
\item[language] 用於\latexline{lstset}中,表示設置只對該語言生效。該參數有可選參數,比如[Sharp]C表示C\#,具體需要查看listings宏包。
\item[basicstyle] 基礎輸出格式,一般是\latexline{small\char`\\ttfamily}
\item[commentstyle] 註釋樣式。
\item[keywordstyle] 保留詞樣式。
\item[sensitive] 保留詞是否大小寫敏感。默認false,可選true/false。
\item[stringstyle] 字符串樣式。
\item[showstringspaces] 顯示字符串中的空格。
\item[numbers] 行號樣式,默認是none. 可選left/right.
\item[stepnumber] 默認是1,即每行都顯示行號。
\item[numberfirstline] 默認false,即如果stepnumber$>1$,首行不顯示數字。
\item[numberstyle] 行號樣式。
\item[numberblanklines] 默認true,即在空白的代碼行也顯示行號。
\item[firstnumber] 默認auto,可選last或填入數字,表示起始行號。
\item[frame] 默認single,即trbl,分別表示top, right, buttom, left四條邊的線都是單線。如果想變某邊為雙線,大寫它:trBL.
\item[frameround] 在\latexline{lstset}中,從右上角順時針設置,代碼框為直角或圓角。比如fttt表示僅右上為直角。
\item[framerule] 代碼框的線寬。
\item[backgroundcolor] 定義frame裏代碼的背景色,如\latexline{color\{red\}}。
\item[belowskip] 默認\latexline{medskipamount}。代碼框下端到下文正文的豎直距離。
\item[aboveskip] 類似。
\item[columns] 設置行內單詞間距的處理方式。fixed 選項會強行按列對齊,可能產生字母覆蓋問題;space-flexible 選項會調整現有空距嘗試對齊列;flexible 選項可能在原本無空距的地方插入空距來嘗試對齊;而 fullflexible 選項(本文使用)則完全不管列對齊。
\item[emptylines] 設置最多允許的空行,比如=1,會使多於1行的空行全部刪除,並不計入行號。如果寫為=*1,被刪除的行的行號仍然會保留計數。
\item[esacpeinside] 暫時脱離代碼環境而輸入一些\LaTeX\ 支持的命令,比如臨時輸入斜體。一般設置為一對重音符號(鍵盤數字1左側的符號)。
\end{para}

代碼環境的調用方式是\envi{lstlisting}環境,例如:
\begin{latex}
\begin{lstlisting}[language=Python]
for loopnum in lst:
    sum += lst[loopnum]
\end{lstlisting}
\end{latex}

也可以利用\latexline{lstnewenvironment}命令定義一個代碼輸出環境:
\begin{latex}
\lstnewenvironment{`\textit{envi-name}`}
    [`\textit{opt}`][`\textit{opt default}`]
    {`\textit{before-code}`}{`\textit{after-code}`}
% 調用:
\begin{envi-name}...\end{envi-name}
\end{latex}

如果只想在行內輸出,可以像這樣用\latexline{lstinline}定義:
\begin{latex}
\newcommand{\inlatexline}[1]{{\lstinline
    [language=TeX,basicstyle=\small\ttfamily]{#1}}}
% 要防止“#include”雙寫“#”,請調用 xparse 宏包:
\NewDocumentCommand{\cppline}{v}{{\lstinline
    [language=C++]{#1}}}
\end{latex}

有時候,一些關鍵詞並沒有被宏包成功高亮,或者你需要更多種類的高亮方式,這時候你可以自己設置:
\begin{latex}
\lstset{language=..., classoffset=0,
    morekeywords={begin,end},
    keywordstyle=\color{brown},
    classoffset=1,...}
\end{latex}

除了morekeywords外,你可以使用:
\begin{latex}
\lstdefinestyle{...}{
    morecomment=[l]{//}, % 單行註釋
    morecomment=[s]{/*}{*/}, % 多行註釋,不可嵌套
    morecomment=[n]{(*}{*)}, % 多行註釋,可嵌套
    morestring=[b]", % 字符串
% 如果想在字符串內輸出該符,前加反斜槓即可
\end{latex}

代碼環境在複製的時候怎麼才能不復制前面的行號呢?在導言區加上,
\begin{latex}
\usepackage{accsupp}
\newcommand{\emptyaccsupp}[1]
    {\BeginAccSupp{ActualText={}}#1\EndAccSupp{}}
% 在lstset的numberstyle中加入
...numberstyle=...\emptyaccsupp...,
\end{latex}

\RED{並請用Adobe Reader等功能健全的PDF閲讀打開pdf}!如果是Sumatra仍然可能會選中前面的行號。

最後,給出本手冊舊版中使用的\LaTeX\ 的\latexline{lstset},顏色另行定義:
\begin{latex}
% \emptyaccsupp 來自前文利用宏包 accsupp 的自定義
\lstset{language=[LaTeX]TeX,
    basicstyle=\small\ttfamily,
    commentstyle=\color{commentcolor},
    keywordstyle=\color{keywordcolor},
    stringstyle=\color{stringcolor},
    showstringspaces=false,
    % Package/Tikz-Lib Using
    classoffset=0,
    morekeywords={begin,end,usetikzlibrary},
    keywordstyle=\color{keywordcolor},
    classoffset=1,
    morekeywords={article,report,book,xeCJK,tikz,calc},
    keywordstyle=\color{packagecolor},
    classoffset=2,
    morekeywords={document,tikzpicture},
    keywordstyle=\color{envicolor},
    % Line Number Style
    numbers=left,stepnumber=1,
    numberstyle=\tiny\emptyaccsupp,    
    % Frame and Background Color
    frame=single,framerule=0pt,    
    backgroundcolor=\color{backcolor},
    % Spaces
    emptylines=1,escapeinside=`\texttt{\char96{}\char96{}}`}
\end{latex}

\subsection{\texttt{tcolorbox}宏包}
\label{subsec:tcolorbox}
本手冊在修訂過程中發現了一個可以方便地“一側寫源代碼,另一側展示結果”的宏包,名叫\pkg{tcolorbox};因此基本用其替換了原有的\pkg{listings}宏包——但是引擎仍然使用的是\pkg{listings},在使用新宏包時也需要藉助舊宏包選項來設置引擎。

該宏包支持下,本手冊舊版使用的\latexline{newtcblisting}定義了\LaTeX\ 代碼環境:
\begin{latex}
\usepackage{tcolorbox}
  \tcbuselibrary{listings,skins,breakable}
% listings是代碼展示引擎,breakable為了可跨頁
\newtcblisting{latex}{breakable,skin=bicolor,colback=gray!30!white,
  colbacklower=white,colframe=cyan!75!black,listing only, 
  left=6mm,top=2pt,bottom=2pt,fontupper=\small,
  % listing style
  listing options={style=tcblatex,
  keywordstyle=\color{blue},commentstyle=\color{green!50!black},
  numbers=left,numberstyle=\tiny\color{red!75!black}\emptyaccsupp,
  emptylines=1,escapeinside=``}}
\end{latex}

其中很多選項意義顯然,就不贅述了。需要指明的有:
\begin{para}
\item[skin] bicolor,讓源代碼和顯示結果可以分開設置背景色。
\item[colbacklower] 背景色。\texttt{tcolorbox}分為兩段,下段(或右段)叫lower。
\item[fontupper] 上段(或左段)叫upper,這是設置在進入upper前插入的格式命令,不侷限於字號。
\item[代碼展示參數] 該參數經常用到的是:
    \begin{para}
      \item[listing only] 僅展示源代碼。也有text only選項。
      \item[listing and text] 上段源代碼,下段結果。本手冊的\envi{codeshowabove}環境採用該參數。如果text與listing交換,即上段結果下段源代碼。
      \item[listing side text] 左段源代碼,右段結果。本手冊的\envi{codeshow}環境就採用了該參數。同樣也可以交換,變成左段結果右段源代碼。
      \item[listing outside text] 同上,只是結果“看起來”在盒子外。
    \end{para}
\item[listing option] 除了特殊的style字段,這些參數都會被傳遞給引擎(本手冊是listings宏包)。\texttt{style=tcblatex}是\pkg{tcolorbox}宏包預定義的。
\end{para}

該環境也支持使用可選參數,用法類似原生 \latexline{newcommand} 命令。但是,如果你只有一個參數,而且它是可選的,請使用 \latexline{NewTCBListing} 命令(需要用 \latexline{tcbuselibrary} 命令 加載 \texttt{xparse} 庫)以避免可能的問題。例如,利用如下命令\footnote{\TeX\ Live 2019版的宏包目前存在一個問題,在定義單個參數且其為可選參數時,需要在字母O前添加感嘆號。}可以定義一個左側源碼、右側排版結果的展示環境:
\begin{latex}
% 調用可無參或單參,例:\begin{codeshow}
% 或 \begin{codeshow}[listing and text]
\NewTCBListing{codeshow}{ !O{listing side text} }{
  skin=bicolor,colback=gray!30!white,
  colbacklower=pink!50!yellow,colframe=cyan!75!black,
  valign lower=center,
  left=6mm,righthand width=0.4\linewidth,fontupper=\small,
  % listing style
  listing options={style=latexcn},#1
}
\end{latex}
其中,\texttt{O} 這種用法請參考 \pkg{xparse} 宏包文檔,常用的有必選參數“m”、可選參數“o”、可選參數並指定缺省值“O\{<default>\}”。

通過該宏包的\latexline{newtcbox}命令,本手冊實現了對於命令、環境、宏包的高亮。以下給出本手冊中對宏包名稱的高亮作為例子,參數含義顯然:
\begin{latex}
\newtcbox{\pkg}[1][orange!70!red]{on line,
    before upper={\rule[-0.2ex]{0pt}{1ex}\ttfamily},
    arc=0.8ex,colback=#1!30!white,colframe=#1!50!black,
    boxsep=0pt,left=1.5pt,right=1.5pt,top=1pt,bottom=1pt,    boxrule=1pt}
\end{latex}

實際上\pkg{tcolorbox}的強大之處遠不止此,它能做出顏值很高的箱子樣式。更多的內容請自行查閲其宏包文檔學習。最後,附上\pkg{tcolorbox}預定義的 \texttt{tcblatex} 的 \texttt{style}選項\footnote{該定義位於\texttt{texmf-dist/tex/latex/tcolorbox/tchlistings.code.tex}中。},是用 \TeX\ 編寫的,讀者可以與本手冊源碼中的定義進行對比:
\begin{latex}
\lstdefinestyle{tcblatex}{language={[LaTeX]TeX},
     aboveskip={0\p@ \@plus 6\p@}, belowskip={0\p@ \@plus 6\p@},
     columns=fullflexible, keepspaces=true,
     breaklines=true, breakatwhitespace=true,
     basicstyle=\ttfamily\small, extendedchars=true, nolol,
     inputencoding=\kvtcb@listingencoding}
\end{latex}


\subsection{行號}
\begin{linenumbers}
\modulolinenumbers[3]
行號使用\pkg{lineno}宏包進行生成,在此簡單介紹宏包選項:
\begin{para}
  \item[left]默認選項,行號出現在左頁邊。
  \item[right]右頁邊。
  \item[switch]對於雙頁排版的文檔,偶數頁左頁邊,奇數頁右頁邊。
  \item[switch*]對於雙頁排版的文檔,置於內側頁邊。
  \item[running]默認選項。整個文檔進行計數。
  \item[pagewise]每頁行號重新從1計數。
  \item[modulo]每5行顯示行號。
  \item[displaymath]自動將\LaTeX\ 默認行間公式放在新定義的\envi{linenomath}環境中。
  \item[mathline]如果使用了\envi{linenomath}環境,則對其中的數學公式也編行號。
\end{para}

如果你想開始編號,使用\latexline{linenumbers[number]}命令,其中\textit{number}表示起始行號;並用\latexline{nolinenumbers}來結束。或者選擇使用\envi{(running)linenumbers}環境,並且仿上設置起始行號。如果使用\latexline{linenumbers*}, \latexline{runninglinenumbers}或者\envi{linenumbers*}、\envi{runninglinenumbers*}環境,那麼行號會自動從1開始。

你可以使用\latexline{resetlinenumber[number]},在某處把行號設置為某個數值。

如果想要每N行顯示行號,使用\latexline{modulolinenumbers[N]}命令。比如:
\end{linenumbers}

\begin{latex}
\begin{linenumbers}
    \modulolinenumbers[3]
    ...TEXT...
\end{linenumbers}
\end{latex}

本節的subsection後到代碼段前的所有內容就是包含在如上的環境中的。